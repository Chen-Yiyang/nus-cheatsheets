%%%%%%%%%%%%%%%%%%%%%%%%%%%%%%%%%%%%%%%%%%%%%%%%%%%%%%%%%%%%%%%%%%%%%%
% Original Source: Dave Richeson (divisbyzero.com), Dickinson College
% Modified By: Chen Yiyang
% 
% A one-size-fits-all LaTeX cheat sheet. Kept to two pages, so it 
% can be printed (double-sided) on one piece of paper
% 
% Feel free to distribute this example, but please keep the referral
% to divisbyzero.com
% 
% Guidance on the use of the Overleaf logos can be found here:
% https://www.overleaf.com/for/partners/logos 
%%%%%%%%%%%%%%%%%%%%%%%%%%%%%%%%%%%%%%%%%%%%%%%%%%%%%%%%%%%%%%%%%%%%%%

\documentclass[10pt,landscape,letterpaper]{article}
\usepackage{amssymb}
\usepackage{amsmath}
\usepackage{amsthm}
\usepackage{physics} % for vectors
%\usepackage{fonts}
\usepackage{multicol,multirow}
\usepackage{spverbatim}
\usepackage{graphicx}
\usepackage{ifthen}
\usepackage[landscape]{geometry}
\usepackage[colorlinks=true,urlcolor=olgreen]{hyperref}
\usepackage{booktabs}
\usepackage{fontspec}
\setmainfont[Ligatures=TeX]{TeX Gyre Pagella}
\setsansfont{Fira Sans}
\setmonofont{Inconsolata}
\usepackage{unicode-math}
\setmathfont{TeX Gyre Pagella Math}
\usepackage{microtype}

\usepackage{empheq}

% new:
\def\MT@is@uni@comp#1\iffontchar#2\else#3\fi\relax{%
  \ifx\\#2\\\else\edef\MT@char{\iffontchar#2\fi}\fi
}
\makeatother

\ifthenelse{\lengthtest { \paperwidth = 11in}}
    { \geometry{margin=0.4in} }
	{\ifthenelse{ \lengthtest{ \paperwidth = 297mm}}
		{\geometry{top=1cm,left=1cm,right=1cm,bottom=1cm} }
		{\geometry{top=1cm,left=1cm,right=1cm,bottom=1cm} }
	}
\pagestyle{empty}
\makeatletter
\renewcommand{\section}{\@startsection{section}{1}{0mm}%
                                {-1ex plus -.5ex minus -.2ex}%
                                {0.5ex plus .2ex}%x
                                {\sffamily\large}}
\renewcommand{\subsection}{\@startsection{subsection}{2}{0mm}%
                                {-1explus -.5ex minus -.2ex}%
                                {0.5ex plus .2ex}%
                                {\sffamily\normalsize\itshape}}
\renewcommand{\subsubsection}{\@startsection{subsubsection}{3}{0mm}%
                                {-1ex plus -.5ex minus -.2ex}%
                                {1ex plus .2ex}%
                                {\normalfont\small\itshape}}
\makeatother
\setcounter{secnumdepth}{0}
\setlength{\parindent}{0pt}
\setlength{\parskip}{0pt plus 0.5ex}


\usepackage{academicons}



% Declare math
\DeclareMathOperator*{\argmin}{argmin}
\DeclareMathOperator*{\argmax}{argmax}



% -----------------------------------------------------------------------
\begin{document}

\definecolor{mathBlue}{cmyk}{1,.72,0,.38}
\definecolor{defOrange}{cmyk}{0, 0.5, 1, 0.3}
\definecolor{codeInlineRed}{cmyk}{0, 0.9, 0.9, 0.45}

\everymath{\color{mathBlue}}
\everydisplay{\color{mathBlue}}

% for vector notation in this module
\newcommand{\vect}[1]{\vec{#1}}
\newcommand{\deff}[1]{\textcolor{defOrange}{\textbf{#1}}}
\newcommand{\codein}[1]{\textcolor{codeInlineRed}{\texttt{#1}}}
\newcommand{\citeqn}[1]{\underline{\textit{#1}}}

\footnotesize
%\raggedright

\begin{center}
  {\huge\sffamily\bfseries CS4261 Cheatsheet} \huge\bfseries\\
  by Yiyang, AY22/23
\end{center}
\setlength{\premulticols}{0pt}
\setlength{\postmulticols}{0pt}
\setlength{\multicolsep}{1pt}
\setlength{\columnsep}{1.8em}
\begin{multicols}{3}


% -----------------------------------------------------------------------

\section{Nash Equilibrium}
A \deff{Game} is a general abstract framework for strategic interactions, with usually 1) A set of \textbf{Players} $N = \{1, ..., n \}$, and subsequently for each player $i \in N$, 2) A set of possible \textbf{Actions} $A_i = \{a_{i1}, a_{i2}... \}$, and 3) a \textbf{Utility Function} $u_i : A \mapsto \mathbb{R}$, which indicates the utility Player $i$ can get from an action profile, then lastly 4) a \deff{(Pure) Action Profile} that denotes the actions taken by all the players $\vec{a} \in A_1 \times A_2 \times ... \times A_n = A$.
\\
A \deff{Normal Form Game} is a Matrix Representation of the player utilities for a 2-Player game. Conventionally, each element in the Normal Form Game is a pair of real values $C_{ij} = (u, v) \in \mathbb{R}^2$ where $u$ and $v$ are the utility of the row and column player given an action profile $\vec{a} = (a_{1, i}, a_{2, j})$.


\subsection{Pure Nash Equilibrium}
Given actions taken by everyone else $\vec{a}_{-i}$, the \deff{Best Response} set of Player $i$ is defined
\[
BR_i(\vec{a}_{-i}) = \{ b \in A_i | b \in \text{argmax }  u_i(\vec{a}_{-i}, b) \}
\]
\\
An action profile is a \deff{Pure Nash Equilibrium} if it is one best response for everyone given what others have chosen, i.e. 
\[
\forall i \in N, a_i \in BR_i(\vec{a}_{-i})
\]
\underline{Analysis}: Not all games have Pure Nash Equilibria.


\subsection{Mixed Nash Equilibrium}
Let $\vec{p} \in \Delta(A_i)$ be the probability distribution over Player $i$'s actions. A \deff{Mixed / Randomised Strategy Profile} is given by $\vec{p} = (\vec{p}_1, ..., \vec{p}_n) \in \Delta(A_1) \times ... \times \Delta(A_n)$. Player utility is $u_i(\vec{p}) = \sum_{\vec{a} \in A} u_i(\vec{a}) P(\vec{a}) = \mathbb{E} _{\vec{a} \sim \vec{p}} [u_i(\vec{a})]$.
\\
A mixed profile is a \deff{Mixed Nash Equilibrium} if 
\[
\forall i \in N, \vec{q}_i \in \Delta(A_i), \ u_i(\vec{p}) \ge u_i(\vec{p}_{-1}, \vec{q}_i)
\]
\underline{Analysis}: All games have Mixed Nash Equilibria.

\smallskip

In a Mixed Nash Equilibrium, a player is \deff{Indifferent} if he gets the same \textbf{expected utility} from choosing any action (as a result of the other player playing a mixed strategy).

\subsubsection{Compute Nash Equilibria in 2 Player Games}
\begin{itemize}
    \item Compute all NE in which at least one player plays a pure strategy.
    \item Compute all NE in which both players play mixed strategies. In this case, each player must be  indifferent between the two strategies.
\end{itemize}


\subsection{Dominant Strategies}
A strategy $\vec{p} \in \Delta(A_i)$ \deff{Dominates} $\vec{q} \in \Delta(A_i)$ if
\[
\forall \vec{p}_{-i} \in \Delta(A_{-i}), \ u_i(\vec{p}_{-i}, \vec{p}) \ge u_i(\vec{p}_{-i}, \vec{q})
\]
There are similar definitions for \deff{Strictly Domination}.
\\
\underline{Intuitively}, no mather what others do, playing $\vec{p}$ is always better than $\vec{q}$.

\subsubsection{Dominant Strategy Theorem}
If an action $a \in A_i$ is \textbf{strictly dominated} by some strategy $\vec{p} \in \Delta(A_i)$, then action $a$ is never played with any positive probability in any Nash Equilibrium.
\\
\underline{Note}: The theorem enables us to prune actions that will not occur in any Nash Equilibria.





\section{Auction}
\subsection{Single-Item Auction}
Types of Single-Item Auction
\begin{itemize}
    \item \deff{English Auction} - Auctioneer sets a starting price. Bidders take turn raising their bids. The bidder makes the last bid wins and pay his bid.
    \item \deff{Japanese Auction} - Auctioneer sets a starting price and raises it. A bidder can drop out and not return once dropped. The last standing bidder gets the item and pays the current price.
    \item \deff{Vickrey / Second-Price Auction} - All bidders submit bids simultaneously. The highest bidder wins and pays the second highest price.
\end{itemize}

\subsubsection{Vickrey Auction Problem Specification}
There are $n$ players $N = \{ 1, 2, ..., n\}$, each with a valuation of the item $v_i$. The actions are to place a bid at different prices. The payoff for a player is $v - p$ if getting the item, and $0$ otherwise.

\subsubsection{Analysis}
Vickrey Auctions are \deff{Truthful}, i.e. bidding according to one's true valuation is a dominant strategy.
\\
First-Price Auctions are \textbf{Not Truthful}.
\\
\underline{Note}: Dominant strategies are Nash Equilibria in Auction games, but they are not necessarily the only Nash Equilibria.



\subsection{Multi-Unit Auction}
There are $n$ players $N = \{ 1, 2, ..., n\}$, each with a valuation of the item $v_i$. There are $k \le n$ identical copies of the item.

\smallskip

The objective is to design a mechanism where
\begin{itemize}
    \item Truthful bidding is a dominant strategy
    \item Items are allocated to the $k$ highest bidders
\end{itemize}

\subsubsection{Vickrey Clarke Groves (VCG) Mechanism}
\begin{enumerate}
    \item Choose some outcome $o^*$ that maximises social welfare $\sum_{i} v_i(o^*)$
    \item Calculate the payment that Player $j$ must take with $p_j = \sum_{i \neq j} v_i(o_{-j}^*) - \sum_{i \neq j} v_i(o^*)$, where $o_{-j}^*$ is the outcome that maximises $\sum_{i \neq j} v_i(o_{-j})$.
\end{enumerate}
\underline{Note}: The payment for each Player is essentially the \deff{Externality} that he imposes on other players, which is the difference in the max welfare of others between if he is absent and if present.
\\
\underline{Analysis}: VCG is truthful. Vickrey Auction is a special case of VCG.



\subsection{Combinatorial Auction}
There are $n$ Players and $m$ possibly distinct items for sale. Each player has a valuation for each subset of the $m$ objects.

\smallskip

VCG is truthful, but it can be computationally intensive, and suffers from \deff{Revenue Non-Monotonicty}, a paradox where adding more players in the bidding game may lead to a decrease in the \deff{Revenue}, i.e. sum of all players' payment $R = \sum_{i=1}^n p_i$.
\\
\underline{Note}: Single-Item Auctions have no revenue non-monotonicity.





\section{Facility Location}
\subsection{Overview}
There are $n$ players $N = \{1, ..., n\}$, each with a location $x_i \in \mathbb{R}$, assuming $x_1 \le x_2 \le ... \le x_n$ for convenience.

\smallskip

The objective is to design $f : \mathbb{R}^n \mapsto \mathbb{R}$ that minimises either of
\begin{itemize}
    \item \deff{Total Cost} - $\sum_{i \in N} |f(\vec{x}) - x_i|$ 
    \item \deff{Max Cost} - $\max_{i \in N} |f(\vec{x}) - x_i|$ 
\end{itemize}

\underline{Analysis}: OPT for Total Cost is \textbf{Not Truthful}. OPT for Max Cost is \textbf{Truthful} if it always "snaps" to a median player.


\subsection{Max Cost Approximation Theorems}
\subsubsection{Deterministic Case}
Any deterministic truthful mechanism for facility location has a worst-case approximation ratio $\le 2$ to the maximum cost.

\subsubsection{Randomised Case}
Any randomised truthful mechanism for facility location has a worst-case approximation ratio $\le \frac{3}{2}$ to the maximum cost.





\section{Routing Games}
In a traffic network, players are drivers trying to find a route that minimises their total traffic time.
\begin{itemize}
    \item \deff{Proportion Version} There is $1$ unit of traffic to allocate in total. Drivers are considered proportion of the total traffic.
    \item \deff{Atomic Version} The traffic consists of $k \in \mathbb{N}$ drivers, each being an atomic entity.
\end{itemize}

\smallskip

\deff{Price of Anarchy} (PoA) is the ratio of the social cost under the worst case Nash Equilibrium and under socially optimal solution
\[
PoA = \frac{\text{WorstNash}(G)}{OPT(G)}
\]
\underline{Analysis}: 1) $PoA \ge 1$ with the smaller being the better. 2) All Nash Equilibria in a Routing Game have the same social cost.


\subsection{Atomic Version}
\subsubsection{Atomic Routing Game Theorem}
In an atomic routing game, a pure NE flow always exists.

\subsubsection{Higher Level Idea}
Every atomic routing game is a potential game, where all players are inadvertently and collectively optimising a potential function, $\Phi(f)$,
\[
\Phi(f) = \sum_{e \in E} \sum_{i=1}^{f_e} c_e(i)
\]
\underline{Analysis}: When a player deviates (changes path), change in the deviator’s individual cost is equal to $\triangle \Phi $. “Alignment in individual and social objective”.



\section{Cooperative Games}
\subsection{Definitions}
A \deff{Cooperative Game} $\mathcal{G}(N, v)$ consists of a set of players $N = \{ 1, ..., n \}$, and a valuation function $v: 2^n \to \mathbb{R}_{\ge 0 }$ for each player. A \deff{Coalition Structure} (CS) is a partition of $N$ while $\text{OPT}(G) = \max_\text{CS} \sum_{S \in \text{CS}} v(S)$ is the optimal.

\subsubsection{Properties, Cooperative Games}
\deff{Monotone} - For all $S \subseteq T \subseteq N$, $v(S) \le v(T)$.
\\
\deff{Simple} - Monotone and for all $S \subseteq N$, $v(S) \in \{ 0, 1\}$.
\\
\deff{Super-additive} - For any disjoint $S, T \subseteq N$, $v(S) + v(T) \le v(S \cup T)$.
\\
\deff{Convex} - For $S \subseteq T \subseteq N$ and any $i \in N \backslash T$, $v(S \cup \{ i \}) - v(S) \le v(T \cup \{ i \}) - v(T)$.


\subsubsection{Properties, Game Payoff}
\deff{Imputation} - An efficient \& individual rational payoff allocation $\vect{x}$.
\\
$\vect{x}$ is \deff{Efficient} if satisfying $\sum_{i \in N} x_i = v(N)$.
\\
$\vect{x}$ is \deff{Individual Rational} if $x_i \ge v(i)$ for all $i \in N$.


\subsection{The Core}
An imputation $\vect{x}$ is in the \deff{Core} if it satisfies for all $S \subseteq N$,
\[
\sum_{i \in S} x_i = x(s) \ge v(S)
\]
\underline{Note}: A core is a set of vectors, not a set of players.

\subsubsection{Properties, the Core}
Assume the game $\mathcal{G} = (N, v)$ is simple:
\\
\deff{Winning} / \deff{Losing Coalition} - A coalition with value $1$ / $0$.
\\
\deff{Veto Player} - a player that is in every winning coalition.
\\
\underline{Analysis}: For $p$ veto, any coalition without $p$ cannot win, and any wit $p$ does not necessarily win.

\subsubsection{Veto Player Theorem}
For a simple game, $\text{Core}(\mathcal{G}) \neq \emptyset$ iff. $\mathcal{G}$ has veto players.
\\
The core only distribute payoffs among the veto players.

\subsection{Shapley Value}
For Player $i$ and $S \subseteq N$, the \deff{Marginal Contribution} of $i$ to $S$ is
\[
m_i(S) = v(S \cup \{i\}) - v(S)
\]
Given a permutation $\sigma \in \Pi(N)$, the \deff{Predecessors} of $i$ in $\sigma$ are
\[
P_i(\sigma) = \{ j \in N | \sigma(j) < \sigma(i) \}
\]
We can write $m_i(\sigma) = m_i(P_i(\sigma))$ for marginal contribution.
\\
\deff{Shapley Value} of a player in a Coalition is his expected marginal contribution.
\[
Sh_{i} = \mathbb{E} \Big[ m_i(\sigma) \Big] 
= \frac{1}{n!} \sum_{\sigma \in \Pi(N)} m_i(\sigma)
\]

\subsubsection{Shapley Value Theorem}
Shapley Value is the only payoff allocation value satisfying efficiency, linearity, dummy and symmetry.
\\
\underline{Note}: [1] \deff{Symmetry}: symmetric players are paid equally. [2] \deff{Dummy}: Dummy players are not paid.



\subsection{Induced Subgraph Games}
An example of Cooperative Games $\mathcal{G} = (V, E)$ where players are vertices and $e = (u, v) \in E$ with weight $w_e$ describing the utility of $u$ and $v$ in a coalition.

\subsubsection{Induced Subgraph Game Core Theorem}
The core of an induced subgraph game is non-empty iff. the graph has no negative cut.
\\
\underline{Note}: A negative cut is a cut (set of edges that parition the graph into two) where the sum of edge weights is negative.

\subsubsection{Shapley Value, Induced Subgraph Game}
The payoff for each player i will be
\[
\phi_i = \frac{1}{2} \sum_{j \in N} w(i, j)
\]




\section{Nash Bargaining Solution}
\subsection{Definitions}
A \deff{Bargaining Game} is a pair $(S, \vec{d})$ for $S \subseteq \mathbb{R}^2$ and $\vec{d} \in \mathbb{R}^2$ with at least one point $(x_1, y_1) \in S$ such that $x_1 \ge d_1$ and $y_1 \ge d_2$ for $\vec{d} = (d_1, d_2)$. \textbf{Two players} chooses $x, y \in \mathbb{R}$ respectively. If $(x, y) \in S$, the two players receives $x$ and $y$ respectively. Otherwise, they receive $d_1$ and $d_2$. A \textbf{solution} is a function $\vec{f} = (f_1, f_2)$ that takes in $(S, \vec{d})$ and outputs a value for the two players each.

\subsubsection{Pareto Optimality}
An outcome $(x_1, x_2)$ \deff{Pareto Dominates} another outcome $(x_2, y_2)$ if $x_1 \ge x_2$ and $y_1 \ge y_2$ and at least one of these two inequalities is strict. The dominating one is the \deff{Pareto Improvement}. An outcome without a Pareto Improvement is \deff{Pareto Optimal}.

In a Bargaining game, $S$'s top right boundary is the Pareto Frontier.

\subsubsection{Properties, Bargaining Game Solutions}
\deff{Efficiency} - No outcome $(v_1, v_2)$ dominates $\Big( f_1(S, \vec{d}), f_2(S, \vec{d} \Big)$.

\deff{Symmetry} - Let $S^T = \{ (y, x): (x, y) \in S \}$ and $\vec{d}^T = (d_2, d_1)$, then
\[
\Big( f_1(S^T, \vec{d}^T), f_2(S^T, \vec{d}^T \Big) = \Big( f_1(S, \vec{d}), f_2(S, \vec{d} \Big)
\]

\deff{Independence of Irrelevant Alternative} (IIA) -  Let $S’ \subseteq S$ such that $\Big( f_1(S, \vec{d}), f_2(S, \vec{d} \Big) \in S’$, then,
\[
\Big( f_1(S', \vec{d}), f_2(S', \vec{d} \Big) = \Big( f_1(S, \vec{d}), f_2(S, \vec{d} \Big)
\]

\deff{Invariance under Equivalent Representations} (IER) - For any $\alpha_1, \alpha_2 \in \mathbb{R}, \vec{\beta} \in \mathbb{R}^2$,
\[
f_i((\alpha_1, \alpha_2)S + \vec\beta, \ (\alpha_1, \alpha_2)\vec{d} + \vec\beta)
 = \alpha_i f_i(S, \vec{d}) + \beta_i, \ i = 1, 2 \]


\subsection{Nash Bargaining Solution}
The \deff{Nash Bargaining Solution} for a bargaining game is the solution
\[
\argmax_{(v_1, v_2) \in S} (v_1 - d_1)(v_2 - d_2)
\]

The Nash Bargaining Solution is the only solution that satisfies Efficiency, Symmetry, IIA and IER.

\subsubsection{Utility Maximises}
\begin{itemize}
    \item \deff{Utilitarian}: $\max \sum_i u_i(A)$. 
    \item \deff{Nash}: $\max \prod_i u_i(A)$. 
    \item \deff{Egalitarian}: $\max \min_i u_i(A)$.
\end{itemize}





\section{Stable Matching}
\subsection{Definitions}
% Criterion for a good market, 1 sentence
In the case of matching medical students $S$ to hospitals $H$, with $|S| = n, |H| = m$: Each student $s$ has a strict preference order over $H$, denoted by $\succ_s$. Similarly for $\succ_h$. A matching $M: S \to H$ is a one-to-one mapping.

\smallskip

The goal is to design a market that \textbf{Thick}, \textbf{Safe} (truthful, fair and encouraging participation), and \textbf{Timely}.

\subsubsection{Stable Matching}
A pair $(s, h) \in S \times H$ \deff{Blocks} matching $M$, if
\[
h \succ_s M(s) \land s \succ_h M^{-1}(h)
\]
A matching $M$ is \deff{Stable} if there are no blocking pairs.

\underline{Analysis}: [1] A stable matching always exists. [2] A stable matching can always be found in polynomial time. e.g. GS Algorithm.

\subsection{Gale-Shapley Deferred Acceptance Algorithm}
Procedures:
\begin{enumerate}
    \item Start with all students unassigned.
    \item While there are unassigned students: Each unassigned student proposes to favourite \textbf{not-yet-proposed-to} hospital. Then each hospital looks at \textbf{students proposed to it in this round and whoever currently assigned} to it, and picks most preferred; all others remain unassigned.
    \item Repeat until all matched.
\end{enumerate}

\underline{Analysis}: [1] Terminate in $\le n^2$ iterations with a stable matching.

\subsubsection{Fairness for GS Algorithm}
Given $s \in S$, a \deff{Valid} hospital $h \in H$ is one that there exists some stable matching $M$ such that $M(s) = h$. \deff{best(s)} and \deff{worse(s)} are the most and least highly ranked valid hospital for $s$.

\smallskip

\subsubsection{Theorem}
GS Algorithm (with student proposing) assigns each student $s \in S$ to hospital $\text{best}(s)$, and each hospital $h \in H$ to student $\text{worst}(h)$.





\section{Fair Allocation of Indivisible Goods}
\subsection{Definitions}
With players $N = \{ 1,2,..., \}$ and indivisible goods $G = \{g_1, g_2, ..., g_m\}$, Player $i$ has value $v_i(g)$ for good $g$. An \deff{Allocation}, $A = (A_1, ..., A_n)$ is a partition, with \deff{Bundle} $A_i$ is allocated to Player $i$.
\\
Valuation is \deff{Additive} if for all $G' \subseteq G$,
$
v_i(G') = \sum_{g\in G'} v_i(g)
$


\subsubsection{Fairness}
\deff{Proportionality} - $v_i(A_i) \ge \frac{1}{n} \cdot v_i(G)$ for all $i \in N$.
\\
\deff{Envy-freness} -  $v_i(A_i) \ge v_i(A_j)$ for all $i, j \in N$.
\\
\deff{Envy-freeness Up to One Good} (EF1) - For any $i, j \in N$, if $A_j \neq \emptyset$, there exists $g \in A_j$ such that
\[
v_i(A_i) \ge v_i(A_j \backslash \{ g \})
\]
\underline{Analysis}: EF1 allocations always exist.
\\
\deff{Envy-freeness Up to Any Good} (EFX) - For any $i, j \in N$ and any $g \in A_j$, we have
\[
v_i(A_i) \ge v_i(A_j \backslash \{ g \})
\]

% analysis
\underline{Analysis}: [1] EFX is stronger than EF1. [2] EFX allocations always for $n=2$ (Cut-and-Choose protocol) and $n=4$, and its existence for $n \ge 4$ remains an open problem.



\subsubsection{Maximum Nash Welfare}
An allocation that maximises the Nash welfare, known as \deff{Maximum Nash Welfare} (MNW) allocation, satisfies EF1.
\\
\underline{Note}: If $MNW =0$, maximising the number of players with positive utility, then maximise Nash welfare among these players.
\\
\underline{Analysis}: The allocation is also Pareto Optimal.




\subsection{Envy-free}
\subsubsection{Round-Robin}
\underline{Procedures}: Let players take turns choosing their favourite good from the remaining, in the order $1,2,...,n,1,2,...,n,1,2,...$ until goods run out.
\\
\underline{Note}: Require \textbf{additive} valuations.


\subsubsection{Envy-Cycle Elimination}
\underline{Procedures}: [1] Allocate one good at a time in an arbitrary order. [2] Maintain an \textbf{envy graph} with $e: i \to j$ for each $i$ envying $j$. [3] At each step, the next good is allocated to a player with no incoming edges. [3'] Any cycle that arises is eliminated by giving $j$'s entire bundle to any $i$ for $i \to j$.
\\
\underline{Note}: Require \textbf{Monotone} valuation, not necessarily \textbf{additive}.


\subsection{Proportionality}
\deff{Maximin Share} (MMS) of Player $i$: Player $i$ divides all goods into $n$ bundles so as to maximise the values of the value of the minimum-value bundle. An relaxation of proportionality:
\[
\text{MMS}_i \le \frac{v_i(G)}{n}
\]
\underline{Analysis}: [1] Solution achieving MMS for every player exists for $n=2$ (Cut-and-Choose), but might not exist for $n \ge 3$. [2] Solution with at least $\frac{3}{4}\text{MMS}_i$ for each always achievable for all $n$. 


\subsection{Query Complexity}
The Envy-Cycle Elimination Algoritm can be implemented using $O(nm)$ queries, even with monotonic valuations.
\\
For EF1 solution for two agents with monotonic valuations, $O(\log m)$ queries suffice.
\\
Any deterministic EF1 algorithm needs $\Omega(\log m)$ queries.
\\
Any deterministic EFX algorithm needs queries \textbf{exponential} in $m$.





\section{Cake Cutting}
\subsection{Definitions}
Given a cake represented as interval $[0, 1]$, players $N = \{ 1, ..., n \}$, and valuation $v_1, ..., v_n$ over the cake, the goal is to find an allocation $A = (A_1, ..., A_n)$ where each $A_i$ is a union of \textbf{finitely many} intervals.

\subsubsection{Valuation Function}
Each valuation function $v_i$ is defined with a density function $f_i$
\[
v_i(B) = \int_B f_i(x), \ \forall B \subseteq [0, 1]
\]
\underline{Note}: $v_i(x, y) = v_i([x, y])$ denotes the utility of Player $i$ getting $[x, y]$.
\\
Valuation function is [1] Non-negative, [2] Additive (over disjoint pieces), [3] Non-atomic (where value of a point is $0$), and [4] Normalised (where value of entire cake is $1$).


\subsubsection{Robertson-Webb Model}
\deff{Robertson-Webb Model} defines two types of queries:
\begin{itemize}
    \item \codein{Eval\_i(x, y)} - Return $v_i(x, y)$.
    \item \codein{Cut\_i(x, a)} - Return the leftmost point $y$ s.t. $v_i(x, y) = a$ or such points do not exist.
\end{itemize}
Complexity of a model can be determined based on numbers of RW Model queries needed.


\subsection{Proportionality}
\deff{Cut-and-Choose Protocol} (for $n=2$ players) - One player cuts, the other player chooses first.
\\
\underline{Analysis}: It is always proportional and envy-free.

\medskip

\deff{Dubin-Spanier Protocol} - 1) Referee moves a knife over the cake from the left. 2) Whenever the current piece is worth $1/n$ to some player, that player leaves the procedure with that piece. 3) The last player gets the remaining cake.
\\
\underline{Analysis}: [1] Proportional for any $n$. [2] $O(n^2)$ queries.

\medskip

\deff{Even-Paz Protocol} (for $n=2^k$ players) - 1)Each player marks the point that divides the cake into two of equal value according to his valuation. 2) Let $t$ be the $n/2$-th mark from the left, recurse on $[0, t], [t, 1]$ each with $n/2$ agents. 3) Terminate with one player getting the whole cake.
\\
\underline{Analysis}: [1] Proportional for any $n$. [2] Require $O(n \log n)$ queries (optimal query number in all proportional protocols).


\subsection{Envy-freeness}
Envy-free protocols are complicated. \deff{Cut-and-Choose} is EF for $n=2$. \deff{Selfridge-Conway Protocol} is EF for $n=3$.

\subsubsection{Envy-Free Approximation}
1) Referee moves a knife over the cake from the left. 2) Whenever the current piece is worth $1/3$ to some player, that player leaves the procedure with that piece. 3) When the knift reaches the right end, 3a) if there are still agents left, the remaining is given to any of them; 3b) otherwise the remaining is given to the last agent.
\\
\underline{Analysis}: EF up to $1/3$, i.e. any player envies any other by at most a numeric value of $1/3$.

\subsection{Truthfulness}
Cut-and-Choose is truthful for the choose not for the cutter.


\section{Rent Division}
\subsection{Definitions}
Given players (roommates) $N = \{ 1, ..., n \}$ and $n$ rooms of an apartment, and rental price for the entire apartment, $r$, each Player $i$ has a value for room $j$, $v_{ij}$ subject to $\sum_j v_{ij} = r$.
\\
The objective is to output an outcome $(\sigma, \vec{p})$
\begin{itemize}
    \item Room allocation function $\sigma : N \to N$, mapping player to room
    \item Rent division $\vec{p} = (p_1, ..., p_n)$ s.t. $\sum_{j}p_j = r$
\end{itemize}
\underline{Note}: $p_j$ is the rent for Room $j$, not for Player $j$ by convention.


\subsection{Allocation Algorithms}
\subsubsection{Envy-free Outcomes}
An outcome $(\sigma, \vec{p})$ is \deff{Envy-Free} if for all $i, j \in N$,
\[
v_{i, \sigma(i)} - p_{\sigma(i)} \ge v_{ij} - p_j
\]
Intuitively, it means no one prefers anyone's room for the price that person is charged.
\\
\underline{Analysis}: EF Outcomes can be computed using Linear Programming, but it is not always fair.

\subsubsection{Equitability \& Maximin Outcomes}
\deff{Equitability} - Maximum difference in utility of any two players. To maximise equitability is to minimise the following expression
\[
D(\vec{p}) = \max_{i, j \in N}
(u_i(\vec{p}) - u_j(\vec{p}))
\]

\smallskip

\deff{Maximin} - Minimum utility any player derives. It is the \textbf{Egalitarian Utility}, defined as 
\[
U(\vec{p}) = \min_{i \in N} u_i(\vec{p})
\]

\smallskip

\underline{Theorem}: There is a unique Maximin price vector and it is also equitable. (Yet there are still equitable, non-Maximin price vectors).





\section{Committee Voting}
\subsection{Definitions}
\deff{Approval Voting} is the game where each voter submits a subset of candidates he / she approves. It consists of voters $N = \{1, ..., n\}$, candidates $C = \{1, ..., m \}$, and set of candidates $A_i \subseteq C$ that Voter $i$ approves for $i=1, ..., n$. The goal is to choose a committee $W \subseteq C$ of size $|W| = k$ given.
\\
Voter $i$'s utility is number of committee members he approves:
\[
u_i(W) = |A_i \cap W|
\]


\subsubsection{Social Welfare \& Coverage}
For a given committee,
\\
\deff{Social Welfare} - $\sum_{i \in N} u_i(W)$, total number of approvals committee members receive.
\\
\deff{Coverage} - Total number of voters who approves at least one committee member. \underline{Note}:A candidate \deff{Covers} a voter if the voter approves the candidate.
\\
\deff{Approval Voting} (AC) - Select a committee that maximises social welfare,
\[
W_{AC} = \max_{W \subseteq C, |W| = k} \sum_{i \in N} u_i(W)
\]
\deff{Chamberlin-Courant} (CC) - Select a committee that maximises coverage,
\[
W_{CC} = \max_{W \subseteq C, |W| = k} |\{ i \in N | A_i \cap W \neq \emptyset \}|
\]
\underline{Note}: Both AV and CC solutions are not unique.



\subsection{Justified Representation}
A group of voters $S \subseteq N$ is a \deff{Cohesive Group} if
$
|S| \ge n / k \land  |\cap_{i \in S} A_i| \ge 1
$
\\
\deff{Justified Representation} (JR) - For any cohesive group of voters $S \subseteq N$, there exists $i \in S$ such that
$
A_i \cap W \neq \emptyset
$

\subsubsection{GreedyCC}
Find a CC committee is NP-hard, equivalent to SET-COVER.
\\
\deff{GreedyCC} is one greedy solution: 1) Start with an empty set of candidates. 2) In each step, choose a candidate that covers as many uncovered voters as possible. 3) Repeat this until $k$ candidates have been chosen.
\\
\underline{Analysis}: GreedyCC satisfies JR.



\subsection{Extended JR}
For a positive integer $t \in \mathbb{Z}_{> 0}$, a group of voters $S \subseteq N$ is a \deff{$t$-Cohesive Group} if
$
|S| \ge t \times n / k \land  |\cap_{i \in S} A_i| \ge t
$
\\
\deff{Extended Justified Representation} (EJR) - For any positive integer $t \in \mathbb{Z}_{> 0}$ and any $t$-cohesive group of voters $S \subseteq N$, there exists $i \in S$ such that
$
|A_i \cap W| \ge t
$
\\
\underline{Note}: A cohesive group is a $1$-cohesive group. JR is only EJR of the case $t = 1$.


\subsubsection{Thiele Method}
\deff{Thiele Method} - A generalised method for finding out a group of candidates that satisfies a specific criterion.
\begin{itemize}
    \item Choose an infinite non-increasing sequence $\{ s_n \}^\infty_{n=1}$, based on the criterion of interest.
    \item Choose a committee $W$ that maximises
    \[
    \sum_{i \in N} \sum_{j \in u_i(W)} s_j = \sum_{i \in N } (s_1 + s_2 + ... + s_{u_i(W)})
    \]
\end{itemize}
\underline{Note}: If a voter approves $r$ candidates, he contributes a value of $s_1 + ... + s_r $ to the score.
\\
\underline{Analysis}: [1] $s_i = 1, \forall i$ for AV. [2] $s_1 = 1, \ s_2 = s_3 = ... = 0$ for CC.


\subsubsection{Proportional Approval Voting}
\deff{Proportional Approval Voting} (PAV) - The solution when we set $s_n = 1/n$, i.e. the Harmonic Sequence, then apply Thiele Method.
\\
PAV satisfies EJR.
\\
\underline{Note}: PAV is NP-Hard. Greedy variant of PAV is not EJR or even JR.



\subsubsection{Method of Equal Shares}
\deff{Method of Equal Shares} (MES) - Another procedure for finding a EJR solution
\begin{enumerate}
    \item Each voter has a budget of $k/n$. Each candidate costs $1$. The voters who approve this candidate have to “pool” their money to add this candidate to the committee.
    \item Start with an empty committee.
    \item In each round, we want to add a candidate whose approved voters have a total budget of $≥ 1$ left.
    \item If there are several such candidates, choose one such that the maximum amount that any agent has to pay is minimised.
    \item If no more candidate can be afforded but the committee still has size $< k$, fill in the rest of the committee using some tie-breaking criterion.
\end{enumerate}
\underline{Note}: We do not require all players to pool a candidate must provide the equal amount.
\underline{Analysis}: MES satisfies EJR, and can be implemented in polynomial time.

\subsection{Summary}
\begin{center}
\begin{tabular}{c|c c c}
    \hline
            & JR    & EJR   & Poly-time \\
    \hline
     AV     & No    & No    & Yes       \\
     CC     & Yes   & No    & No        \\
     GreedyCC & Yes & No    & Yes       \\
     PAV    & Yes   & Yes   & No        \\
     MES    & Yes   & Yes   & Yes       \\
     \hline
\end{tabular}
\end{center}








\section{Tournaments}
\subsection{Definitions}
A \deff{Tournament} $T = (A, \succ)$ consists of a set of \deff{Alternatives} (players), $A$, and a dominance relation $\succ$ such that $\forall i \neq j \in A, i \succ j \lor j \succ i$.
\\
\underline{Note}: [1] By convention, we draw a direct edge from $i$ to $j$ for $i \succ j$.

\smallskip

\deff{Outdegree} of an alternative $x \in A$ is number of alternatives dominated by $x$. A \deff{Condorcet Winner} / \deff{Loser} is an alternative that dominates / is dominated by all others (i.e. outdegree $n-1$ / $0$).

\smallskip

A \deff{Tournament Solution}, $S$, is a method for choosing tournament winners, satisfying conditions: [1] It returns a \textbf{non-empty} subset of alternatives for any tournament (can be $A$ trivially), $\forall T = (A, \succ), S(T) \subseteq A$. [2] \deff{Invariant Under Isomorphism}: for $h: A \to A'$ an isomorphism between $T = (A, \succ)$ and $T' = (A', \succ')$, satisfying
\[
S(T') = h(S(T))
\]
\underline{Analysis}: By symmetric, in a cyclic tournament of $|A|=3$, every tournament solution must select all alternatives.


\subsection{Tournament Solutions}
A summary of types of tournament solutions:
\begin{itemize}
    \item \deff{Copeland Set} (CO) - Alternatives with highest outdegrees.
    \item \deff{Top Cycle} (TC) - Alternatives that can reach every other alternative via a directed path (of any length)
    \item \deff{Uncovered Set} (UC) - Alternatives that can reach every other alternative via a directed path of length $\le 2$
    \item \deff{Banks Set} (BA) - Alternatives that appear as the maximal (i.e., strongest) element of some transitive sub-tournament that cannot be extended
\end{itemize}
\underline{Note}:  \deff{Transitive Tournament} - A tournament where alternatives can be ordered a sequence $a_1, …, a_k$ s.t. $\forall 1 \le i < j \le k, a_i \succ a_j$.
\\
\underline{Containment Relations}: $UC \subseteq TC$, $CO \subseteq UC$, $BA \subseteq UC$.

\subsubsection{Top Cycle}
\underline{Equivalent Definition} for TC: The unique smallest nonempty set $B$ of alternatives such that all in $B$ dominates all outside $B$.

\subsubsection{Uncovered Set}
$x$ \deff{Covers} $y$ if [1] $x \succ y$, and [2] $\forall z \in A, y \succ z \implies x \succ z$.
\\
\underline{Equivalent Definition} fot UC: The set of all uncovered alternatives.

\subsubsection{Axioms}
CO, TC, UC, and BA all satisfy Condorcet-Consistency and Monotonicity
\begin{itemize}
    \item \deff{Condorcet-Consistency} - If there is a Condorcet winner $x$, then $x$ is uniquely chosen.
    \item \deff{Monotonicty} - If $x$ is chosen, then it should remain chosen when it is strengthened against another alternative $y$ (and everything else stays the same).
\end{itemize}
For $y \succ x$, \deff{Strengthen} $x$ against $y$ is to change $y \succ x$ to $x \succ y$.


\subsection{Tournament Fixing Problem}
In a \textbf{balanced knockout tournament}, given alternatives $A$, (probabilistic) dominance relation $\succ$ and our favourite alternative $x \in A$, the goal is find a tournament bracket that maximises win-rate for $x$. If such a bracket exists, such a winner is called \deff{Knockout Winner}.
\\
\underline{Analysis}: Any alternative who dominates $< \log_2 n$ others are not knockout winners.

\subsubsection{String King Theorem}
An alternative is $x$ is a \deff{King} if it is in UC. \underline{Note}: it beats any other directly or via a third alternative.

For $x$ is a king, suppose $x \in A$ beats $P \subseteq A$ and loses to $Q \subseteq A$. if $|P| \ge n / 2$, $x$ is a knockout winner.
\\
\underline{Analysis}: Any alternative in CO is a knockout winner.
\\
\underline{Note}: Players not in UC, or those in UC with $|P| < n/2$ can still be knockout winners.





\hline

\section{Imtermediate Results}
\subsection{From Assignments}
\citeqn{Assignment5 Qn3} For a cooperative game with non-empty core, the Shapley vector is not necessarily in the core.





\end{multicols}
\end{document}
