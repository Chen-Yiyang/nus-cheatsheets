%%%%%%%%%%%%%%%%%%%%%%%%%%%%%%%%%%%%%%%%%%%%%%%%%%%%%%%%%%%%%%%%%%%%%%
% Original Source: Dave Richeson (divisbyzero.com), Dickinson College
% Modified By: Chen Yiyang
% 
% A one-size-fits-all LaTeX cheat sheet. Kept to two pages, so it 
% can be printed (double-sided) on one piece of paper
% 
% Feel free to distribute this example, but please keep the referral
% to divisbyzero.com
% 
% Guidance on the use of the Overleaf logos can be found here:
% https://www.overleaf.com/for/partners/logos 
%
% Credits: 
% Jovyn Tan, https://github.com/jovyntls
% for the nice navy blue colour for in-line and display Math.
%%%%%%%%%%%%%%%%%%%%%%%%%%%%%%%%%%%%%%%%%%%%%%%%%%%%%%%%%%%%%%%%%%%%%%

\documentclass[10pt,landscape,letterpaper]{article}
\usepackage{amssymb}
\usepackage{amsmath}
\usepackage{amsthm}
%\usepackage{fonts}
\usepackage{multicol,multirow}
\usepackage{spverbatim}
\usepackage{graphicx}
\usepackage{ifthen}
\usepackage[landscape]{geometry}
\usepackage[colorlinks=true,urlcolor=olgreen]{hyperref}
\usepackage{booktabs}
\usepackage{fontspec}
\setmainfont[Ligatures=TeX]{TeX Gyre Pagella}
\setsansfont{Fira Sans}
\setmonofont{Inconsolata}
\usepackage{unicode-math}
\setmathfont{TeX Gyre Pagella Math}
\usepackage{microtype}

\usepackage{empheq}

% new:
\def\MT@is@uni@comp#1\iffontchar#2\else#3\fi\relax{%
  \ifx\\#2\\\else\edef\MT@char{\iffontchar#2\fi}\fi
}
\makeatother

\ifthenelse{\lengthtest { \paperwidth = 11in}}
    { \geometry{margin=0.4in} }
	{\ifthenelse{ \lengthtest{ \paperwidth = 297mm}}
		{\geometry{top=1cm,left=1cm,right=1cm,bottom=1cm} }
		{\geometry{top=1cm,left=1cm,right=1cm,bottom=1cm} }
	}
\pagestyle{empty}
\makeatletter
\renewcommand{\section}{\@startsection{section}{1}{0mm}%
                                {-1ex plus -.5ex minus -.2ex}%
                                {0.5ex plus .2ex}%x
                                {\sffamily\large}}
\renewcommand{\subsection}{\@startsection{subsection}{2}{0mm}%
                                {-1explus -.5ex minus -.2ex}%
                                {0.5ex plus .2ex}%
                                {\sffamily\normalsize\itshape}}
\renewcommand{\subsubsection}{\@startsection{subsubsection}{3}{0mm}%
                                {-1ex plus -.5ex minus -.2ex}%
                                {1ex plus .2ex}%
                                {\normalfont\small\itshape}}
\makeatother
\setcounter{secnumdepth}{0}
\setlength{\parindent}{0pt}
\setlength{\parskip}{0pt plus 0.5ex}
% -----------------------------------------------------------------------

\usepackage{academicons}

\begin{document}

\definecolor{myblue}{cmyk}{1,.72,0,.38}
\everymath{\color{myblue}}
\everydisplay{\color{myblue}}

\footnotesize
%\raggedright

\begin{center}
  {\huge\sffamily\bfseries ST2131 Cheatsheet} \huge\bfseries\\
  by Yiyang, AY20/21
\end{center}
\setlength{\premulticols}{0pt}
\setlength{\postmulticols}{0pt}
\setlength{\multicolsep}{1pt}
\setlength{\columnsep}{1.8em}
\begin{multicols}{3}

\section{Chapter 01 - Combinatorial Analysis}
\subsection{Some Combinatorial Identities}
For all non-negative integers $m, n, k$ and $k \leq n$,
\begin{itemize}
    \item $k {{n}\choose{k}} = (n-k+1) {{n}\choose{k-1}} = n {{n-1}\choose{k-1}}$ \ \emph{(AY20/21Sem2 Tut1Qn7)}
    \item $\sum_{k=1}^{n}{k {n \choose{k}}} = n2^{n-1}$ \ \emph{(AY20/21Sem2 Tut1Qn8)}
    \item ${{n+m}\choose{k}} = {{n}\choose{0}}{{m}\choose{k}} + {{n}\choose{1}}{{m}\choose{k-1}} + ... + {{n}\choose{r}}{{m}\choose{0}}$ \ \emph{(AY20/21Sem2 Tut1Qn9)}
    \item ${{2n}\choose{n}} = \sum_{k=0}^{n} {{n}\choose{k}}^2$ \ \emph{(AY20/21Sem2 Tut1Qn10)}
\end{itemize}

\section{Chapter 02 - Axioms of Probability}
\subsection{Inclusion/Exclusion Principle (Prop. 2.6)}
Let $E_1, E_2, ..., E_n$ be any $n$ events, then
\[
P(E_1 \cup E_2 \cup ... E_n) = \sum_{r=1}^{n} \Big( (-1)^{r+1} \sum_{1 \leq i_1 < ... < i_r \leq n} P(E_{i_1} \cap ... \cap E_{i_r}) \Big)
\]

\subsection{Tut2Qn16 - Generalised Bonferroni's Inequality}
Let $E_1, E_2, ..., E_n$ be any $n$ events, then
\[
P(E_1E_2 \ ... \ E_n) \geq P(E_1) + ... + P(E_n) - (n-1)
\]
When $n=2$,
\[
P(E_1E_2) \geq P(E_1) + P(E_2) - 1
\]

\section{Chapter 03 - Conditional Probability}
\subsection{Bayes' Formulae}
Suppose events $A_1$, $A_2$, ..., $A_n$ partition the sample space, and $P(A_i) > 0$, for all $i = 1, 2, ..., n$. 
\\
Then for any event $B$, and any $1 \leq i \leq n$,
\begin{align}
P(B) &= \sum_{j=1}^{n} {P(B|A_j)P(A_j)}
\\
P(A_i|B) &= \frac{P(B|A_i)P(A_i)}{\sum_{j=1}^{n} {P(B|A_j)P(A_j)}}
\end{align}

\subsection{Some Identities Involving Conditional Probability}
For any events $A$, $B$, $C$,
\begin{align*}
   P(A|C) &= P(AB|C) + P(AB^C|C) 
   \\
   &= P(A|BC)P(B|C) + P(A|B^CC)P(B^C|C) 
\end{align*}



\section{Chapter 04 - Random Variables}
\subsection{Tail Sum Formula}
For \emph{non-negative integer-valued} random variable $X$,
if $X$ is a D.R.V. (i.e. $X = 0, 1,...,2$),
\[
E(X) = \sum_{k=1}^{\infty} P(X \geq k) = \sum_{k=0}^{\infty} P(X > k)
\]
or if $X$ is a C.R.V.,
\[
E(X)= \int_{0}^{\infty} P(X>x) \,dx = \int_{0}^{\infty} P(X \geq x) \,dx
\]

\section{Chapter 05 - Continuous Random Variable}
\subsection{Distribution of a Function of R.V.}
For r.v. $X$ with pdf. $f_X(x)$, assume $g(x)$ is a function of $X$ that is \textbf{strictly monotonic} and \textbf{differentiable}. Then the pdf. of $Y=g(X)$,
\[
f_Y(y) = \begin{cases}
f_X(g^{-1}(y)) \big| \frac{d}{dy} g^{-1}(y)  \big|, \ &y=g(x) \text{ for some $x$}
\\
0, \ &\text{otherwise}
\end{cases}
\]

\subsection{Binomial to Normal Approximation}
(Remember \textbf{Continuity Correction}!!!)
\\
For $X \sim Bin(n, p)$ where $npq$ is large (generally good when $npq \geq 10$),
\[
Bin(n,p) \approx N(np, npq) \text{ , i.e. } \frac{X-np}{\sqrt{npq}} \approx Z
\]

\subsection{Binomial to Poisson Approximation}
For $X \sim Bin(n, p)$ where $n$ is large and $p$ (or $q$) is small so that $np$ (or $nq$) is moderate.
\begin{itemize}
    \item when $p < 0.1$, $Bin(n, p) \approx Poisson(np)$
    \item when $p > 0.9$, $Bin(n, q) \approx Poisson(nq)$
\end{itemize}

\section{Chapter 06 - Joint Distributions}
\subsection{Convolution of Independent Distributions}
\begin{align*}
F_{X+Y}(a) &= \int_{-\infty}^{\infty}F_Y(a-x) \ f_X(x)\,\, dx = \int_{-\infty}^{\infty}F_X(a-y) \ f_Y(y) \, dy
\\
f_{X+Y}(a) &= \int_{-\infty}^{\infty}f_Y(a-x) \ f_X(x) \, dx = \int_{-\infty}^{\infty}f_X(a-y) \ f_Y(y) \, dy
\end{align*}

\subsection{Prop.6.4 - Sum of Independent Gamma R.V.s}
Assume $X \sim Gamma(\alpha, \lambda)$ and $Y \sim Gamma(\beta, \lambda)$ are independent.
\[ X+Y \sim Gamma(\alpha+\beta, \lambda)
\]

\subsection{Prop.6.5 - Sum of Independent Normal R.V.s}
Assume $X_i, i=1,2,...,n$ are independent random variables that are normally distributed with parameters $\mu_i, \sigma_i^2, \ i = 1,2,...,n$.
\[ \sum_{i=1}^n X_i \sim N(\sum_{i=1}^n \mu_i \ \sum_{i=1}^n \sigma_i^2)
\]

\subsection{Ex.6.18 - Sum of Independent Poisson R.V.s}
Assume $X \sim Poisson(\lambda), \ Y \sim Poisson(\mu)$ are independent.
\[
X+Y \sim Poisson(\lambda + \mu)
\]

\subsection{Ex.6.19 - Sum of Independent Binomial R.V.s}
Assume $X \sim Bin(n,p), \ Y \sim Bin(m, p)$ are independent.
\[ X+Y \sim Bin(n+m, p)
\]
Note: This statement only works when the second parameter of both R.V.s are the same (i.e. both $p$). For problems with different parameters and large values, can consider using Normal Approximation with \emph{(Prop.6.5)}.

\section{Ch 07 - Properties of Expectation}
\subsection{Ex.7.20 - Expectation of a Random Sum}
Suppose $X_1, X_2, ...$ are i.i.d. with common mean $\mu$. Suppose $N$ is a non-negative integer-valued random variable independent of the $X_i$.
\[ \sum_{k=1}^N X_k = \mu E[N]
\]

\subsection{Common Moment Generating Functions}
\begin{itemize}
    \item $X \sim Be(p), \ M_X(t) = 1 - p + pe^t$
    \item $X \sim Bin(n, p), \ M_X(t) = (1-p+pe^t)^n$
    \item $X \sim Geom(p), \ M_X(t) = \frac{pe^t}{1-qe^t}$
    \item $X \sim Poisson(\lambda), \ M_X(t) = e^{\lambda(e^t-1)}$
    \item $X \sim U(\alpha, \beta), \ M_X(t) = \frac{e^{\beta t} - e^{\alpha t}}{(\beta - \alpha t) t}$
    \item $X \sim Exp(\lambda), \ M_X(t) = \frac{\lambda}{\lambda - t}, \text{for } t < \lambda$
    \item $X \sim N(\mu, \sigma^2), \ M_X(t) = e^{(\mu t + \sigma^2 t^2 / 2)}$
\end{itemize}

\subsection{Less Common MGFs}
\begin{itemize}
    \item \emph{(Ex.7.31)} X is a chi-squared r.v. with $n$ deg. of freedom, $M_X(t) = {(E[e^{tZ^2}])}^n = {(1-2t)}^{-n/2}$
\end{itemize}

\subsection{Ex.7.34 - "Partitioned" Poisson Distribution}
Let $X$ be the r.v. that denotes total number of events. Suppose each event is a Ber. process with $p$ probability of being $A$ and $q = (1-p)$ being $B$. Let $X_A, X_B$ denote the number of events that are $A$ and $B$ respectively.
If $X \sim Poisson(\lambda)$,
\[
X_A \sim Poisson(p\lambda), \ X_B \sim Poisson(q\lambda)
\]

\section{Ch 08 - Limit Theorems}
\subsection{Markov's Inequality}
For \textbf{non-negative} r.v. $X$ and any $a > 0$,
\[
P(X \geq a) \leq \frac{E{X}}{a}
\]

\subsection{Chebyshev's Inequality}
Let $X$ be a r.v. with mean $\mu$, then for any $a > 0$,
\[
P(|X-\mu| \geq a) \leq \frac{\text{var}(X)}{a^2}
\]

\subsection{One-sided Chebyshev's Inequality}
Let $X$ be a r.v. with \textbf{zero mean} and variance $\sigma^2$, then for any $a > 0$,
\[
P(X \geq a) \leq \frac{\sigma^2}{\sigma^2 + a^2}
\]

\subsection{Central Limit Theorem}
(Remember \textbf{Continuity Correction} when a CRV is used to approximate a DRV!!!)
For a sequence of i.i.d. r.v.s $X_1, X_2, ... $, each with mean $\mu$ and variance $\sigma^2$, 
\[
\frac{X_1 + ... + X_n - n\mu}{\sigma \sqrt{\pi}} \to Z, \ \text{as } n \to \infty
\]

\subsection{WLLN & SLLN}

\subsection{Jensen's Inequality}
For any r.v. $X$ and convex function $g(X)$,
\[
E[g(X)] \geq g(E[X])
\]
, provided the expectations exist and are finite.

\midrule

\section{D.R.V. Models}
\subsection{Bernoulli}
$X \sim Be(p)$, indicate whether an event is successful.
\\
\textbf{Parameter} - $p = P(X=1)$ : success rate
\\
\textbf{Distribution} - $P(X=1) = p, \ P(X=0) = q = 1-p$
\\
$E(X) = p, \ \text{var}(X) = pq = p(1-p)$


\subsection{Binomial}
$X \sim Bin(n, p)$, total number of successes in $n$ i.i.d. $Be(p)$ trials.
\\
\textbf{Parameters}
\begin{itemize}
    \item $n$ : number of trials
    \item $p$ : success rate for each Bernoulli trial
\end{itemize}

\textbf{Distribution}
\[
P(X=k) = {n \choose k} p^x q^{n-x}, \ k = 0, 1, ..., n
\]
\\
$E(X) = np, \ \text{var}(X) = npq = np(1-p)$

\subsection{Geometric}
$X \sim Geom(p)$, number of i.i.d $Be(p)$ trials until one success. $X = 1, 2, ...$
\\
\textbf{Parameter} - $p$ : success rate
\\
\textbf{Distribution}
\[
P(X=k) = pq^{k-1}, \ k = 1, 2, ... 
\]
Memoryless Property: $P(X > s+t | X > s) = P(X > t), \ s, t > 0$.
\\
$E(X) = \frac{1}{p}, \ \text{var}(X) = \frac{1-p}{p^2}$



\subsection{Negative Binomial}
$X \sim NB(r, p)$, number of i.i.d $Be(p)$ trials for first $r$ successes. $X = r, r+1, ...$
\\
\textbf{Parameter}
\begin{itemize}
    \item $r$ : successes needed
    \item $p$ : success rate
\end{itemize}

\textbf{Distribution}
\[
P(X=k) = {{k-1} \choose {r-1}} \ p^r q ^ {x-r}, \ k = r, r+1, ...
\]
\\
$E(X) = \frac{r}{p}, \ \text{var}(X) = \frac{r(1-p)}{p^2}$
\\
$Geom(p) = NB(1, p)$

\subsection{Poisson}
$X \sim Poisson(\lambda)$
\\
\textbf{Parameter} - $\lambda$ : "average occurrence rate in unit time interval"
\\
\textbf{Distribution}
\[
P(X = k) = e^{-\lambda} \frac{\lambda^k}{k!}, \ k = 0, 1, ...
\]
\\
$E(X) = \text{var}(X) = \lambda$



\subsection{Hypergeometric}
Suppose there are $N$ identical balls, $m$ of them are red and $N-m$ are blue.
$X \sim H(n, N, m)$ is the number of red balls obtained in $n$ draws without replacement.
\\
\textbf{Parameter}
\begin{itemize}
    \item $N$ : total number of objects ("red and blue balls")
    \item $m$ : number of objects considered success ("red balls")
    \item $n$ : number of trials without replacement ("draws")
\end{itemize}
\textbf{Distribution}
\[
P(X = k) = \frac{{m \choose k}{{N-m} \choose {n-k}}}{{N \choose n}}, \ k = 0, 1, ..., n 
\]
\\
$E(X) = \frac{nm}{N}, \ \text{var}(X) = \frac{nm}{N} \big[ \frac{(n-1)(m-1}{N-1} + 1 - \frac{nm}{N}  \big]$

\midrule

\section{C.R.V. Models}
\subsection{Uniform}
$X \sim U(a, b)$, where $X$ has equal probability of taking any value in $(a, b)$.
\\
\textbf{Parameters} - $a$ and $b$ : the start and end value for the interval
\\
\textbf{Distribution}
\[
f(x) = \begin{cases}
\frac{1}{b-a}, \ & a < x < b
\\
0, \ & \text{otherwise}
\end{cases}
\]
\\
\[
F(x) = \begin{cases}
0, \ & x \leq a
\\
\frac{x-a}{b-a}, \ & a < x < b
\\
1, \ & b \leq x
\end{cases}
\]
\\
$E(X) = \frac{a+b}{2}, \ \text{var}(X) = \frac{(b-a)^2}{12}$


\subsection{Exponential}
$X \sim Exp(\lambda)$ usually models the life time of a product, for $\lambda > 0$
\\
\textbf{Distribution}
\[
f(x) = \begin{cases}
\lambda e^{-\lambda x}, \ & x \geq 0
\\
0, \ & otherwise
\end{cases}
\]
\\
\[
F(x) = \begin{cases}
0, \ & x \leq 0
\\
1 - e^{-\lambda x}, \ & x > 0
\end{cases}
\]
\\
Memoryless Property: $P(X > s+t | X > s) = P(X > t), \ s, t > 0$.
\\
$E(X) = \frac{1}{\lambda}, \ \text{var}(X) = \frac{1}{\lambda^2}$


\subsection{Normal}
$X \sim N(\mu, \sigma^2)$. Special case : $Z \sim N(0, 1)$ standard normal
\\
\textbf{Parameters}
\begin{itemize}
    \item $\mu$ : mean
    \item $\sigma$ : standard deviation
\end{itemize}
\textbf{Distribution}
\begin{align*}
    f_Z(z) &= \frac{1}{\sqrt{2 \pi}} e^{-z^2/2}, \ z \in \mathbb{R}
    \\
    f_X(x) &= \frac{1}{\sqrt{2 \pi} \sigma} e^{-(x-\mu)^2/(2\sigma^2)}, \ x \in \mathbb{R}
\end{align*}
\\
$E(X) = \mu, \ \text{var}(X) = \sigma^2$


\subsection{Gamma}
$X \sim Gamma(\alpha, \lambda)$ can be seen as the sum of $\alpha$ independent $Exp(\lambda)$, for $\alpha, \lambda > 0$. (Refer to \emph{Prop.6.4})
\\
\textbf{Parameters}
\begin{itemize}
    \item $\alpha$ : shape parameter
    \item $\lambda$ : rate parameter
    \item ($\frac{1}{\lambda}$ : scale parameter)
\end{itemize}
\textbf{Distribution}
\[
f(x) = \begin{cases}
\frac{\lambda e^{\lambda x}(\lambda x)^{\alpha-1}}{\Gamma(\alpha)}, \ & x \geq 0
\\
0, \ & x < 0
\end{cases}
\]
\\
$Exp(\lambda) = Gamma(1, \lambda)$ is a special case of Gamma r.v.
\\
$E(X) = \frac{\alpha}{\lambda}, \ \text{var}(X) = \frac{\alpha}{\lambda^2}$
\\
\textbf{Gamma Function} $\Gamma(\alpha) = \int_0^{\infty} e^{-y}y^{\alpha-1} \,dy$
\\
It satisfies that
\begin{itemize}
    \item $\Gamma(1) = 1, \ \Gamma(\frac{1}{2}) = \sqrt{\pi}$
    \item $\Gamma(\alpha) = (\alpha-1)\Gamma(\alpha-1), \ \alpha > 0$
    \item $\Gamma(n) = (n-1)!, \ n \in \mathbb{Z}^{+}$
    
\end{itemize}

\subsection{Weibull Distribution}
$S \sim W(\nu, \alpha, \beta)$ can be seen as the generalised form of Exponential r.v.
\begin{itemize}
    \item $E(\lambda) = W(1, \lambda, 0)$
    \item If $X \sim E(\lambda)$, then linear transformation $Y = \alpha X + \nu \ \sim W(\nu, \alpha, \lambda)$ \emph{(Tut7Qn15)}
\end{itemize}

\subsection{Cauchy}
$X \sim \text{Cauchy}(\theta, \alpha)$ for $\theta \in \mathbb{R}, \alpha > 0$ if it has the distribution:
\[
f(x) = \frac{1}{\pi \alpha \big[1 + (\frac{x-\theta}{\alpha})^2 \big]}, \ x \in \mathbb{R}
\]
\\
$E(X)$ and $\text{var}(X)$ do not exist for Cauchy r.v.

\subsection{Beta}
$X \sim B(a, b)$. Specifically, $U(0, 1) = B(1,1)$ is a special case of Beta r.v.

\midrule

\section{Common Identities}
$\sum_{i=1}^{\infty} i r^{i-1} = \frac{1}{(1-r)^2}$, for $|r| < 1$

\end{multicols}
\end{document}
