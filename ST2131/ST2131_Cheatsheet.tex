%%%%%%%%%%%%%%%%%%%%%%%%%%%%%%%%%%%%%%%%%%%%%%%%%%%%%%%%%%%%%%%%%%%%%%
% Original Source: Dave Richeson (divisbyzero.com), Dickinson College
% Modified By: Chen Yiyang
% 
% A one-size-fits-all LaTeX cheat sheet. Kept to two pages, so it 
% can be printed (double-sided) on one piece of paper
% 
% Feel free to distribute this example, but please keep the referral
% to divisbyzero.com
% 
% Guidance on the use of the Overleaf logos can be found here:
% https://www.overleaf.com/for/partners/logos 
%%%%%%%%%%%%%%%%%%%%%%%%%%%%%%%%%%%%%%%%%%%%%%%%%%%%%%%%%%%%%%%%%%%%%%

\documentclass[10pt,landscape,letterpaper]{article}
\usepackage{amssymb,amsmath,amsthm,amsfonts}
\usepackage{multicol,multirow}
\usepackage{spverbatim}
\usepackage{graphicx}
\usepackage{ifthen}
\usepackage[landscape]{geometry}
\usepackage[colorlinks=true,urlcolor=olgreen]{hyperref}
\usepackage{booktabs}
\usepackage{fontspec}
\setmainfont[Ligatures=TeX]{TeX Gyre Pagella}
\setsansfont{Fira Sans}
\setmonofont{Inconsolata}
\usepackage{unicode-math}
\setmathfont{TeX Gyre Pagella Math}
\usepackage{microtype}
% new:
\def\MT@is@uni@comp#1\iffontchar#2\else#3\fi\relax{%
  \ifx\\#2\\\else\edef\MT@char{\iffontchar#2\fi}\fi
}
\makeatother

\ifthenelse{\lengthtest { \paperwidth = 11in}}
    { \geometry{margin=0.4in} }
	{\ifthenelse{ \lengthtest{ \paperwidth = 297mm}}
		{\geometry{top=1cm,left=1cm,right=1cm,bottom=1cm} }
		{\geometry{top=1cm,left=1cm,right=1cm,bottom=1cm} }
	}
\pagestyle{empty}
\makeatletter
\renewcommand{\section}{\@startsection{section}{1}{0mm}%
                                {-1ex plus -.5ex minus -.2ex}%
                                {0.5ex plus .2ex}%x
                                {\sffamily\large}}
\renewcommand{\subsection}{\@startsection{subsection}{2}{0mm}%
                                {-1explus -.5ex minus -.2ex}%
                                {0.5ex plus .2ex}%
                                {\sffamily\normalsize\itshape}}
\renewcommand{\subsubsection}{\@startsection{subsubsection}{3}{0mm}%
                                {-1ex plus -.5ex minus -.2ex}%
                                {1ex plus .2ex}%
                                {\normalfont\small\itshape}}
\makeatother
\setcounter{secnumdepth}{0}
\setlength{\parindent}{0pt}
\setlength{\parskip}{0pt plus 0.5ex}
% -----------------------------------------------------------------------

\usepackage{academicons}


\begin{document}
\footnotesize
%\raggedright

\begin{center}
  {\huge\sffamily\bfseries ST2131 Cheatsheet} \huge\bfseries\\
  by Yiyang, AY20/21
\end{center}
\setlength{\premulticols}{0pt}
\setlength{\postmulticols}{0pt}
\setlength{\multicolsep}{1pt}
\setlength{\columnsep}{1.8em}
\begin{multicols}{3}

\section{Chapter 01 - Combinatorial Analysis}
\subsection{Multinomial Theorem}
Let $n$ be a non-negative integer, then
\[
(x_1 + x_2 + ... + x_r)^n = \sum_{n_1 + ... + n_r = n} {{n}\choose{n_1, n_2, ..., n_r}} \ x_1^{r_1} x_2^{r_2} ... x_r^{n_r} 
\]
where ${{n}\choose{n_1, n_2, ..., n_r}}$ is the Multinomial Coefficient, and it satisfies 
\begin{align*}
    {{n}\choose{n_1, n_2, ..., n_r}} &= {{n}\choose{n_1}}{{n-n_1}\choose{n_2}} \ ... \ {{n-n_1-...-n_{r-1}}\choose{n_r}} 
    \\
    &= \frac{n!}{n_1!n_2!...n_r!}
\end{align*}

\subsection{Some Combinatorial Identities}
For all non-negative integers $n, m, k$ and $k \leq n$,
\begin{itemize}
    \item $k {{n}\choose{k}} = (n-k+1) {{n}\choose{k-1}} = n {{n-1}\choose{k-1}}$ \ \emph{(AY20/21Sem2 Tut1Qn7)}
    \item $\sum_{k=1}^{n}{k {n \choose{k}}} = n2^{n-1}$ \ \emph{(AY20/21Sem2 Tut1Qn8)}
    \item ${{n+m}\choose{k}} = {{n}\choose{0}}{{m}\choose{k}} + {{n}\choose{1}}{{m}\choose{k-1}} + ... + {{n}\choose{r}}{{m}\choose{0}}$ \ \emph{(AY20/21Sem2 Tut1Qn9)}
    \item ${{2n}\choose{n}} = \sum_{k=0}^{n} {{n}\choose{k}}^2$ \ \emph{(AY20/21Sem2 Tut1Qn10)}
\end{itemize}

\section{Chapter 02 - Axioms of Probability}
\subsection{Inclusion/Exclusion Principle (Prop. 2.6)}
Let $E_1, E_2, ..., E_n$ be any $n$ events, then
\begin{align*}
P(E_1 \cup E_2 \cup ... E_n) = & \sum_{i=1}^{n}P(E_i) - \sum_{1 \leq i_1 < i_2 \leq n}P(E_{i_1} \cap E_{i_2}) + ... 
\\
&+ (-1)^{r+1} \sum_{1 \leq i_1 < ... < i_r \leq n} P(E_{i_1} \cap ... \cap E_{i_r})
\\
&+ ... + (-1)^{n+1}P(E_1 \cap ... \cap E_n)
\end{align*}

\subsection{Generalised Bonferroni's Inequality}
\emph{(AY20/21Sem2 Tut2Qn16)} Let $E_1, E_2, ..., E_n$ be any $n$ events, then
\[
P(E_1E_2 \ ... \ E_n) \geq P(E_1) + ... + P(E_n) - (n-1)
\]
When $n=2$,
\[
P(E_1E_2) \geq P(E_1) + P(E_2) - 1
\]

\end{multicols}
\end{document}
