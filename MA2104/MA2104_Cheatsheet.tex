%%%%%%%%%%%%%%%%%%%%%%%%%%%%%%%%%%%%%%%%%%%%%%%%%%%%%%%%%%%%%%%%%%%%%%
% Original Source: Dave Richeson (divisbyzero.com), Dickinson College
% Modified By: Chen Yiyang
% 
% A one-size-fits-all LaTeX cheat sheet. Kept to two pages, so it 
% can be printed (double-sided) on one piece of paper
% 
% Feel free to distribute this example, but please keep the referral
% to divisbyzero.com
% 
% Guidance on the use of the Overleaf logos can be found here:
% https://www.overleaf.com/for/partners/logos 
%%%%%%%%%%%%%%%%%%%%%%%%%%%%%%%%%%%%%%%%%%%%%%%%%%%%%%%%%%%%%%%%%%%%%%

\documentclass[10pt,landscape,letterpaper]{article}
\usepackage{amssymb}
\usepackage{amsmath}
\usepackage{amsthm}
\usepackage{physics} % for vectors
%\usepackage{fonts}
\usepackage{multicol,multirow}
\usepackage{spverbatim}
\usepackage{graphicx}
\usepackage{ifthen}
\usepackage[landscape]{geometry}
\usepackage[colorlinks=true,urlcolor=olgreen]{hyperref}
\usepackage{booktabs}
\usepackage{fontspec}
\setmainfont[Ligatures=TeX]{TeX Gyre Pagella}
\setsansfont{Fira Sans}
\setmonofont{Inconsolata}
\usepackage{unicode-math}
\setmathfont{TeX Gyre Pagella Math}
\usepackage{microtype}

\usepackage{empheq}

% new:
\def\MT@is@uni@comp#1\iffontchar#2\else#3\fi\relax{%
  \ifx\\#2\\\else\edef\MT@char{\iffontchar#2\fi}\fi
}
\makeatother

\ifthenelse{\lengthtest { \paperwidth = 11in}}
    { \geometry{margin=0.4in} }
	{\ifthenelse{ \lengthtest{ \paperwidth = 297mm}}
		{\geometry{top=1cm,left=1cm,right=1cm,bottom=1cm} }
		{\geometry{top=1cm,left=1cm,right=1cm,bottom=1cm} }
	}
\pagestyle{empty}
\makeatletter
\renewcommand{\section}{\@startsection{section}{1}{0mm}%
                                {-1ex plus -.5ex minus -.2ex}%
                                {0.5ex plus .2ex}%x
                                {\sffamily\large}}
\renewcommand{\subsection}{\@startsection{subsection}{2}{0mm}%
                                {-1explus -.5ex minus -.2ex}%
                                {0.5ex plus .2ex}%
                                {\sffamily\normalsize\itshape}}
\renewcommand{\subsubsection}{\@startsection{subsubsection}{3}{0mm}%
                                {-1ex plus -.5ex minus -.2ex}%
                                {1ex plus .2ex}%
                                {\normalfont\small\itshape}}
\makeatother
\setcounter{secnumdepth}{0}
\setlength{\parindent}{0pt}
\setlength{\parskip}{0pt plus 0.5ex}
% -----------------------------------------------------------------------

\usepackage{academicons}

\begin{document}

\definecolor{mathBlue}{cmyk}{1,.72,0,.38}
\everymath{\color{mathBlue}}
\everydisplay{\color{mathBlue}}

% for vector notation in this module
\newcommand{\vect}[1]{\boldsymbol{#1}}

\footnotesize
%\raggedright

\begin{center}
  {\huge\sffamily\bfseries MA2104 Cheatsheet} \huge\bfseries\\
  by Wei En \& Yiyang, AY21/22
\end{center}
\setlength{\premulticols}{0pt}
\setlength{\postmulticols}{0pt}
\setlength{\multicolsep}{1pt}
\setlength{\columnsep}{1.8em}
\begin{multicols}{3}




% -----------------------------------------------------------------------

\section{Chapter 01 -  Vectors in 3D Space}
\subsection{Vectors}
Vector projection of $\vb*{a}$ onto $\vb*{b}$: $\text{proj}_{\vb*{b}} \vb*{a} = \frac{\vb*{a} \cdot \vb*{b}}{\vb*{b} \cdot \vb*{b}} \vb*{b}$
\\
Scalar projection of $\vb*{a}$ onto $\vb*{b}$: $\text{comp}_{\vb*{b}} \vb*{a} = \frac{\vb*{a} \cdot \vb*{b}}{\norm{\vb*{b}}}$

\subsection{Dot \& Cross Product}
$$\vb*{a} \cdot \vb*{b} = \|a\|\|b\|\cos\theta,\quad
\|\vb*{a} \times \vb*{b}\| = \|a\|\|b\|\sin\theta$$ where $\theta$ is the angle between vectors $\vb*{a}$ and $\vb*{b}$.

\subsection{Prop Ch01.3.5 - Scalar Triple Product}
$$|\vb*{a} \cdot (\vb*{b} \cross \vb*{c})| = \left|\det\begin{pmatrix}
    \vb*{a}_1 & \vb*{a}_2 & \vb*{a}_3 \\
    \vb*{b}_1 & \vb*{b}_2 & \vb*{b}_3 \\
    \vb*{c}_1 & \vb*{c}_2 & \vb*{c}_3 \\
\end{pmatrix}\right|$$ is the volume of the parallelepiped determined by vectors $\vb*{a}, \vb*{b}, \vb*{c}$.

\section{Chapter 02 - Curves and Surfaces}
\subsection{Curve}
\subsubsection{Tangent Vector}
Tangent vector to a curve $C$ paramaterised by $R(t) = (f(t), g(t), h(t))$ at $R(a)$ on the curve is given by
\[
R'(a) = \langle f'(a), g'(a), h'(a) \rangle.
\]

\subsubsection{Arc Length Formula} 
The length of curve $C: R(t) = (f(t), g(t), h(t))$ between $R(a)$ and $R(b)$ is
\[
\int_a^b \ \norm{R'(t)} \,dt = \int_a^b \ \sqrt{f'(t)^2 + g'(t)^2 + h'(t)^2} \,dt.
\]
provided the first derivatives are continuous.


\subsection{Surfaces}
\subsubsection{Cylinder}
A surface is a cylinder if there is a plane $P$ such that all the planes parallel to $P$ intersect the surface in the same curve.

\subsubsection{Quadric Surfaces}
\begin{itemize}
    \item Elliptic Paraboloid: $\frac{x^2}{a^2} + \frac{y^2}{b^2} = \frac{z}{c}$
    \item Hyperbolic Paraboloid: $\frac{x^2}{a^2} - \frac{y^2}{b^2} = \frac{z}{c}$
    \item Ellipsoid: $\frac{x^2}{a^2} + \frac{y^2}{b^2} + \frac{z^2}{c^2} = 1$
    \item Elliptic Cone: $\frac{x^2}{a^2} + \frac{y^2}{b^2} - \frac{z^2}{c^2} = 0$
    \item Hyperboloid of 1 Sheet: $\frac{x^2}{a^2} + \frac{y^2}{b^2} - \frac{z^2}{c^2} = 1$
    \item Hyperboloid of 2 Sheet: $\frac{x^2}{a^2} + \frac{y^2}{b^2} - \frac{z^2}{c^2} = -1$
\end{itemize}



\section{Chapter 03 - Multivariable Functions}
\subsection{Limit, Continuity \& Differentiability}
\subsubsection{Limit for 2D Functions}
For function $f$ with domain $D \subset \mathbb{R^2}$ that contains points arbitrarily close to $(a, b)$, then
\[
\lim_{(x, y) \to (a, b)} f(x, y) = L
\]
if for any number $\epsilon > 0$ there exists a number $\delta > 0$ such that $|f(x,y) - L| < \epsilon$ whenever $0 < \sqrt{(x-a)^2 + (y-b)^2} < \delta$.

The limit exists iff. the limit \textbf{exists and is the same for all continuous paths} to $(a, b)$.

\subsubsection{Clairaut's Theorem}
For function $f$ defined on $D \subset \mathbb{R}^2$ that contains $(a, b)$, if the functions $f_{xy}$ and $f_{yx}$ are both continuous on $D$, then
\[
f_{xy}(a,b) = f_{yx}(a, b).
\]

\subsubsection{Differentiability for 2D Functions}
For function $f$ defined on $D \subset \mathbb{R}^2$ and differentiable at $(a, b)$ within the interior of $D$,
\[
\lim_{(h, k) -> (0,0)} \ \frac{f(a+h, b+k)-f(a,b) - L(h,k)}{\sqrt{h^2+k^2}} = 0,
\]
where $L: \mathbb{R}^2 \to \mathbb{R}$ is a linear map defined as the total derivative of $f$ at $(a, b)$:
\[
L(h, k) = D_{f(a, b)} (h, k) = f_x(a, b)h + f_y(a, b)k.
\]

\subsubsection{Notes about Differentiability}
Consider $f$ at a point $(a, b)$:
\begin{itemize}
    \item $f_x$ and $f_y$ exist $\not \implies$ $f$ differentiable
    \item $f_x$ and $f_y$ exist \& continuous $\implies$ $f$ differentiable \emph{(Differentiability Theorem)}
    \item  $f$ differentiable $\not{\implies}$ $f_x$ and $f_y$ continuous
\end{itemize}

\subsubsection{Linear Approximation}
\[
f(a+h, b+k) \approx f(a,b) + f_x(a, b)h + f_y(a, b)k
\]


\subsection{Gradient Vector}
\subsubsection{Gradient Vector}
The gradient vector of $f$ defined on $D \subset \mathbb{R}^2$ at $(a, b) \in D$ is defined as:
\[
\nabla f(a,b) = \langle f_x(a, b), f_y(a, b) \rangle
\]

\subsubsection{Directional Directive}
The directional directive of $f$ defined on $D \subset \mathbb{R}^2$ in the direction of the unit vector $\vb*{u} = \langle \vb*{u}_1, \vb*{u}_2 \rangle$ is
\[
D_{f(a,b)}(\vb*{u}) = \lim_{h \to 0} \frac{f(a + h\vb*{u}_1, b + h\vb*{u}_2) - f(a, b)}{h} = \nabla f(a,b) \cdot \vb*{u}.
\]

\subsubsection{Perpendicular Vector of Level Sets}
$\nabla f(a,b)$ is orthogonal to the $f(a,b)$-level curve of $f$ at $(a,b)$.

\section{Chapter 04 - Calculus on Surfaces}
\subsection{Implicit Differentiation}
\subsubsection{Prop Ch04.1.4}
For $F$ defined on $D \subset \mathbb{R}^3$ where $F(a,b,c)=k$ defines $z$ as a differentiable function of $x$ and $y$ near $(a,b,c)$, and $F_z(a,b,c) \neq 0$,
\[
\frac{\partial z}{\partial x}(a,b,c) = - \frac{F_x(a,b,c)}{F_z(a,b,c)}, \ 
\frac{\partial z}{\partial y}(a,b,c) = - \frac{F_y(a,b,c)}{F_z(a,b,c)}
\]

\subsection{Extrema}
\subsubsection{Extreme Value Theorem}
If $f : D \to \mathbb{R}$ is continuous on a \textbf{closed and bounded} set $D \subset \mathbb{R}^2$, then $f$ has at least one global maximum and one global minimum.

\subsubsection{Steps for Finding Global Extrema}
For $f : D \to \mathbb{R}$ where $D$ is closed and bounded,
\begin{enumerate}
    \item Find all critical points of $f$ and their corresponding $f$-values.
    \item Find the extreme values of $f$ on boundary of $D$.
    \item Compare.
\end{enumerate}


\subsubsection{Method of Lagrange Multiplier}
To find the extrema of differentiable $f : D \to \mathbb{R}$ subject to curve $C: g(x,y)=k$ for some $k \in \mathbb{R}$,
\begin{enumerate}
    \item Find all points $(a,b)$ for $\nabla g(a,b) \ne 0$ and values $\lambda$ s.t.
    \[
    \nabla f(a,b) = \lambda \nabla g(a,b), \ g(a,b) = k,
    \] and evaluate $f$ at all these points.
    \item Find the extreme values of $f$ on the boundary of $C$.
    \item Compare.
\end{enumerate}



\section{Chapter 05 \& 06 - Integration}
\subsection{Fubini's Theorem}
If $f$ is continuous on the rectangle $D = [a,b] \times [c,d]$, then,
\[
\iint_D f(x,y) dA = \int_{a}^{b} \int_{c}^{d} f(x,y) dy dx = \int_{c}^{d}\int_{a}^{b} f(x,y) dx dy 
\]

Its equivalence in $\mathbb{R}^3$ for triple integral also holds.

\subsection{Change of Coordinates}
\subsubsection{Double Integral in Polar Coordinates}
Transform between $(x,y)$ and $(r, \theta)$:
\begin{align*}
&\left\{
\begin{aligned}
x &= r \cos\theta \\
y &= r \sin\theta \\
\end{aligned}\right.
&
\left\{\begin{aligned}
r &= \sqrt{x^2 + y^2} \\
\theta &= \tan^{-1}{(y/x)} \\
\end{aligned}\right.
\end{align*}
In addition, $dA = dx dy = r dr d\theta$.

\subsubsection{Triple Integral in Cylindrical Coordinates}
Transform between $(x,y,z)$ and $(r, \theta, z)$:
\begin{align*}
&\left\{
\begin{aligned}
x &= r \cos\theta \\
y &= r \sin\theta \\
z &= z \\
\end{aligned}\right.
&
\left\{\begin{aligned}
r &= \sqrt{x^2 + y^2} \\
\theta &= \tan^{-1}{(y/x)} \\
z &= z \\
\end{aligned}\right.
\end{align*}
In addition, $dV = dx dy dz = r dr d\theta dz$.

\subsubsection{Triple Integral in Spherical Coordinates}
Transform between $(x,y,z)$ and $(\rho, \theta, \phi)$:
\begin{align*}
&\left\{
\begin{aligned}
x &= \rho \cos\theta \sin\phi \\
y &= \rho \sin\theta \sin\phi \\
z &= \rho \cos\phi
\end{aligned}\right.
&
\left\{\begin{aligned}
\rho &= \sqrt{x^2 + y^2 + z^2} \\
\theta &= \tan^{-1}{(y/x)} \\
\phi &= \cos^{-1}(z/\rho) \le \pi \\
\end{aligned}\right.
\end{align*}
In addition, $dV = dx dy dz = \rho^2 \sin\phi d\rho d\theta d\phi$.


\subsection{Application of integration}
For a given region $D$ in $\mathbb{R}^2$, the area of the region can be calculated as $\text{Area}(D) = \iint_D 1 dA$.
\\
For a given solid $E$ in $\mathbb{R}^3$, the volume of the solid can be calculated as $\text{Volume}(E) = \iiint_E 1 dV$.


\section{Chapter 07 - Change of Coordinates}
\subsection{Planar Transformation}
A map $T: S \to R$ is a planar transformation if it is a differentiable map whose inverse is differentiable.

Therefore, to show $T: S \to R$ is a planar transformation, it must satisfy:
\begin{enumerate}
    \item $T$ is differentiable
    \item The inverse $T^{-1}$ exists
    \item The inverse $T^{-1}$ is differentiable
\end{enumerate}

\subsection{Change of Coordinates}
\subsubsection{2D Jacobian Determinant}
The \emph{Jacobian} of the transformation $T(u, v) = (x(u,v), y(u,v))$ is defined
\[
\frac{\partial (x,y)}{\partial (u,v)} 
    = \begin{vmatrix} x_u & x_v \\ y_u & y_v\end{vmatrix}
    = \frac{\partial x}{\partial u} \frac{\partial y}{\partial v}
        - \frac{\partial x}{\partial v} \frac{\partial y}{\partial u}
\]

\subsubsection{Change of Variable in Double Integral}
Let $T : S \to R$ be a planar transformation, where $S$ lies in the $uv$-plane and $R$ lies in the $xy$-plane. Let $A$ and $A'$ denote the area in the $xy$- and $uv$-plane respectively. For a two-var. function from $xy$-plane to $\mathbb{R}$,
\[
\iint_{R} f(x,y) dA = \iint_{R} f \circ T(u, v) \left| \frac{\partial (x,y)}{\partial (u,v)}  \right|  dA'
\]
Equivalently,
\[
\iint_{R} f(x,y)dx dy = \iint_S f(x(u,v), y(u,v)) \left| \frac{\partial (x,y)}{\partial (u,v)}  \right|  du dv
\]

The 3D Jacobian has similar expressions and properties.

\subsection{Inverse Function Theorem}

\section{Chapter 08 - Line Integrals}
\subsection{Line Integral}
\subsubsection{Line Integral of Functions}
For curve $C$ parameterised by $R(t) = (x(t), y(t), z(t), \ a \le t \le b$, and a 3-var function $f(x,y, z)$, the line integral
\[
\int_C f(x, y, z) ds = \int_{a}^{b} f(x(t), y(t), z(t)) \norm{R'(t)} dt
\]

\subsubsection{Line Integral of Vector Fields}
Let $\vb{C} = (C, o)$ be a smooth oriented curve in $\mathbb{R}^3$ parameterised by $R(t) = (x(t), y(t), z(t)), \ a \le t \le b$, and let $\vb{F} = \vb{F}(x, y, z)$ be a continuous vector field along $C$. Then, the line integral of $\vb{F}$ along $\vb{C}$
\[
\int_{\vb{C}} \vb{F} \cdot d\vb{r} = \int_{a}^{b} \vb{F}(x(t), y(t), z(t)) \cdot R'(t) dt
\]
For $\vb{F} = \langle X,Y,Z \rangle$ in its component form,
\[
\int_{\vb{C}} \vb{F} \cdot d\vb{r} = \int_{\vb{C}} Xdx + Ydy + Zdz
\]

In addition, if $-\vb{C}$ is the curve $\vb{C}$ with opposite orientation,
\[
\int_{\vb{C}} \vb{F} \cdot d\vb{r} = - \int_{-\vb{C}} \vb{F} \cdot d\vb{r}
\]


\subsection{Conservative Vector Fields}
Whenever a vector field $\vb{F}$ on some open domain stratifies $\vb{F} = \nabla f$ for some differentiable function $f$, then we call $\vb{F}$ a \emph{conservative vector field} and $f$ the \emph{potential function} of $\vb{F}$.

\subsubsection{Tests for Conservativity}
\begin{itemize}
    \item To show conservative: find a $f$ s.t. $\nabla f = \vb{F}$ (by definition).
    \item To show conservative: if $\vb{F}(x, y) = \langle X,Y \rangle$ is defined over an \textbf{open and simply-connected} region $D \subset \mathbb{R}^2$, then need to show
    \[
    \frac{\partial X}{\partial y} = \frac{\partial Y}{\partial x}
    \]
    \item To show non-conservative: find two oriented curves, $\vb{C_1}$ and $\vb{C_2}$, with same starting and ending points, s.t.
    \[
    \int_{\vb{C_1}} \vb{F} \cdot d\vb{r} \ne \int_{\vb{C_2}} \vb{F} \cdot d\vb{r}
    \]
\end{itemize}

\subsubsection{Gradient Theorem}
For a 3-var function $f$ whose gradient vector $\nabla f$ is continuous along $\vb{C} = (C, 0)$ parameterised by $R(t) = (x(t), y(t), z(t)), \ a \le t \le b$,
\[
\int_{\vb{C}} \nabla f \cdot d\vb{r} = f(R(b)) - f(R(a))
\]

\subsubsection{Green's Theorem, Version I}
Let $\vb{C} = (C, o)$ be a \textbf{positively oriented}, piecewise differentiable loop in $\mathbb{R}^2$, and let $D$ be the region bounded by $C$. Then for vector field $\vb{F} = \langle X, Y \rangle$,
\[
\int_{\vb{C}} \vb{F} \cdot d\vb{r} = \iint_{D} \left( \frac{\partial Y}{\partial x} - \frac{\partial X}{\partial y} \right) \,dA
\]

\subsubsection{Green's Theorem, Version II}
Let $\vb{C} = (C, o)$ be a \textbf{positively oriented}, piecewise differentiable loop in $\mathbb{R}^2$, and let $D$ be the region bounded by $C$. Then for vector field $\vb{F}$, let $\vb{n}(x, y)$ denote the \textbf{outward pointing} unit normal vector to $S$,
\[
\int_{\vb{C}} \vb{F} \cdot \vb{n} \,ds = \iint_{D} \text{div } \vb{F} \,dA
\]
Algebraically, the outward pointing unit normal vector is
\[
\vb{n}(t) = \frac{\langle y'(t), -x'(t) \rangle}{\norm{R'(t)}}
= \frac{\langle y'(t), -x'(t) \rangle}{\sqrt{x'(t)^2 + y'(t)^2}}
\]
Note, the integral is called the outward flux of $\vb{F}$ across $C$.


\section{Chapter 09 - Surface Integrals}
\subsection{Surface Integral}
\subsubsection{Surface Integral of Functions}
Let $R: D \to S$ be a (differentiable) parameterisation of surface $S$, then
\[
\iint_{S} f(x,y,z) dS = \iint_{D} f(x(u,v), y(u,v), z(u,v)) \norm{R_{u} \times R_{v}} \,dA
\]
\textbf{Special case:} when $S$ is the graph of a 2-var function $g(x,y)$ for $(x,y) \in D$ for some domain $D$,
\[
\iint_{S} f(x,y,z) dS = \iint_{D} f(x,y,g(x,y)) \Big( \sqrt{g_{x}^2 + g_{y}^2 + 1}  \Big) \,dA
\]

\subsubsection{Orientation on Surface}
A (differentiable) surface $S \in \mathbb{R}^3$ is \emph{orientable} if it is possible to define for every $(x,y,z) \in S$, a unit normal vector $\vb{n}(x,y,z)$ to $S$ with initial point $(x,y,z)$ such that $\vb{n}$ varies continuously.   
\\
An orientable surface has two orientations:
\[
\vb{n} = \pm \frac{R_u \times R_v}{\norm{R_u \times R_v}}
\]

\subsubsection{Surface Integral of Vector Fields}
Let $\vb{S} = (S, \vb{n})$ be an oriented surface where $R: D \to S$ is the parameterisation for $S$, and let $\vb{F}$ be a vector field along the surface,
\[
\iint_{\vb{S}} \vb{F} \cdot d\vb{S} 
= \iint_{\vb{S}} \vb{F} \cdot \vb{n} \,dS 
= \iint_D \vb{F} \cdot (R_u \times R_v) \,dA
\]
\textbf{Special case:} when $S$ is the graph of a 2-var function $g(x,y)$, let $\vb{S}$ denote $S$ with \textbf{upward orientation}, and let $\vb{F} = \langle X, Y, Z \rangle$,
\[
\iint_{\vb{S}} \vb{F} \cdot d\vb{S} 
= \iint_D \Big( -\vb{X}\frac{\partial g}{\partial x} -\vb{Y}\frac{\partial g}{\partial Y} + \vb{Z} \Big) \,dA
\]

\subsection{Gauss' Theorem}
\subsubsection{Divergence}
For a vector field $\vb{F} = \langle X, Y, Z \rangle$, the \emph{divergence} of $\vb{F}$ is defined as
\[
\text{div } \vb{F} = \nabla \cdot \vb{F}
= \frac{\partial X}{\partial x}(x,y,z) + \frac{\partial Y}{\partial y}(x,y,z) + \frac{\partial Z}{\partial z}(x,y,z)
\]

\subsubsection{Gauss' Theorem}
Let $E$ be a solid region where the boundary surface $S$ is piece-wise smooth, and let $\vb{S}$ denote $S$ with \textbf{outward orientation}. Then for a vector field $\vb{F}$ defined over $S$,
\[
\iint_{\vb{S}} \vb{F} \cdot d\vb{S} = \iiint_E \text{div } \vb{F} \,dV
\]


\subsection{Stokes' Theorem}
\subsubsection{Curl}
For a vector field $\vb{F} = \langle X,Y,Z \rangle$, the \emph{curl} of $\vb{F}$ is defined as
\[
\text{curl } \vb{F} = \nabla \times \vb{F} = \left\langle 
        \frac{\partial Z}{\partial y} - \frac{\partial Y}{\partial z},
        \frac{\partial X}{\partial z} - \frac{\partial Z}{\partial x},
        \frac{\partial Y}{\partial x} - \frac{\partial X}{\partial y}
        \right\rangle
\]

\subsubsection{Induced Orientation}
For oriented surface $\vb{S} = (S, \vb{n})$ with boundary $C$ being a simple loop, the \emph{induced orientation}, $\vb{o}$ of $\vb{n}$ is one such that if you want along $C$ in the orientation $\vb{o}$ with your head pointing in the direction of $\vb{n}$, then $S$ will always be on your left.

\subsubsection{Stokes' Theorem}
For oriented surface $\vb{S} = (S, \vb{n})$ bounded by a simple curve $C$, and let $\vb{C} = (C, \vb{o})$ be the oriented loop with induced orientation, then for a vector field $\vb{F}$,
\[
\iint_{\vb{S}} \text{curl } \vb{F} \cdot d\vb{S} = \int_{\vb{C}} \vb{F} \cdot d\vb{r}
\]

\subsection{Properties of Gradient, Divergence \& Curl}
For any function $f = f(x,y,z)$,
\[
\nabla \times (\nabla f) = \text{curl }( \nabla f ) = \vb{0}
\]

For any vector fields $\vb{F} = \vb{F}(x,y,z)$,
\[
\nabla \cdot (\nabla \times \vb{F} )
= \text{div } ( \text{curl } \vb{F} )= 0
\]


\end{multicols}
\end{document}
