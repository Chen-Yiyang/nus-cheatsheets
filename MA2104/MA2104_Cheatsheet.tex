%%%%%%%%%%%%%%%%%%%%%%%%%%%%%%%%%%%%%%%%%%%%%%%%%%%%%%%%%%%%%%%%%%%%%%
% Original Source: Dave Richeson (divisbyzero.com), Dickinson College
% Modified By: Chen Yiyang
% 
% A one-size-fits-all LaTeX cheat sheet. Kept to two pages, so it 
% can be printed (double-sided) on one piece of paper
% 
% Feel free to distribute this example, but please keep the referral
% to divisbyzero.com
% 
% Guidance on the use of the Overleaf logos can be found here:
% https://www.overleaf.com/for/partners/logos 
%%%%%%%%%%%%%%%%%%%%%%%%%%%%%%%%%%%%%%%%%%%%%%%%%%%%%%%%%%%%%%%%%%%%%%

\documentclass[10pt,landscape,letterpaper]{article}
\usepackage{amssymb}
\usepackage{amsmath}
\usepackage{amsthm}
\usepackage{physics} % for vectors
%\usepackage{fonts}
\usepackage{multicol,multirow}
\usepackage{spverbatim}
\usepackage{graphicx}
\usepackage{ifthen}
\usepackage[landscape]{geometry}
\usepackage[colorlinks=true,urlcolor=olgreen]{hyperref}
\usepackage{booktabs}
\usepackage{fontspec}
\setmainfont[Ligatures=TeX]{TeX Gyre Pagella}
\setsansfont{Fira Sans}
\setmonofont{Inconsolata}
\usepackage{unicode-math}
\setmathfont{TeX Gyre Pagella Math}
\usepackage{microtype}

\usepackage{empheq}

% new:
\def\MT@is@uni@comp#1\iffontchar#2\else#3\fi\relax{%
  \ifx\\#2\\\else\edef\MT@char{\iffontchar#2\fi}\fi
}
\makeatother

\ifthenelse{\lengthtest { \paperwidth = 11in}}
    { \geometry{margin=0.4in} }
	{\ifthenelse{ \lengthtest{ \paperwidth = 297mm}}
		{\geometry{top=1cm,left=1cm,right=1cm,bottom=1cm} }
		{\geometry{top=1cm,left=1cm,right=1cm,bottom=1cm} }
	}
\pagestyle{empty}
\makeatletter
\renewcommand{\section}{\@startsection{section}{1}{0mm}%
                                {-1ex plus -.5ex minus -.2ex}%
                                {0.5ex plus .2ex}%x
                                {\sffamily\large}}
\renewcommand{\subsection}{\@startsection{subsection}{2}{0mm}%
                                {-1explus -.5ex minus -.2ex}%
                                {0.5ex plus .2ex}%
                                {\sffamily\normalsize\itshape}}
\renewcommand{\subsubsection}{\@startsection{subsubsection}{3}{0mm}%
                                {-1ex plus -.5ex minus -.2ex}%
                                {1ex plus .2ex}%
                                {\normalfont\small\itshape}}
\makeatother
\setcounter{secnumdepth}{0}
\setlength{\parindent}{0pt}
\setlength{\parskip}{0pt plus 0.5ex}
% -----------------------------------------------------------------------

\usepackage{academicons}

\begin{document}

\definecolor{mathBlue}{cmyk}{1,.72,0,.38}
\everymath{\color{mathBlue}}
\everydisplay{\color{mathBlue}}

\footnotesize
%\raggedright

\begin{center}
  {\huge\sffamily\bfseries MA2104 Cheatsheet} \huge\bfseries\\
  by Yiyang, AY21/22
\end{center}
\setlength{\premulticols}{0pt}
\setlength{\postmulticols}{0pt}
\setlength{\multicolsep}{1pt}
\setlength{\columnsep}{1.8em}
\begin{multicols}{3}




% -----------------------------------------------------------------------

\section{Chapter 01 -  Vectors in 3D Space}
\subsection{Vectors}
Vector project of $\vb*{a}$ onto $\vb*{b}$: $\text{proj}_{\vb*{b}} \vb*{a} = \frac{\vb*{a} \cdot \vb*{b}}{\vb*{b} \cdot \vb*{b}} \vb*{b}$
\\
Scalar project of $\vb*{a}$ onto $\vb*{b}$: $\text{comp}_{\vb*{b}} \vb*{a} = \frac{\vb*{a} \cdot \vb*{b}}{\norm{\vb*{b}}}$

\subsection{Prop Ch01.3.5 - Scalar Triple Product}
$|\vb*{a} \cdot (\vb*{b} \cross \vb*{c})|$ is the volume of the parallelepiped determined by the vectors $\vb*{a}$, $\vb*{b}$, and $\vb*{c}$.

\section{Chapter 02 - Curves and Surfaces}
\subsection{Curve}
\subsubsection{Tangent Vector}
Tangent vector to a curve $C$ paramaterised by $R(t) = (f(t), g(t), h(t))$ at $R(a)$ on the curve is given by:
\[
R'(a) = \langle f'(a), g'(a), h'(a) \rangle
\]
, given that the three component functions are all differentiable at $a$.


\subsubsection{Arc Length Formula} 
The length of curve $C: R(t) = (f(t), g(t), h(t))$ between $R(a)$ and $R(b)$ is:
\[
\int_a^b \ \norm{R'(t)} dt = \int_a^b \ \sqrt{f'(t)^2 + g'(t)^2 + h'(t)^2} dt
\]
, given that the component functions are differentiable and their corresponding first derivatives are continuous.


\subsection{Surfaces}
\subsubsection{Cylinder}
A surface is a cylinder if there is a plane $P$ such that all the planes parallel to $P$ intersect the surface in the same curve.

\subsubsection{Quadric Surfaces}
Below are a list of common quadric surfaces with their equations
\begin{itemize}
    \item Elliptic Paraboloid - $\frac{x^2}{a^2} + \frac{y^2}{b^2} = \frac{z}{c}$
    \item Hyperbolic Paraboloid - $\frac{x^2}{a^2} - \frac{y^2}{b^2} = \frac{z}{c}$
    \item Ellipsoid - $\frac{x^2}{a^2} + \frac{y^2}{b^2} + \frac{z^2}{c^2} = 1$
    \item Elliptic Cone - $\frac{x^2}{a^2} + \frac{y^2}{b^2} - \frac{z^2}{c^2} = 0$
    \item Hyperboloid of 1 Sheet - $\frac{x^2}{a^2} + \frac{y^2}{b^2} - \frac{z^2}{c^2} = 1$
    \item Hyperboloid of 2 Sheet - $\frac{x^2}{a^2} + \frac{y^2}{b^2} - \frac{z^2}{c^2} = -1$
\end{itemize}



\section{Chapter 03 - Multivariable Functions}
\subsection{Limit, Continuity \& Differentiability}
\subsubsection{Limit for 2D Functions}
Limit for two variable functions: For function $f$ with domain $D \subset \mathbb{R^2}$ that contains points arbitrarily close to $(a, b)$, then
\[
\lim_{(x, y) \to (a, b)} f(x, y) = L
\]
if for any number $\epsilon > 0$ there exists a number $\delta > 0$ such that $|f(x,y) - L| < \epsilon$ whenever $0 < \sqrt{(x-a)^2 + (y-b)^2} < \delta$.

The limit exists iff. the limit \textbf{exists and is the same for all continuous paths} to $(a, b)$.

\subsubsection{Clairaut's Theorem}
For function $f$ defined on $D \subset \mathbb{R}^2$ that contains $(a, b)$, if the functions $f_{xy}$ and $f_{yx}$ are both defined and continuous on $D$, then
\[
f_{xy}(a,b) = f_{yx}(a, b)
\]

\subsubsection{Differentiability for 2D Functions}
If $f$ is a 2-var function differentiable at $(a,b)$ in the interior of its domain, then
\[
\lim_{(h, k) -> (0,0)} \ \frac{f(a+h, b+k)-f(a,b) - L(h,k)}{\sqrt{h^2+k^2}} = 0
\]
, where $L: \mathbb{R}^2 \to \mathbb{R}$ is a linear map, and it is equal to the total derivative of $f$ at $(a, b)$ \textbf{when $f$ is differentiable at $(a, b)$}. Then it can be defined as: 
\[
L(h, k) = D_{f(a, b)} (h, k) = f_x(a, b)h + f_y(a, b)k
\]

\subsubsection{Notes about Differentiability}
Consider $f$ at a point $(a, b)$:
\begin{itemize}
    \item $f_x$ and $f_y$ exist $\not \implies$ $f$ differentiable
    \item $f_x$ and $f_y$ exist \& continuous $\implies$ $f$ differentiable \emph{(Differentiability Theorem)}
    \item  $f$ differentiable $\not{\implies}$ $f_x$ and $f_y$ continuous. (i.e. the converse does not hold.)
\end{itemize}

\subsubsection{Linear Approximation}
\[
f(a+h, b+k) \approx f(a,b) + f_x(a, b)h + f_y(a, b)k
\]


\subsection{Gradient Vector}
\subsubsection{Gradient Vector}
The gradient vector of $f$ at $(a, b)$ in its domain is defined as:
\[
\nabla f(a,b) = \langle f_x(a, b), f_y(a, b) \rangle
\]
Therefore if $f$ is differentiable at $(a,b)$ then 
\[
D_{f(a,b)}(\vb*{u}) = \nablaf(a,b) \cdot \vb*{u}
\]

\subsubsection{Perpendicular Vector of Level Sets}
\begin{itemize}
    \item $\nabla f(a,b)$ is orthogonal to the $f(a,b)$-level curve of $f$ at $(a,b)$.
    \item $\nabla f(a,b,c)$ is orthogonal to the $f(a,b,c)$-level surface of $f$ at $(a,b,c)$.
\end{itemize}



\section{Chapter 04 - Calculus on Surfaces}
\subsection{Implicit Differentiation}
\subsubsection{Prop Ch04.1.4}
For a 3-var function $F$ where $F(a,b,c)=k$ defines $z$ as a differentiable function of $x$ and $y$ near $(a,b,c)$, and $F_z(a,b,c) \neq 0$, then
\[
\frac{\partial z}{\partial x}(a,b,c) = - \frac{F_x(a,b,c)}{F_z(a,b,c)}, \ 
\frac{\partial z}{\partial y}(a,b,c) = - \frac{F_y(a,b,c)}{F_z(a,b,c)}
\]

\subsection{Extrema}
\subsubsection{Extreme Value Theorem}
If $f : D \to \mathbb{R}^2$ is continuous on a \textbf{closed and bounded} set $D \subset \mathbb{R}^2$, then $f$ has at least one global maximum and one global minimum.

\subsubsection{Steps for Finding Gloabl Extrema}
For $f : D \to \mathbb{R}^2$ where $D$ is closed and bounded,
\begin{enumerate}
    \item Find all critical points of $f$ and their corresponding $f$-values.
    \item Find the extreme values of $f$ on boundary of $D$.
    \item Compare.
\end{enumerate}


\subsubsection{Lagrange Multiplier}
To find the extrema of differentiable $f : D \to \mathbb{R}^2$ subject to curve $C: g(x,y)=k$ for some $k \in \mathbb{R}$,
\begin{enumerate}
    \item Find all points $(a,b)$ and \textbf{non-zero} value $\lambda$ s.t.
    \[
    \nabla f(a,b) = \lambda \nabla g(a,b), \ g(a,b) = k
    \], and valuate $f$ at all these points.
    \item Find the extreme values of $f$ on the boundary of $C$.
    \item Compare.
\end{enumerate}



%\section{Chapter 05 - }



\end{multicols}
\end{document}
