%%%%%%%%%%%%%%%%%%%%%%%%%%%%%%%%%%%%%%%%%%%%%%%%%%%%%%%%%%%%%%%%%%%%%%
% Original Source: Dave Richeson (divisbyzero.com), Dickinson College
% Modified By: Chen Yiyang
% 
% A one-size-fits-all LaTeX cheat sheet. Kept to two pages, so it 
% can be printed (double-sided) on one piece of paper
% 
% Feel free to distribute this example, but please keep the referral
% to divisbyzero.com
% 
% Guidance on the use of the Overleaf logos can be found here:
% https://www.overleaf.com/for/partners/logos 
%%%%%%%%%%%%%%%%%%%%%%%%%%%%%%%%%%%%%%%%%%%%%%%%%%%%%%%%%%%%%%%%%%%%%%

\documentclass[10pt,landscape,letterpaper]{article}
\usepackage{amssymb}
\usepackage{amsmath}
\usepackage{amsthm}
\usepackage{physics}  % for vectors
\usepackage{bbm}  % for mathbb-ed digits
%\usepackage{fonts}
\usepackage{multicol,multirow}
\usepackage{spverbatim}
\usepackage{graphicx}
\usepackage{ifthen}
\usepackage[landscape]{geometry}
\usepackage[colorlinks=true,urlcolor=olgreen]{hyperref}
\usepackage{booktabs}
\usepackage{fontspec}
\setmainfont[Ligatures=TeX]{TeX Gyre Pagella}
\setsansfont{Fira Sans}
\setmonofont{Inconsolata}
\usepackage{unicode-math}
\setmathfont{TeX Gyre Pagella Math}
\usepackage{microtype}

\usepackage{empheq}

% new:
\def\MT@is@uni@comp#1\iffontchar#2\else#3\fi\relax{%
  \ifx\\#2\\\else\edef\MT@char{\iffontchar#2\fi}\fi
}
\makeatother

\ifthenelse{\lengthtest { \paperwidth = 11in}}
    { \geometry{margin=0.4in} }
	{\ifthenelse{ \lengthtest{ \paperwidth = 297mm}}
		{\geometry{top=1cm,left=1cm,right=1cm,bottom=1cm} }
		{\geometry{top=1cm,left=1cm,right=1cm,bottom=1cm} }
	}
\pagestyle{empty}
\makeatletter
\renewcommand{\section}{\@startsection{section}{1}{0mm}%
                                {-1ex plus -.5ex minus -.2ex}%
                                {0.5ex plus .2ex}%x
                                {\sffamily\large}}
\renewcommand{\subsection}{\@startsection{subsection}{2}{0mm}%
                                {-1explus -.5ex minus -.2ex}%
                                {0.5ex plus .2ex}%
                                {\sffamily\normalsize\itshape}}
\renewcommand{\subsubsection}{\@startsection{subsubsection}{3}{0mm}%
                                {-1ex plus -.5ex minus -.2ex}%
                                {1ex plus .2ex}%
                                {\normalfont\small\itshape}}
\makeatother
\setcounter{secnumdepth}{0}
\setlength{\parindent}{0pt}
\setlength{\parskip}{0pt plus 0.5ex}
% -----------------------------------------------------------------------

\usepackage{academicons}

\begin{document}

\definecolor{mathBlue}{cmyk}{1,.72,0,.38}
\definecolor{defOrange}{cmyk}{0, 0.5, 1, 0.3}
\definecolor{codeInlineRed}{cmyk}{0, 0.9, 0.9, 0.45}

\everymath{\color{mathBlue}}
\everydisplay{\color{mathBlue}}

% for vector notation in this module
\newcommand{\vect}[1]{\pmb{#1}}
\newcommand{\deff}[1]{\textcolor{defOrange}{\textbf{#1}}}
\newcommand{\codein}[1]{\textcolor{codeInlineRed}{\texttt{#1}}}
\newcommand{\citeqn}[1]{\underline{\textit{#1}}}

\footnotesize
%\raggedright

\begin{center}
  {\huge\sffamily\bfseries CS4247 Cheatsheet} \huge\bfseries\\
  by Yiyang, AY22/23
\end{center}
\setlength{\premulticols}{0pt}
\setlength{\postmulticols}{0pt}
\setlength{\multicolsep}{1pt}
\setlength{\columnsep}{1.8em}
\begin{multicols}{3}


% -----------------------------------------------------------------------
%\section{1. Modern OpenGL}
%\section{2. Geometric Transformation}
%\section{3. GLSL}

\section{Introduction}
\subsection{OpenGL Shaders}
\subsubsection{Vertex Shader}

\subsubsection{Fragment Shader}


\section{Texture Mapping}
\subsection{OpenGL Cube Map}





\section{Shaders}
\subsection{Cartoon Style Shading}
%TODO: very view-dependent

\subsection{Procedural Bump \& Normal Mapping}
Procedures for Procedural Bump Mapping:
\begin{enumerate}
	\item At each vertex, transform normal  \& tangent to eye space
	\item Interpolate eye-space normal \& tangent to fragments
	\item At each fragment, compute \deff{Binormal vector} \& compute tangent-space \deff{Perturbantion Vector} based on texture coordinates
	\item Transofmration perturbantion from tangent space to eye space and use it for lighting computation
\end{enumerate}
\deff{Binormal Vector} - unit vector perpendicular to both normal \& tangent vectors.

\smallskip

In \deff{Normal Mapping}, use a normal map where each texel contains 3 perturbation vector components in  RGB channels.
\\
\underline{Note}: Need to encode from $[–1,1]$ to $[0,1]$ for each value read.


\subsection{Refraction \& Reflection Mapping}
Procedures:
\begin{enumerate}
	\item Set up an \deff{Environment Map} (e.g. cubemap)
	\item Compute the refracted ray of the view ray
	\item Use (interpolated) refraction ray direction to access env.map at fragment shader
\end{enumerate}

\underline{Note}: [1] We only compute refraction at the first surface (entering the object) and ignore when the ray exits the object since it is too complicated. [2] Refraction mapping is similar to reflection mapping, but need to account for Fresnel Effects and Chromatic Aberration.

\subsubsection{Fresnel Effects}
\deff{Fresnel Effects} - Refraction depends on viewing angles. Look at an object from top (side), the surface looks shinnier with more refraction (less shiny and more reflection).
\\
\deff{Fresnel's Equation} determines the \textbf{ratio} of reflected and transmitted light energy from a perfect surface:
\[
F = \frac{1}{2} \Big(  
\frac{\sin^2(\phi - \theta)}{\sin^2(\phi + \theta)} + 
\frac{\tan^2(\phi - \theta)}{\tan^2(\phi + \theta)}
\Big)
\]
, where $\phi$ and $\theta$ are angles of incidence and refraction, and $\mu = \sin\phi / \sin\theta$ is the refractive index of the material.
\\
\deff{Schlick's Approximation} approximates Fresnel's Equation:
\[
F = f + (1 - f) (1 - V \cdot N)^5,  \; \text{for }  f = \frac{(1 - n_1 / n_2)^2}{1 + n_1 / n_2)^2}
\]

\subsubsection{Chromatic Aberration}
\deff{Chromatic Aberration} - Realistic (flawed) cameras produce images where different colour rays do not converge on the same point. Instead, there are slightly different extents of refraction for different colour components.
\\
\underline{Solution}: Use different refractive indices for RGB channels.





\section{FBO \& Real-Time Shadow}
\deff{Multi-Pass Rendering} - render 3D scenes for multiple times (passes), and combine the multiple rendered images to synthesize the final frame.

\subsection{Framebuffer Object}
\deff{Framebuffer Object} (FBO) - Framebuffers created in OpenGL for storing rendering outputs, that can be non-displayable.
\\
Each FBO contains many \deff{Attachment Points}, i.e. rendering destinations, with two types of FBO attachable images:
\begin{enumerate}
	\item Texture images, render to texture
	\item Render buffer images, offsceen rendering
\end{enumerate}
\underline{Note}: \deff{Mutliple Render Targets} (MRT) enables a fragment shader writes to (the same position of) multiple FBOs in one pass at the same time.


\subsection{Real-Time Shadow}
Point light sources produce shadows with \textbf{hard boundaries}, while extended light sources give \textbf{soft bounaries}, with \deff{Umbra} and \deff{Penumbra}.
\\
\underline{Note}: Real-time shadow rendering techniques assume \textbf{point light sources} and generate \deff{Geometric Shadows} (shadows with correct shapes and arbitrary intensities).

\subsubsection{Shadow Volume Method}
\underline{Idea}: An occluder casts a shadow volume. After determining the occlusion boundaries for each occluder, use stencil buffer to keep trace of how many are occluding the current pixel.
\\
\deff{Stencil Buffer} has the same resolution as colour buffer, and each stencil “pixel” is an integer counter.

\smallskip

Procedure for \deff{Depth Pass Method}:
\begin{enumerate}
	\item Find silhouette edges
	\item Extend silhouette edges from light source to form shadow volume polygons
	\item Render the scene as if it were completely in shadow
	\item Disable writes to the depth and color buffers
	\item Draw all \textbf{front faces} of shadow volume, and increment stencil buffer by 1 on \textbf{depth pass}
	\item Draw all \textbf{back faces} of shadow volume, and decrement stencil buffer by 1 on \textbf{depth pass}
	\item Render the scene again as if it were completely lit, using the stencil buffer to mask the shadowed areas
\end{enumerate}
\underline{Analysis}: [1] It draws only on pixels whose stencil value is 0. [2] \textbf{Fail} if the viewpoint is in a shadow volume or some shadow volume polygons are clipped by the near plane.

\smallskip

Procedure for \deff{Depth Fail / Carmack's Reverse Method}:
\\
Same steps as Depth Pass Method except for how it draws (i.e. Step 5 and 6)
\begin{itemize}
	\item Draw all \textbf{back faces} of shadow volume, and increment stencil buffer by 1 on \textbf{depth fail}
	\item Draw all \textbf{front faces} of shadow volume, and decrement stencil buffer by 1 on \textbf{depth fail}
\end{itemize} 


\subsubsection{Shadow Mapping Method}
\underline{Idea}: Objects are in shadow when their depth map values when viewing from the light source are smaller.
\\
Procedures:
\begin{enumerate}
	\item Render the scene using the light source as viewpoint
	\item Save the depth buffer known as \deff{Shadow Map}
	\item Clear the framebuffer
	\item Render the scene from camera’s viewpoint: For each fragment, transform it to the "light space" and compare its "light space" $z$-value with shadow map $z$-value.
	If "light space" $z$-value is larger, the fragment is in shadow \& lit with only ambient light. Otherwise, the fragment is not in shadow \& is fully lit.
\end{enumerate}

For each point $\vec{p}_M$ in modelling coordinates, its shadow mapp coordinates $\vec{p}_L$ is
\[
\vec{p}_L = B \cdot P_L \cdot V_L \cdot M \cdot \vec{p}_M
\]
, where $M, V_L, P_L$ are the \textbf{Modelling Matrix}, \textbf{Light's View Transformation Matrix}, \textbf{Light's Projection Matrix}, and $B$ is defined as
\[
B = S(\frac{1}{2}, \frac{1}{2}, \frac{1}{2}) \ T(1, 1, 1) = \begin{bmatrix}
	0.5 	& 0 	& 0		& 0.5 \\
	0	 	 & 0.5 	& 0		& 0.5 \\
	0        & 0 	 & 0.5	& 0.5 \\
	0 		 & 0 	& 0		& 1
\end{bmatrix}
\]

\underline{Issue 1}: Shadow Acnes \& Self-Shadowing
\\
\underline{Solution}:  [1] Subtract a tolerance value from \codein{ShadowCoord.z}. [2] Offset the scene backwards when generating the shadow map using \codein{glPolygonOffset()}.

\smallskip

\underline{Issue 2}: Jaggies at shadow boundaries
\\
\underline{Solution}: Use \deff{Percentage Closer Filtering} to smoothen, which averages depth comparison results in a small neighbourhood in the shadow map.





\section{Post-Rendering Processing}
\deff{Post Rendering Image Processing} \textasciitilde Further processing of images produced by 3D rendering, in order to produce final frame.

\subsection{Edge Detection}
It uses 2D digital convolution with \deff{Sobel Operator}, which is defined as:
\[
S_x = \begin{bmatrix}
-1 & 0 & 1 \\
-2 & 0 & 2 \\
-1 & 0 & 1
\end{bmatrix} \ , \;
S_y = \begin{bmatrix}
-1 & -2 & -1 \\
0 & 0 & 0 \\
1 & 2 & 1
\end{bmatrix}
\]

Procedures for edge detection:
\begin{enumerate}
    \item Apply convolution to the image with Sober Operator kernels. Let results be $s_x$ and $s_y$. 
    \item Compute for each pixel $g = \sqrt{s^2_x + s^2_y}$.
    \item If a pixel has $g$ above a threshold, consider it an edge pixel. 
\end{enumerate}


\subsubsection{Window Coordinates \& Texel Coordinates}
\underline{Window Coordinates}: A fragment’s 2D position relative to the window. Consider \textbf{the center of} the fragment. For bottom-leftmost pixel, it is \codein{gl\_FragCoord.x == gl\_FragCoord.y == 0.5}.
\\
\underline{Texel Coordinates}: Indices of the texture map (as a 2D array). For bottom-leftmost toxel, it is $(0, 0)$. Used in \codein{texelFetch()}.
\\
\underline{Texture Coordinates}: Normalised coordinate values of the texture map $\in [0, 1]^2$. For the bottom-leftmost corner, it is $(0.0, 0.0)$. To calculate the coordinates of the center of a texel, average the coordinates of the corners. Used in \codein{texture()}.
\\
\underline{Note}: For all coordinate systems above, $x$ \& $y$-coordinate correspond to width from left to right, \& height from bottom to top.

\subsection{Gaussian Blurring}
2D \deff{Gaussian Kernel / Filter} is defined as:
\[
\begin{aligned}
G(x, y) &= \frac{1}{2\pi\sigma^2} e^{-\frac{x^2+y^2}{2\sigma^2}}
\\
&= G(x)G(y) \ , \; G(x) = \frac{1}{2\pi\sigma^2} e^{-x^2/2\sigma^2}
\end{aligned}
\]

It is \deff{separable} as we can express the operation as two 1D Gaussian operations.
\\
2D Digital Convolution for Gaussian Blurring is:

\[
\begin{aligned}
C_{lm} 
& \leftarrow \sum_{i=-4}^{4} \sum_{i=-4}^{4} \frac{G(i,j)}{k} C_{l+i, m+j}
\\
& \leftarrow \sum_{i=-4}^{4} \frac{G(i)}{k} \sum_{j=-4}^{4} \frac{G(i,j)}{k} C_{l+i, m+j}
\\
\text{for } k &= \sum_{i=-4}^4 G(i)
\end{aligned}
\]

\underline{Note}: All Gaussian weights must sum to $1$.

Since it can be done using two 1D convolution, we can perform it in 3 passes efficiently:
\begin{enumerate}
    \item Bind to 1st FBO \& render 3D scene normally to a texture.
    \item Bind to 2nd FBO \& apply \textbf{vertical} Gaussian blur to Pass 1 texture \& write to another texture.
    \item Bind to 3rd FBO \& apply \textbf{horizontal} Gaussian blur to Pass 2 texture \& write to output colour buffer.
\end{enumerate}













% \noindent\rule{8cm}{0.4pt}
% \section{Others}
% \subsection{From Tutorials \& Assignments}


% \subsection{From Past Year Papers}
% \citeqn{(AY20/21 Sem2 Midterm QnD3)} Differences between \codein{texture()} and \codein{texelFetch()}:
% \begin{itemize}
%     \item Data - Former takes a normalised value in $[0, 1]^2$ while latter uses exact int-valued texel coordinates.
%     \item Retrieval - Former performs texture filtering based on the sampling parameters specified, while latter retrieves the exact one texel.
% \end{itemize}

% \citeqn{(AY20/21 Sem2 Midterm QnG)} When to use Shadow Volume vs. Shadow Mapping Algorithms:
% \\
% \citeqn{(AY21/22 Sem2 Midterm Qn14)} Advantage of Shadow Volume methods over Shadow Mapping: 













% -----
\noindent\rule{8cm}{0.4pt}
\section{GLSL Coding}
\subsection{Math}
\codein{genType sqrt(genType x)} - To compute the square root of input.
\\
\codein{genType min(genType x, genType y)} - To compute the min of two inputs. Similarly for \codein{min(x,y)}.
\\
\codein{genType pow(genType x, genType y)} - To compute the power $x^y$.
\\
\codein{genType dFdx(genType p)} - To compute the partial derivative of input wrt. $x$. Similarly for \codein{dFdy()}.
\codein{float dot(genType x, genType y)} - To compute the dot produce. Similarly for \codein{cross()}.
\\
\codein{float length(genType x)} - To compute the L2 length of the vector.
\\
\codein{genType normalize(genType v)} - To normalize the input vector.
\\
\codein{genType reflect(genType I, genType N)} - To calculate reflected direction vector.



\subsection{Fragment Shader}
\codein{gvec4 texture(sampler, P)} - To sample texture from the sampler using texture coordinates.
\\
\codein{gvec4 textureProj(sampler, P)} - To first compute the query vector of $P$ using perspective division, then sample texture.
\\
\codein{gvec4 textureProjOffset(sample, P, offset)} - To compute a texture lookup with projection and offset.
\\
\codein{gvec4 texelFetch(sampler, P, lod)} - To sample texture from sampler using texel coordinates. Usually \codein{lod = 0} for Level of Details.
\\
\underline{Note}: Return types, \codein{P} \& \codein{offset} can all be vectors of size $1~4$ depending on params.

% -----
% KIV:
% 

% -----
% Others:
% 5. check whats n_1, n_2 in Schlick's Approx.


% -----
% Include img:
% opengl pipeline


\end{multicols}
\end{document}
