%%%%%%%%%%%%%%%%%%%%%%%%%%%%%%%%%%%%%%%%%%%%%%%%%%%%%%%%%%%%%%%%%%%%%%
% Original Source: Dave Richeson (divisbyzero.com), Dickinson College
% Modified By: Chen Yiyang
% 
% A one-size-fits-all LaTeX cheat sheet. Kept to two pages, so it 
% can be printed (double-sided) on one piece of paper
% 
% Feel free to distribute this example, but please keep the referral
% to divisbyzero.com
% 
% Guidance on the use of the Overleaf logos can be found here:
% https://www.overleaf.com/for/partners/logos 
%%%%%%%%%%%%%%%%%%%%%%%%%%%%%%%%%%%%%%%%%%%%%%%%%%%%%%%%%%%%%%%%%%%%%%

\documentclass[10pt,landscape,letterpaper]{article}
\usepackage{amssymb}
\usepackage{amsmath}
\usepackage{amsthm}
\usepackage{physics}  % for vectors
\usepackage{bbm}  % for mathbb-ed digits
%\usepackage{fonts}
\usepackage{multicol,multirow}
\usepackage{spverbatim}
\usepackage{graphicx}
\usepackage{ifthen}
\usepackage[landscape]{geometry}
\usepackage[colorlinks=true,urlcolor=olgreen]{hyperref}
\usepackage{booktabs}
\usepackage{fontspec}
\setmainfont[Ligatures=TeX]{TeX Gyre Pagella}
\setsansfont{Fira Sans}
\setmonofont{Inconsolata}
\usepackage{unicode-math}
\setmathfont{TeX Gyre Pagella Math}
\usepackage{microtype}

\usepackage{empheq}

% new:
\def\MT@is@uni@comp#1\iffontchar#2\else#3\fi\relax{%
  \ifx\\#2\\\else\edef\MT@char{\iffontchar#2\fi}\fi
}
\makeatother

\ifthenelse{\lengthtest { \paperwidth = 11in}}
    { \geometry{margin=0.4in} }
	{\ifthenelse{ \lengthtest{ \paperwidth = 297mm}}
		{\geometry{top=1cm,left=1cm,right=1cm,bottom=1cm} }
		{\geometry{top=1cm,left=1cm,right=1cm,bottom=1cm} }
	}
\pagestyle{empty}
\makeatletter
\renewcommand{\section}{\@startsection{section}{1}{0mm}%
                                {-1ex plus -.5ex minus -.2ex}%
                                {0.5ex plus .2ex}%x
                                {\sffamily\large}}
\renewcommand{\subsection}{\@startsection{subsection}{2}{0mm}%
                                {-1explus -.5ex minus -.2ex}%
                                {0.5ex plus .2ex}%
                                {\sffamily\normalsize\itshape}}
\renewcommand{\subsubsection}{\@startsection{subsubsection}{3}{0mm}%
                                {-1ex plus -.5ex minus -.2ex}%
                                {1ex plus .2ex}%
                                {\normalfont\small\itshape}}
\makeatother
\setcounter{secnumdepth}{0}
\setlength{\parindent}{0pt}
\setlength{\parskip}{0pt plus 0.5ex}
% -----------------------------------------------------------------------

\usepackage{academicons}

\begin{document}

\definecolor{mathBlue}{cmyk}{1,.72,0,.38}
\definecolor{defOrange}{cmyk}{0, 0.5, 1, 0.3}
\definecolor{codeInlineRed}{cmyk}{0, 0.9, 0.9, 0.45}

\everymath{\color{mathBlue}}
\everydisplay{\color{mathBlue}}

% for vector notation in this module
\newcommand{\vect}[1]{\pmb{#1}}
\newcommand{\deff}[1]{\textcolor{defOrange}{\textbf{#1}}}
\newcommand{\codein}[1]{\textcolor{codeInlineRed}{\texttt{#1}}}
\newcommand{\citeqn}[1]{\underline{\textit{#1}}}

\footnotesize
%\raggedright

\begin{center}
  {\huge\sffamily\bfseries ST2137 Cheatsheet} \huge\bfseries\\
  by Yiyang, AY22/23
\end{center}
\setlength{\premulticols}{0pt}
\setlength{\postmulticols}{0pt}
\setlength{\multicolsep}{1pt}
\setlength{\columnsep}{1.8em}
\begin{multicols}{3}


% -----------------------------------------------------------------------
\section{4. Numerical Data Analysis}
Variables can be \deff{Quantitative} (\deff{Discrete} or \deff{Continuous}) or \deff{Categorical} (\deff{Ordinal} or \deff{Nominal}).

\subsection{Single Quantitative Variable}
For a unimodal distribution, \deff{Skewness} value represents the amount and direction of skew:
\[
	\frac{\sqrt{n(n-1)}}{n-2} \times \frac{m_3}{(m_2)^{3/2}}
\]
where $m_2 = \frac{1}{n}\sum_{i=1}^n (x_i - \bar x)^2$ and $m_3 = \frac{1}{n}\sum_{i=1}^n (x_i - \bar x)^3$.
\\
A distribution is \deff{Skewed Right} / \deff{Positively Skewed} if peak is towards the left and the right tail is longer (e.g. income). A \deff{Symmetric} distribution has close to $0$ skewness.

\smallskip

\deff{Kurtosis} measures shape of a distribution, with higher (lower) values indicate a higher \& shaper (lower \& less distinct) peak:
\[
\frac{n-1}{(n-2)(n-3)} \Big[ \frac{(n+1)m_4}{m_2^2} - 3(n-1) \Big]
\]

\smallskip

Graphical summaries for 1 quantitative: [1] Histogram \& Density Plot, [2] Boxplot, [3]  QQ Plots, plots of standardised sample quantiles against theoretical quantiles of a standard normal.



\subsection{Association between Two Variables}

Summaries for 2 quantitiative: [1] Correlation Val., [2] Scatterplot.
\\
Summaries for quantitative \& categorical: [1] Boxplots by Groups, [2] Histogram by Groups.




\section{5. Robust Estimators}
A statistical method is \deff{Robust} wrt. a particular assumption if it performs adequately even when that assumption is modestly violated.


\subsection{Robust Estimation of Location}
Location Estimators: [1] Arithmetic mean, [2] Trimmed mean, [3] Winsorized mean, [4] M-Estimates.


\smallskip

The \deff{$100 \alpha \%$ Trimmed Mean} is calculated by: [1] Discard lowest $100 \alpha \%$ and highest $100 \ \alpha \%$. [2] Arithmetic mean of remaining.
\\
\underline{Note}: [1] $2 \alpha$ of extreme data discarded. [2] Usually $\alpha \in [0.1, 0.2]$.


\smallskip

The \deff{$100 \alpha \%$ Winsorized Mean} is calculated by: [1] Sort observations as $x_{(1)}, x_{(2)}, …, x_{(n)}$. [2] Replace $[n\alpha]$ smallest observations with $x_{([n\alpha]+1)}$, and $[n\alpha]$ largest with $x_{(n-[n\alpha])}$. Here, $[a]$ denotes as the nearest integer of $a$. [3] Arithmetic mean of replaced.


\smallskip

\deff{M-Estimator} with a non-constant error function $\rho$, $T$, is defined as
% TODO: \argmin not \arg\min
\[
T = \arg\min_T \sum_{i=1}^n \rho(x_i - T) 
\]


\subsection{Robust Estimation of Scale}
Scale Estimators:
\begin{itemize}
	\item \deff{Inter-Quartile Range} $\text{IQR} = Q_3 - Q_1$
	\item \deff{Median Abs Deviation} $\text{MAD} = \text{med}_i  ( |x_i - \text{med}_j(x_j)| )$
	\item \deff{Gini's Mean Difference} $G = \sum_{i < j} |x_i - x_j| / {n \choose 2}$
\end{itemize}
\underline{Note}: For a normal distribution, $\text{IQR} = 1.35 \sigma, \text{MAD} = 1.4826 \sigma, \sqrt{\pi} G / 2 = \sigma$.




\section{6. Categorical Data Analysis}
Summaries for 1 categorical: [1] Frequency Table (with category of highest frequency as \deff{Modal Category}), [2] Bar plot.


\subsection{Two Categorical}
% TODO
Association between 2 categorical: [1] Contingency Table, [2] 

\subsubsection{Contingency Table}
\underline{Note}:  Row for explanatory variables $x$ and column for response variables $Y$ (success or fail).  Measures of association: [1] Sample Diff. $= p_1 - p_2$, [2] Relative risk $= p_1 / p_2$, [3] Odds Ratio.

\smallskip

For a success prob. $\pi$, \deff{Odds of Success} is $\text{odds} = \pi / (1 - \pi)$. For 2-way contingency table, the \deff{Odds Ratio} (OR), $\theta$, and \deff{Sample OR}, $\hat \theta$, are defined
\[
\theta = \frac{\pi_1 / (1 - \pi_1)}{\pi_2 / (1 - \pi_2)}, \; \hat{\theta} = \frac{p_1 / (1 - p_1)}{p_2 / (1 - p_2)} = \frac{n_{11} \times n_{22}}{n_{12} \times n_{21}}
\]
where $n_{11}, n_{12}, n_{21}, n_{22}$ are $4$ cell counts. 

The $100\%(1-\alpha)$ Confidence Interval for OR:
\[
\exp \{ \log \hat{\theta} \pm z_{\alpha / 2} \times ASE(\log \hat{\theta})  \}
\]
where
\[
ASE(\log \hat{\theta}) = \sqrt{ 1/n_{11} + 1/n_{12} + 1/n_{21} + 1/n_{22}  }
\]
\underline{Note}: If $x$ and $Y$ independent, $\theta = 1$.


\subsubsection{Prospective \& Retrospective Studies}
\deff{Prospective Studies} sample subjects randomly from a population and randomly assign exposure variables or record exposure status. All 3 measures above are valid.
\\
\deff{Retrospective Studies} sample a group of cases and a group of controls (i.e. based on $Y$), and check each subject's exposure. As such, \textbf{cannot} obtain valid estimates of $\pi_1, \pi_2$, as we obtain $Pr(x|Y)$ but need to estiamte $Pr(Y|x)$.



\subsection{Dependence Tests}
\subsubsection{Chi-squared Test}
\underline{Assumption}: All $e_{ij} \ge 5$. \underline{Note}: \deff{Fisher Exact Test} for small samples.

\underline{Null Hypo.}: Two variables are independent.

\underline{Test Statistic}
\[
\chi^2 = \sum \frac{(|o_{ij} - e_{ij}| - 0.5)^2}{e_{ij}} \sim \chi^2_1
\]
, where $o_{ij}, e_{ij}$ are observed and expected count for each cell, and expected count is $\text{RowTotal} \times \text{ColTotal} / \text{Total}$.


\subsubsection{McNemar's Test}
\underline{Settings}: $x$ and $Y$ represent num. of students passing \& failing a test before \& after a lesson. \textbf{Dependent samples}. 

\underline{Null Hypo.}: Before and after are independent. 

\underline{Test Statistic}: let $b$ and $c$ denotes pass-then-fail \& fail-then-pass:
\[
\chi^2 = \frac{(b-c)^2}{b+c}, \; \text{or (if small sample, ) } \frac{(|b-c|-1)^2}{b+c} \sim \chi^2_1
\]


\subsubsection{Chi-Square Test for General Tables}
\underline{Assumption}: Large samples, or $\le 25\%$ cells with expected $< 5$.

\underline{Settings}: Contingency table with $r$ rows \& $c$ cols now.

Same \underline{hypotheses} and \underline{test statistic} as previous case but follows $\chi^2$ with d.f. ${(c-1) \times (r-1)}$ now.

\deff{Standardised / Adjusted Residual} for each cell:
\[
r_{ij} = \frac{o_{ij} - e_{ij}}{SE(o_{ij} - e_{ij})}, 
\; 
SE = \sqrt{ e_{ij} (1 - p_{i+}) (1 - p_{+j}) }
\]
where $p_{i+}$ and $p_{+j}$ marginal prob. of row $i$ and of col $j$. 
\underline{Note}: $|r_{ij}| > 2$ indicates lack of fit of $H_0$ in the cell.

% Test for Ordinal Data & Linear by Linear Test not yet covered



% -----
\noindent\rule{8cm}{0.4pt}
\section{R Coding}




% -----
\noindent\rule{8cm}{0.4pt}
\section{Python Coding}




% -----
\noindent\rule{8cm}{0.4pt}
\section{SAS Coding}




% -----
% KIV:
% 4. how to read qq plots
% 5. rho & T examples for M-estimator




\end{multicols}
\end{document}
