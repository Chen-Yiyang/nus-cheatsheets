%%%%%%%%%%%%%%%%%%%%%%%%%%%%%%%%%%%%%%%%%%%%%%%%%%%%%%%%%%%%%%%%%%%%%%
% Original Source: Dave Richeson (divisbyzero.com), Dickinson College
% Modified By: Chen Yiyang
% 
% A one-size-fits-all LaTeX cheat sheet. Kept to two pages, so it 
% can be printed (double-sided) on one piece of paper
% 
% Feel free to distribute this example, but please keep the referral
% to divisbyzero.com
% 
% Guidance on the use of the Overleaf logos can be found here:
% https://www.overleaf.com/for/partners/logos 
%%%%%%%%%%%%%%%%%%%%%%%%%%%%%%%%%%%%%%%%%%%%%%%%%%%%%%%%%%%%%%%%%%%%%%

\documentclass[10pt,landscape,letterpaper]{article}
\usepackage{amssymb}
\usepackage{amsmath}
\usepackage{amsthm}
\usepackage{physics}  % for vectors
\usepackage{bbm}  % for mathbb-ed digits
%\usepackage{fonts}
\usepackage{multicol,multirow}
\usepackage{spverbatim}
\usepackage{graphicx}
\usepackage{ifthen}
\usepackage[landscape]{geometry}
\usepackage[colorlinks=true,urlcolor=olgreen]{hyperref}
\usepackage{booktabs}
\usepackage{fontspec}
\setmainfont[Ligatures=TeX]{TeX Gyre Pagella}
\setsansfont{Fira Sans}
\setmonofont{Inconsolata}
\usepackage{unicode-math}
\usepackage{listings}
\usepackage{minted}
\setmathfont{TeX Gyre Pagella Math}
\usepackage{microtype}

\usepackage{empheq}

% new:
\def\MT@is@uni@comp#1\iffontchar#2\else#3\fi\relax{%
  \ifx\\#2\\\else\edef\MT@char{\iffontchar#2\fi}\fi
}
\makeatother

\ifthenelse{\lengthtest { \paperwidth = 11in}}
    { \geometry{margin=0.4in} }
	{\ifthenelse{ \lengthtest{ \paperwidth = 297mm}}
		{\geometry{top=1cm,left=1cm,right=1cm,bottom=1cm} }
		{\geometry{top=1cm,left=1cm,right=1cm,bottom=1cm} }
	}
\pagestyle{empty}
\makeatletter
\renewcommand{\section}{\@startsection{section}{1}{0mm}%
                                {-1ex plus -.5ex minus -.2ex}%
                                {0.5ex plus .2ex}%x
                                {\sffamily\large}}
\renewcommand{\subsection}{\@startsection{subsection}{2}{0mm}%
                                {-1explus -.5ex minus -.2ex}%
                                {0.5ex plus .2ex}%
                                {\sffamily\normalsize\itshape}}
\renewcommand{\subsubsection}{\@startsection{subsubsection}{3}{0mm}%
                                {-1ex plus -.5ex minus -.2ex}%
                                {1ex plus .2ex}%
                                {\normalfont\small\itshape}}
\makeatother
\setcounter{secnumdepth}{0}
\setlength{\parindent}{0pt}
\setlength{\parskip}{0pt plus 0.5ex}
% -----------------------------------------------------------------------

\usepackage{academicons}

\begin{document}

\definecolor{mathBlue}{cmyk}{1,.72,0,.38}
\definecolor{defOrange}{cmyk}{0, 0.5, 1, 0.3}
\definecolor{codeInlineRed}{cmyk}{0, 0.9, 0.9, 0.45}

\everymath{\color{mathBlue}}
\everydisplay{\color{mathBlue}}

% for vector notation in this module
\newcommand{\vect}[1]{\pmb{#1}}
\newcommand{\deff}[1]{\textcolor{defOrange}{\textbf{#1}}}
\newcommand{\codein}[1]{\textcolor{codeInlineRed}{\texttt{#1}}}
\newcommand{\citeqn}[1]{\underline{\textit{#1}}}

\footnotesize
%\raggedright

\begin{center}
  {\huge\sffamily\bfseries ST2137 Cheatsheet} \huge\bfseries\\
  by Yiyang, AY22/23
\end{center}
\setlength{\premulticols}{0pt}
\setlength{\postmulticols}{0pt}
\setlength{\multicolsep}{1pt}
\setlength{\columnsep}{1.8em}
\begin{multicols}{3}



\lstset{language=R,
    basicstyle=\small\linespread{0.5},
    otherkeywords={0,1,2,3,4,5,6,7,8,9},
    morekeywords={TRUE,FALSE},
    deletekeywords={data,frame,length,as,character},
    keywordstyle=\color{blue},
    commentstyle=\color{yellow},
}

% -----------------------------------------------------------------------
\subsection{4. Numerical Data Analysis}
% Variables can be \deff{Quantitative} (\deff{Discrete} or \deff{Continuous}) or \deff{Categorical} (\deff{Ordinal} or \deff{Nominal}).

% \subsection{Single Quantitative Variable}
For unimodal distri, \deff{Skewed Right} / \deff{Positively Skewed} if peak is towards the left \& the right tail is longer (e.g. income):
$
\frac{\sqrt{n(n-1)}}{n-2} \times \frac{m_3}{(m_2)^{3/2}}
$
where $m_2 = \frac{1}{n}\sum_{i=1}^n (x_i - \bar x)^2$ and $m_3 = \frac{1}{n}\sum_{i=1}^n (x_i - \bar x)^3$.
\\
Higher (lower) \deff{Kurtosis} values indicate a sharper (less distinct) peak:
$
\frac{n-1}{(n-2)(n-3)} \Big[ \frac{(n+1)m_4}{m_2^2} - 3(n-1) \Big]
$.

Graphical summaries for 1 quantitative: [1] Histogram \& Density Plot, [2] Boxplot, [3]  QQ Plots, plots of standardised sample quantiles against theoretical quantiles of a standard normal.



% \subsection{Association between Two Variables}

Summaries for 2 quantitative: [1] Correlation Val., [2] Scatterplot.
\\
Summaries for quantitative \& categorical: [1] Boxplots by Groups, [2] Histogram by Groups. 

\subsection{5. Robust Estimators}
% A statistical method is \deff{Robust} wrt. a particular assumption if it performs adequately even when that assumption is modestly violated.

% \subsection{Robust Estimation of Location}
\textbf{Location Estimators}: [1] Arithmetic mean, [2] Trimmed mean, [3] Winsorized mean, [4] M-Estimates.
\\
\deff{$100 \alpha \%$ Trimmed Mean} is calculated by: [1] Discard lowest $100 \alpha \%$ and highest $100 \ \alpha \%$. [2] Arithmetic mean of remaining.
\underline{Note}: [1] $2 \alpha$ of extreme data discarded. [2] Usually $\alpha \in [0.1, 0.2]$.
\\
\deff{$100 \alpha \%$ Winsorized Mean} is calculated by: [1] Sort observations as $x_{(1)}, x_{(2)}, …, x_{(n)}$. [2] Replace $[n\alpha]$ smallest observations with $x_{([n\alpha]+1)}$, and $[n\alpha]$ largest with $x_{(n-[n\alpha])}$. Here, $[a]$ denotes as the nearest integer of $a$. [3] Arithmetic mean of replaced.
\\
\deff{M-Estimator} w. non-const err.func $\rho$:
% TODO: \argmin not \arg\min
$
T = \arg\min_T \sum_{i=1}^n \rho(x_i - T) 
$.


% \subsection{Robust Estimation of Scale}
\textbf{Scale Estimators}: [1] \deff{IQR} $\text{IQR} = Q_3 - Q_1$ [2] \deff{Median Abs Devia-n} $\text{MAD} = \text{med}_i  ( |x_i - \text{med}_j(x_j)| )$ [3] \deff{Gini's Mean Diff} $G = \sum_{i < j} |x_i - x_j| / C_2^n$
.For normal, $\text{IQR} = 1.35 \sigma, \sigma = \text{MAD} * 1.4826, \sqrt{\pi} G / 2 = \sigma$.




\subsection{6. Categorical Data Analysis}
Summaries for 1 categorical: [1] Frequency Table (with category of highest frequency as \deff{Modal Category}), [2] Bar plot.
\\
\deff{Contingency Table} - Row for explanatory var $x$ \& column for response $Y$ (success or fail).  Measures of association: [1] Sample Diff. $= p_1 - p_2$, [2] Relative risk $= p_1 / p_2$, [3] Odds Ratio.
\\
For a success prob. $\pi$, \deff{Odds of Success} $\text{odds} = \pi / (1 - \pi)$. For 2-way contingency table, \deff{Odds Ratio} (OR), $\theta$, \& \deff{Sample OR}, $\hat \theta$, are:
$
\theta = \frac{\pi_1 / (1 - \pi_1)}{\pi_2 / (1 - \pi_2)}, \; \hat{\theta} = \frac{p_1 / (1 - p_1)}{p_2 / (1 - p_2)} = \frac{n_{11} \times n_{22}}{n_{12} \times n_{21}}
$ for $n_{ij}$ cell counts. 

The $100\%(1-\alpha)$ Confidence Interval for OR:
$
\exp \{ \log \hat{\theta} \pm z_{\alpha / 2} \times ASE(\log \hat{\theta})  \}
$
and
$
ASE(\log \hat{\theta}) = \sqrt{ 1/n_{11} + 1/n_{12} + 1/n_{21} + 1/n_{22}  }
$
\underline{Note}: If $x$ and $Y$ independent, $\theta = 1$.

% \subsubsection{Prospective \& Retrospective Studies}
\deff{Prospective Studies} sample subjects randomly from a population and randomly assign exposure variables or record exposure status. All 3 measures above are valid.
\\
\deff{Retrospective Studies} sample a group of cases and a group of controls (i.e. based on $Y$), and check each subject's exposure. As such, \textbf{cannot} obtain valid estimates of $\pi_1, \pi_2$, as we obtain $Pr(x|Y)$ but need to estiamte $Pr(Y|x)$. Can use odds ratio for test 
only



% \subsection{Dependence Tests}
\textbf{Dependence Test - Chi-squared Test}
\\
\underline{Assumption}: All $e_{ij} \ge 5$. (\deff{Fisher Exact Test} if small size). 
\underline{Null}: Two var.s independent. 
\underline{Statistic}:
$
\chi^2 = \sum \frac{(|o_{ij} - e_{ij}| - 0.5)^2}{e_{ij}} \sim \chi^2_1
$
for $o_{ij}, e_{ij}$ observed \& expected count. ExpCnt $= \text{RowTotal} \times \text{ColTotal} / \text{Total}$.


\textbf{Dependence Test - McNemar's Test}
\\
\underline{Settings}: $x$ and $Y$ represent num. of students passing \& failing a test before \& after a lesson. \textbf{Dependent samples}. 
\underline{Null}: Before \& after independent. 
\underline{Statistic}: let $b,c$ denotes pass-then-fail \& fail-then-pass:
$
\chi^2 = \frac{(b-c)^2}{b+c}  \sim \chi^2_1$, or if small sample, $\frac{(|b-c|-1)^2}{b+c} \sim \chi^2_1$


\textbf{Dependence Test -  Chi-Square for General Tables}
\\
\underline{Assumption}: Large samples, or $\le 25\%$ cells with expected $< 5$.
\underline{Settings}: Contingency table with $r$ rows \& $c$ cols now. \underline{Null \& Statistic} Same but follows $\chi^2$ with d.f. ${(c-1) \times (r-1)}$ now.

\deff{Standardised / Adjusted Residual} for each cell:
$r_{ij} = \frac{o_{ij} - e_{ij}}{SE(o_{ij} - e_{ij})}, 
\; 
SE = \sqrt{ e_{ij} (1 - p_{i+}) (1 - p_{+j}) }
$ for $p_{i+}$ and $p_{+j}$ marginal prob. of row $i$ and of col $j$. 
\underline{Note}: $|r_{ij}| > 2$ cell's lack of fit of $H_0$.


\textbf{Dependence Test - Linear-by-Linear Ordinal Data}
\underline{Null}: Two var independent. \underline{Statistic} $M^2 \sim \chi^2_1$ approx. for large n.





\subsection{7. Hypothesis Testing}
% \subsection{One-Sampled}
\textbf{One-Sampled t}
\underline{Null}: $\mu = \mu_0$. 
\underline{Statistic}, $t = {\bar{X} - \mu_0} / se(\bar{X}) \sim t_{n-1}$.


\textbf{One-Sampled Wilcoxon Signed Rank}
\underline{Null}: $Med = m_0$.
\underline{Statistic}: let $V^{+} = \sum_{i=1}^n(I(x_i > m_0)$ \& $<$ for $V^{-}$. Then test stat $V = \min(V^+, V^-) \sim Bin(V^{+} + V^{-}, 0.5)$.


\textbf{Two-Sample Dependent} 
Take pair difference \& use one-sampled.


% \subsection{Two-sampled Independent}
\textbf{Two-Sampled t}
\underline{Null}: $\mu_x = \mu_y$.
\underline{Statistic}:  $t = {\bar{X} - \bar{Y}} / se \sim t_{n_1+n_2-2}$. $se = s_p \sqrt{1/n_1 + 1/n_2}$ where $s^2_p = \frac{(n_1-1)s^2_X + (n_2-1)s^2_Y}{n_1+n_2-2}$.


\textbf{Two-Sampled Indep - Mann-Whitney U / Wilcoxon Rank Sum}
\underline{Idea}: Used to check if two grps of data too different, by comparing their rank sum with those uniformly distributed in a pooled grp.





\subsection{8. Analysis of Variance}
\underline{Definition}: For $Y_{ij}$, $j$-th observation of $i$-th grp, \deff{One-Way ANOVA},
$ %$\[
Y_{ij} = \mu + \alpha_i + e_{ij}, \ i = 1, ..., I, j = 1, ..., J,$ subject to $\sum_{i=1}^I \alpha_i = 0$
\\
$SS_W = \sum_{i=1}^I\sum_{j=1}^J (Y_{ij} - \bar{Y}_i)^2$ in-grp varia-n. $SS_B = J \sum_{i=1}^I (\bar{Y}_i - \bar{\bar{Y}})^2$ btw-grp varia-n. $SS_{TOT} = \sum_{i=1}^I\sum_{j=1}^J (Y_{ij} - \bar{\bar{Y}})^2 = SS_W + SS_B$.
\\

\textbf{Tests} - \underline{Null}: $a_i$ all same. \underline{Statistic}: $F = \frac{SS_B / (I-1)}{SS_W / [I(J-1)]} \sim F$.
\\
If grp size $J_1, ..., J_I$ different, total size $n$, $E(SS_W) = \sigma^2 \sum_{i=1}^I (J_i - 1), E(SS_B) = (I-1)\sigma^2 + \sum_{i=1}^I J+i \alpha_i^2$, and $F$ has df $I-1, n-I$.
\\

\underline{Assumptions \& Checks}: [1] Random samples [2] Equal var: (1a) \deff{Bartlett Test} sample assumed normal, (1b) \deff{Levene Test} sample distri unknown. [34] Errors iid.: (2a) \deff{Shapiro Wilk Test} on residual, (2b) \deff{KS Test}, (2c) plot. [5] Additivity of treatment effects.


% \subsubsection{Others}
\deff{Kruskal-Wallis Test}: Non-parametric version of ANOVA: no normal assumption, good for small sample size.
\\
\deff{Multiple Comparisons}: [1] \deff{Bonferroni}: control $k$ hypotheses' $\alpha$ at $a$, $a/k$ for each, no need normal. [2] \deff{Tukey}: For pairs, in ANOVA. [3] \deff{Least Signif Diff}: null grp means same, in ANOVA.





\subsection{9. Regression Analysis}
\underline{Assumptions}: [1] Linear relationship. [2] Normality \& equal \& const var. [3] Regressors uncorrelated.
\\
$R^2 = \frac{SS_R}{SS_T}, R^2_a = 1 - \frac{SS_{res} / (n-p)}{SST_(n-1)}$. $SS_R$: $(\hat{y}_i - \bar{Y})^2$, $SS_{res}$: $(y_i - \hat{y}_i)^2$.
\\
\textbf{Overall}
\underline{Null}: $\beta = 0$. \underline{Statistic}: $F_0 = \frac{SSR/p}{SS_{Res} / (n-p-1)} \sim F$ rej large $F_0$.
\\
\textbf{Individual}
\underline{Null}: $\beta_i = 0$. \underline{Statistic}: $t_i = \frac{\hat{\beta}_i}{se(\hat{\beta}_i)} \sim t_{n-p-1}$.


\underline{Model Check}: [1] \deff{Outlier} if $|sr|$ large. [2] \deff{Influential Point} if Cook' Distance $D_ii = \frac{r_i^2 h_{ii}}{p(1-h_{ii})} > 1$. [3] \deff{Leverage} $h_ii$ of $H = X(X'X)^{-1}X'$ 


\subsection{10. Simulation}
\deff{Congruent Generator}: 1) Choose $a, c, m \in Z$, \& seed $X_0$. 2) Define $X_{n+1} = (aX_n + c) \text{ mod } m$. \underline{Note}: If we need uniform random values, $U_i = X_i / m \in [0, 1)$.

\underline{Theory of Inversion}: [1] For $X$ with CDF $F$, $Y = F(X)\sim U(0,1)$. [2] For $Y\sim U(0,1)$ and $X$ with CDF $F$, $F^{-1}(Y) = F$.

\deff{Inversion Method} for generating from distribution $F$: 1) Generate $U \sim U(0, 1)$. 2) Set $X = F^{-1}(U)$ assuming inverse exists. 3) Output $X$, following $F$.

% TODO: Monte Carlo?






% -----
\noindent\rule{8cm}{0.4pt}
\subsection{R Coding}
\begin{minted}{R}
# Vector
numeric(n); character(n)  # vector with n 0's / ""'s
rep(a, b)  # replicate item a by b times
seq(from=a,to=b,by=c); seq(from=a,to=b,length=d);
# Matrix
matrix(v, nrows=a, ncols=b, byrow=T); rbind(...); cbind(...)
# Dataframes
df <- data.frame(m); names(df) = c(...); row.names(df)= c(...)
df[a,b:c]; df$abc; df[order(val),]; merge(df1, df2, by="id")
df[rev(order(val)),] # asc, desc
\end{minted}

\begin{minted}{R}
if (condition) {...} else {...} # Conditioning
while (condition) {...} # While loop
for (<variable> in <range>) {...} # For loop
read.csv(..., header=T, width=c(...)), read.table(...) # IO
# Note: Use width if each variable spans multiple lines
write.table(data, "C:/...")
cat(...); sink() # print
# Random
set.seed(999); x = rnorm(n,0,1); random std norm size n
\end{minted}

\subsubsection{4. Numerical Data Analysis}
\begin{minted}{R}
# descriptive stats, location
length(x); summary(x); mean(x); median(x); quantile(x)
# descriptive stats, variability
range(x); var(x); sd(x); IQR(x);  x[order(x)[1:5]] # smallest 5
# skewness
skew <- function(x){
  n<-length(x); m3<-mean((x-mean(x))^3); m2<-mean((x-mean(x))^2); 
  sk=m3/m2^(3/2)*sqrt(n*(n-1))/(n-2); return(sk) } 
# kurtosis
kurt = function(x) {
  n=length(X); m4=mean((x-mean(x))^4); m2=mean((x-mean(x))^2)
  kurt=(n-1)/((n-2)*(n-3))*((n+1)*m4/(m2^2)-3*(n-1))}
\end{minted}

\begin{minted}{R}
# Histogram w. Density Plot
hist(mark, freq=FALSE, main="Hist", xlab="mark", ylab="val", 
    axes=RUE, col="grey", nclass=10) x<-seq(0,30,length.out=98); 
y<-dnorm(x,mean(mark),sd(mark)); lines(x, y, col = "red")
boxplot(mark, xlab = "mark") # Boxplots
qqnorm(mark, pch = 20); qqline(mark, col = "red") # QQ plots
# For association between two
cor(v1, v2); plot(v1, v2, pch=20) # Correlation val; Scatterplot
boxplot(energy~type) # Boxplots by Group
# Others
par(mfrow=c(2,2)); ...; # Subplots
par(new=TRUE); ...; # add new plots to same graph
\end{minted}

\subsubsection{5. Robust Estimators}
\begin{minted}{R}
mean(x); mean(x, trim=0.2) # arithmetic & 20% trimmed
winsor <- function(x, alpha=0.2){
  n=length(x); xq=n*alpha; x=sort(x); m=x[(round(xq)+1)]; 
  M=x[(n-round(xq))]; x[which(x<m)]=m; x[which(x>M)]=M
  return(c(mean(x),var(x)))}; winsor(x)
library(MASS); hubers(x, k=0.84) # Or use library
median(abs(x-median(x))); mad(x); IQR(x)
\end{minted}

\subsubsection{6. Categorical Data Analysis}
\begin{minted}{R}
count=table(data$type); barplot(count) # freq table, barplot
# Contingency table
ct <- matrix(c(...), ncol=2, byrow=2)
dimnames(ct)<-list(rowname=c(...),colname=c(...))
test<-prop.test(ct,correct=FALSE)
RR<-(test$estimate[1])/(test$estimate[2])
odds<-test$estimate/(1- test$estimate); OR<-odds[1]/odds[2]
# Fisher Exact Test
fisher.test(ct, alternative="two.sided")
# general Chi-squared Test
chisq.test(ct)
# McNemar Test
mcnemar.test(x, correct=TRUE)
# Linear-by-linear
set=as.table(read.ftable(...)); library(coin)
lbl_test(set,scores=list(MI=c(0,1),Alcohol=c(0,0.5,1.5,4,7)))
\end{minted}

\begin{minted}{R}
# fre table create new column, x for gender:
ggrp=factor(gender); levels(ggrp)=c("F", "M")
ggrp; table(ggrp) # below another method for drive grp
dgrp<-ifelse(drivelic=="Y","Yes","No"); table(dgrp)
tab = table(ggrp,dgrp) # cont table

\end{minted}

\subsubsection{7. Hypothesis Testing}
\begin{minted}{R}
# One-sampled t-Test
t.test(weight, mu=3.3,alternative="less")
# One-sampled Sign Test
weight.non.0=(weight[weight!=3.3]); w.len=length(weight.non.0)
binom.test(sum(weight<3.3), w.len, alternative="less") 
# Wilcoxon Signed Rank Test
wilcox.test(weight.non.0, mu=3.3, alternative="less")
# Equal var test: null is equal; null assume normal
var.test(x,y); bartlett.test(weight_gain~level, data=data)
# Two-sampled t-Test
t.test(x,y, mu=0, var.equal=TRUE)
# Mann Whitney U Test
wilcox.test(bf,no.bf)
\end{minted}

\subsubsection{8. ANOVA}
\begin{minted}{R}
anova<-aov(amount~lab, data=data); summary(anova)
tapply(amount, lab, mean) # get group mean
# Kruskal Wallis
kruskal.test(amount~lab)
# Bonferroni
pairwise.t.test(amount, lab, p.adj = "bonf")
# Tukey
TukeyHSD(anova) # default family alpha 0.05 
# LSD, I(J-1)=63, alpha-0.05
MSW=sum(anova$res^2)/63; lsd<-qt(0.975,63)*sqrt(MSW*2/7)
# Model assumption checks
shapiro.test(anova$res) # Shapiro for residual normality
ks.test(resid,"pnorm",mean(resid),sd(resid)) # KS normality
bartlett.test(amount~lab, data=newdata) # equal var
\end{minted}

\subsubsection{9. Regression Analysis}
\begin{minted}{R}
m1<-lm(weight~height+age, data=data); summary(m1); anova(m1)
plot(weight,height, type = "n") # plot by gender, M then F
points(weight[gender=="M"], height[gender=="M"],col="red")
m1$res; rs=rstandard(m1); m1$fitted.values # r, sr, fitted
summary(m1)$r.squared; summary(m1)$sigma # r2, sigma hat
# QQ Plot of SR
qqnorm(rs,datax=TRUE,ylab="SR", xlab="Z scores",)
qqline(rs,datax=TRUE,col="red") # datax: theory-qnt Y, obs X
# SR against fitted
plot(m1$fitted.values,rs, xlab="fitted"); abline(h=0)
# Predicted
predict(m1, newdata=data.frame(height=c(65,63), age=c(40,36)),
    interval="confidence",level=0.95)
# Model Check
x=cbind(c(rep(1,n)),height); hat=x%*%solve(t(x)%*%x)%*%t(x)
lvg=diag(hat); lvg[which(lvg>2*p/n)] # Leverage & check 
cooks.distance(m1) # Cook's distance
\end{minted}


% -----
\noindent\rule{8cm}{0.4pt}
\subsection{Python Coding}
\begin{minted}{Python}
import pandas as pd
import scipy.stats as scst
import matplotlib.pyplot as plt
import statistics as st
\end{minted}
\begin{minted}{Python}
# matrix
mat=np.asmatrix([[...],...]); mat.T; mat.I
np.vstack((...)); np.column_stack((...))
# Dataframe
dat={'X':[...],'Y':[...]};pd.DataFrame(dat,columns=['X','Y'])
df1=df.rename({'X':'NewX','Y':'NewY'}, axis=1)
\end{minted}

\subsubsection{4. Numerical Data Analysis}
\begin{minted}{Python}
# Descriptive stats
df['x'].median(); df['x'].var(); df['x'].std()
df['x'].quantile(0.25); df['x'].quantile(0.75)
# Histogram w. Density Plot
l=list(np.arange(0,30,0.5))
y=scst.norm.pdf(l,loc=mean(x),scale=st.stdev(x)) # qnorm
plt.plot(l, y); plt.hist(data['x1'], density=True)
plt.title('...'); plt.xlabel('...'); plt.ylabel('...')
# Boxplot
plt.boxplot(data['x1'])
# QQ Plot
scst.probplot(x, dist="norm", plot=pylab); pylab.show()
# Scatterplot
plt.scatter(v1, v2)
# Scatterplot by Group (tut3Qn2)
groups=data.groupby("x11")
for name, grp in groups:
    plt.plot(grp["x"], grp["y"], label=name)
# Boxplots by Group
fig, ax = plt.subplots(figsize=(7,5))
bats.boxplot(column=['energy'], by='type',ax=ax,color='b')
# Others
plt.legend(); plt.show()
# correlation:
np.corrcoef(x, y)[0, 1]
\end{minted}

\subsubsection{6. Categorical Data Analysis}
\begin{minted}{Python}
import statsmodels.api as sm
from statsmodels.stats.contingency_tables import mcnemar
# Table & Barplots
tab=pd.crosstab(index=data["type"],columns=data["count"])
plt.bar(type,counts)
# Cont table, using df or Numpy 2D array
scst.chi2_contingency(ctable, correction = True)
# Fisher Exact Test
scst.fisher_exact(ctable, alternative='two-sided')
# McNemar Test
mcnemar(ctable, exact=False, correction=True)
# General Chi-squared
scst.chi2_contingency(obs, correction=True)
# Linear-by-Linear association test
ct=sm.stats.Table(np.asarray(table)); rsc=np.asarray([0,1])
csc=np.asarray([0,0.5,1.5,4,7]) # scores for 2 rows 5 columns
ct.test_ordinal_association(row_scores=rsc, col_scores=csc))
\end{minted}

\subsubsection{7. Hypothesis Testing}
\begin{minted}{Python}
# One-sampled t-Test
t, p = scst.ttest_1samp(weight, popmean=3.3)
# Wilcoxon Signed Rank test:
scst.wilcoxon(weight-3.3, y=None, zero_method='wilcox', 
    correction=True, alternative='less')
# Equal var test
t, p = scst.bartlett(x,y)
# Two-sampled t-Test
scst.ttest_ind(x, y, axis=0, equal_var=True)
# Two-sampled ManWhitney U test / Wilcoxon Rank Sum Test: 
scst.mannwhitneyu(x,y,use_continuity=True,alternative...)
# Two-sampled Paired t Test
scst.ttest_rel(after,before) #, nan_policy='propagate')
\end{minted}

\subsubsection{8. ANOVA}
\begin{minted}{Python}
import statsmodels.stats.multicomp as mc
# ANOVA
m1=ols('amount~lab', data=newdata).fit()
anova=sm.stats.anova_lm(mod, type=2)
# Another method
anova2=scst.f_oneway(lab1, ..., lab7);print(anova2)
# Kruskal Wallis
krus=scst.kruskal(lab1, ..., lab7); print(krus)
# Bonferroni
comp=mc.MultiComparison(newdata['amount'],newdata['lab'])
res,tb1,tb2=comp.allpairtest(stats.ttest_ind,method="bonf")
print(res)
# Tukey
tukey=comp.tukeyhsd(); print(tukey.summary())
# Model check
scst.shapiro(mod.resid) # normality check
test=np.random.normal(mean(amount),np.std(amount),70)
scst.ks_2samp(amount,test) # KS test for amount, same resid
scst.bartlett(lab1, ..., lab7) # equal var assume norm
scst.levene(lab1, ..., lab7) # equal var
\end{minted}

\subsubsection{9. Regression Analysis}
\begin{minted}{Python}
from statsmodels.formula.api import ols
scst.pearsonr(data['W'], data['H']); df.corr()
m1=ols("W~H+age",data=data).fit(); print(m1.summary())
anova1 = sm.stats.anova_lm(model, typ=1); print(anova1)
m1.bse,m1.mse_resid,np.sqrt(m1.mse_resid) # stderr MSR RSE
fitted = m1.fittedvalues; 
# Model Check
model.resid # std residual
ana = model.get_influence() 
SR = analysis.resid_studentized_internal
leverage = analysis.hat_matrix_diag
cooks_d, p = analysis.cooks_distance
\end{minted}


\subsubsection{Others}
\begin{minted}{Python}
from scipy.stats import norm
x = norm.ppf(0.975)
# Random
np.random.seed(999)
np.random.uniform(0,1,6) # 6 of U(0,1), norm in QQ plot
np.random.exponential(1/5, n); np.random.weibull(4, 10)
...binomial(n=100, p=0.3, size=10) ...poisson(lam=3, size=10)
\end{minted}





% -----
\noindent\rule{8cm}{0.4pt}
\subsection{SAS Coding (\# FOR LINEBREAK)}
\begin{minted}{SAS}
data ex_1;# input subject gender $ CA1 CA2 HW $;
datalines;  # 10 m 80 84 a  # 7 m 85 89 a #;
PROC means data=ex_1  mean var Q1 Median Q3 min max;
  var CA1 CA2; # run;
/* Read from CSV */
FILENAME REFFILE '...'; # PROC IMPORT DATAFILE=REFFILE
  # DBMS=CSV  # OUT=WORK.heat;  # GETNAMES=YES;# RUN;
PROC CONTENTS DATA=WORK.heat; RUN;
/* Read from txt */ PROC IMPORT DATAFILE=REFFILE
  # DBMS=DLM  # OUT=WORK.example1;  # DELIMITER=",";
  GETNAMES=NO;  # DATAROW=1;# RUN;
/* Export data */ PROC EXPORT data=ex_1
  outfile=_dataout  # dbms=csv replace;# run;
/* CHANGING VARIABLE NAMES */ DATA ex_1;
    set ex_1(rename=(var1=id var2=gender ...));# run;
/* To create the labels */ proc format;
    value $gen 'F'='Female' 'M'='Male';# run;
\end{minted}

\subsubsection{6. Categorical Data Analysis}
\begin{minted}{SAS}
/* McNemar Tes, agree means no correction*/
proc freq data=debate;# by gender;
tables before*after/agree; # weight count;
title "Chi-square test for the paired samples";
run;
/* Test for normality */
proc univariate data=datamark normal ;
var mark;# histogram mark /normal;
qqplot /normal (mu=est sigma=est);# run;
\end{minted}

\subsubsection{7. Hypothesis Testing}
\begin{minted}{SAS}
/* One-sampled t-Test, two versions */
/* It includes sign test and signed rank test */
PROC UNIVARIATE data=baby mu0=3.3; 
var weight; run;
PROC TTEST data = baby H0=3.3; *sides = L or U;
var weight; run;
/* Two-sampled Mann-Whitney U Test */
PROC NPAR1WAY data=weightgain wilcoxon;
  class level;  # var weight_gain;  # *exact wilcoxon;
run;
/* Paired t-test*/
PROC TTEST DATA=platelet;
    PAIRED after*before;# RUN;
\end{minted}

\begin{minted}{SAS}
/* Descriptive stats by group/level */
proc means data=weightgain n nmiss mean std
    stderr median min max qrange maxdec=4;
class level; var weight_gain;# run;
/* Test for normality & produce CI on median */
proc univariate data=weightgain normal cipctldf;
class level;
var weight_gain;
histogram weight_gain /normal;
qqplot /normal (mu=est sigma=est);# run;
/* Produce boxplots */
proc sgplot data=weightgain;
title 'Boxplot of weight gain by level of protein';
vbox weight_gain /category=level;# run;
\end{minted}

\subsubsection{8. ANOVA}
\begin{minted}{SAS}
PROC ANOVA data=newdata;# class lab;
model amount=lab;# means; # run;
/* Kruskal Wallis Test */
PROC NPAR1WAY data=newdata wilcoxon dscf;
class lab;# var amount;# run;
/* Bonferroni, Tukey */
PROC ANOVA data=newdata;# class lab;
model amount=lab;
means lab / Bon cldiff alpha=0.05;# run;
means lab / tukey cldiff alpha=0.05; # run;
/* Model check, add after model amt=lab line */
means lab / hovtest=levene alpha-0.05;
means lab / hovtest=BARTLETT alpha-0.05;
/* normality plot */
PROC UNIVARIATE data=newdata normal;
var amount;# histogram amount /normal;
qqplot /normal (mu=est sigma=est);# run;
\end{minted}

\subsubsection{9. Regression Analysis}
\begin{minted}{SAS}
/* Create dummy */
data example1;# set example1;
if gender="M" then gen=1;# if gender="F" then gen=0;
run;
/* Correlation values */
proc corr data=example1 nosimple; 
title "Example of a correlation matrix";
var height weight age;# run;
/* Scatterplot of height vs weight by gender */
proc sgscatter  data = example1;
   plot height * weight
   datalabel = gender group = gender;# run;
/* Multiple model, SS1 is ANOVA SSR*/
proc reg data=example1;
  model weight = height age/SS1;# run;# quit;
/* Model with interaction term, create first */
data example1;# set example1; # hg=height*gen;# run;
proc reg data=example1;
  model weight = height age gen hg;# run;# quit;
/* Normality test for SR */
proc univariate data=analysis normal;
var resid;# histogram resid /normal;
qqplot /normal (mu=est sigma=est);# run;
/* Make prediction */
/* alpha default 0.05; lclm, uclm: lower, upper boundfor CI */
/* for CI; lcl, ulc: PI */
data example1;# set example1 end=last;# output;
if last then do;
  gender = . ;  # height = 64;  # weight = .;
  age =. ;  # output;# end;# run;
proc reg data=example1 alpha = 0.01;
  model weight = height;
output out=predict(where=(weight=.)) p=predicted 
  uclm=UCL_Pred lclm=LCL_Pred;# run;# quit;
/* Model check */
proc reg data=crab;
  model weight = width s1 s2;
output out=check P=yhat STUDENT=SR;# run;# quit;
proc univariate data=check normal;# var SR;
histogram SR /normal;# qqplot /normal (mu=est sigma=est);
run;
proc sgscatter data = check;# plot SR*yhat;# run;
proc sgplot data = check;# SCATTER x=yhat y=SR;
   refline 0 / axis=y lineattrs=(thickness=2 color=darkred);
run;
\end{minted}

\subsubsection{10. Simulation}
\begin{minted}{SAS}
/* Generate random uniform */
data Ugen;# call streaminit(999); /* seed 999 */
do i = 1 to 10;
 x = rand('uniform', 2, 3);  # output;
end;# keep x;# run;
proc print data=Ugen;# var x; # run;
/* Generate other special distributions*/
rand('exponential', 1/5); *rate lambda = 5;
rand('weibull',4); *shape alpha = 4;
rand('normal',mu,sigma); rand('chisq',df);
rand('binom',p,n); rand('poisson',lambda);
\end{minted}
% -----
% KIV:
% 4. how to read qq plots
% 5. rho & T examples for M-estimator




\end{multicols}
\end{document}
