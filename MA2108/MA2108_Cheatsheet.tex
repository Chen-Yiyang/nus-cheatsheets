%%%%%%%%%%%%%%%%%%%%%%%%%%%%%%%%%%%%%%%%%%%%%%%%%%%%%%%%%%%%%%%%%%%%%%
% Original Source: Dave Richeson (divisbyzero.com), Dickinson College
% Modified By: Chen Yiyang
% 
% A one-size-fits-all LaTeX cheat sheet. Kept to two pages, so it 
% can be printed (double-sided) on one piece of paper
% 
% Feel free to distribute this example, but please keep the referral
% to divisbyzero.com
% 
% Guidance on the use of the Overleaf logos can be found here:
% https://www.overleaf.com/for/partners/logos 
%%%%%%%%%%%%%%%%%%%%%%%%%%%%%%%%%%%%%%%%%%%%%%%%%%%%%%%%%%%%%%%%%%%%%%

\documentclass[10pt,landscape,letterpaper]{article}
\usepackage{amssymb}
\usepackage{amsmath}
\usepackage{amsthm}
\usepackage{physics}  % for vectors
\usepackage{bbm}  % for mathbb-ed digits
%\usepackage{fonts}
\usepackage{multicol,multirow}
\usepackage{spverbatim}
\usepackage{graphicx}
\usepackage{ifthen}
\usepackage[landscape]{geometry}
\usepackage[colorlinks=true,urlcolor=olgreen]{hyperref}
\usepackage{booktabs}
\usepackage{fontspec}
\setmainfont[Ligatures=TeX]{TeX Gyre Pagella}
\setsansfont{Fira Sans}
\setmonofont{Inconsolata}
\usepackage{unicode-math}
\usepackage{listings}
\usepackage{minted}
\setmathfont{TeX Gyre Pagella Math}
\usepackage{microtype}
\usepackage{ulem}  % cuz \underline affected by everymath
\usepackage{empheq}

% new:
\def\MT@is@uni@comp#1\iffontchar#2\else#3\fi\relax{%
  \ifx\\#2\\\else\edef\MT@char{\iffontchar#2\fi}\fi
}
\makeatother

\ifthenelse{\lengthtest { \paperwidth = 11in}}
    { \geometry{margin=0.4in} }
	{\ifthenelse{ \lengthtest{ \paperwidth = 297mm}}
		{\geometry{top=1cm,left=1cm,right=1cm,bottom=1cm} }
		{\geometry{top=1cm,left=1cm,right=1cm,bottom=1cm} }
	}
\pagestyle{empty}
\makeatletter
\renewcommand{\section}{\@startsection{section}{1}{0mm}%
                                {-1ex plus -.5ex minus -.2ex}%
                                {0.5ex plus .2ex}%x
                                {\sffamily\large}}
\renewcommand{\subsection}{\@startsection{subsection}{2}{0mm}%
                                {-1explus -.5ex minus -.2ex}%
                                {0.5ex plus .2ex}%
                                {\sffamily\normalsize\itshape}}
\renewcommand{\subsubsection}{\@startsection{subsubsection}{3}{0mm}%
                                {-1ex plus -.5ex minus -.2ex}%
                                {1ex plus .2ex}%
                                {\normalfont\small\itshape}}
\makeatother
\setcounter{secnumdepth}{0}
\setlength{\parindent}{0pt}
\setlength{\parskip}{0pt plus 0.5ex}
% -----------------------------------------------------------------------

\usepackage{academicons}

\begin{document}

\definecolor{mathBlue}{cmyk}{1,.72,0,.38}
\definecolor{defOrange}{cmyk}{0, 0.5, 1, 0.3}
\definecolor{codeInlineRed}{cmyk}{0, 0.9, 0.9, 0.45}
\definecolor{theoremRed}{cmyk}{0, 0.9, 0.9, 0.45}
% Note: code and thm color are pretty similar, but typically they wont appear in the same file. Both are set to be dim so that the file is not overwhelmed by bright colors (since deff organe is quite bright).

\everymath{\color{mathBlue}}
\everydisplay{\color{mathBlue}}

% for vector notation in this module
\newcommand{\vect}[1]{\pmb{#1}}
\newcommand{\deff}[1]{\textcolor{defOrange}{\textbf{#1}}}
\newcommand{\codein}[1]{\textcolor{codeInlineRed}{\texttt{#1}}}
\newcommand{\citeqn}[1]{\underline{\textit{#1}}}
\newcommand{\thm}[1]{
    \color{theoremRed}{
        \uline{{\textbf{#1}}}
    }
    \color{black}
}

\footnotesize
%\raggedright

\begin{center}
  {\huge\sffamily\bfseries MA2108 Cheatsheet} \huge\bfseries\\
  by Yiyang, AY23/34
\end{center}
\setlength{\premulticols}{0pt}
\setlength{\postmulticols}{0pt}
\setlength{\multicolsep}{1pt}
\setlength{\columnsep}{1.8em}
\begin{multicols}{3}




% -----------------------------------------------------------------------

\section{1. Pre-requisites}
% \subsubsection{}
\thm{Well-Ordering Principle of $\mathbb{N}$} Every non-empty subset $S \in \mathbb{N}$ has a least (smallest) element.





\section{2. The Real Numbers}
\subsection{Algebric Properties, \textasciitilde}
Different types of means 
\begin{itemize}
    \item \deff{Arithmetic Means} $A_n = \frac{1}{n} \sum_{k=1}^n a_k$
    \item \deff{Geometric Means} $G_n = \Big( \prod_{k=1}^n a_k \Big) ^{1/n}$ 
    \item \deff{Harmonic Means} $H_n = n \Big( \sum_{k=1}^n a_k^{-1} \Big)^{-1}$
\end{itemize}
, for $n \in \mathbb{N}_{\ge 2}$ and $a_1, a_2, ..., a_n \in \mathbb{R}$ are positive. For the means, we have the \thm{AM-GM-HM Inequality}:
\[
H_n \le G_n \le A_n
\]
, taking "$=$" iff. $a_1 = ... = a_n$.



\thm{Bernoulli's Inequality} For $x > -1$, we have $(1+x)^n \ge 1 + nx$, for any $n \in \mathbb{N}$.



\thm{Triangle Inequity} $|a+b| \le |a| + |b|$, for all $a, b \in \mathbb{R}$.
\\
Derived: [1] $\bigl| |a| - |b| \bigr| \le |a-b|$, [2] $|a-b| \le |a|+|b|$.



\subsubsection{Neighbourhood}
For any $a \in \mathbb{R}$ and $\epsilon > 0$, the \deff{$\epsilon$-neighbourhood of $a$} is the set:
\[
V_\epsilon(a) = \{ 
x \in \mathbb{R}: |x - a| < \epsilon
\}
\]

\thm{Theorem 2.2.8} For $a \in \mathbb{R}$, if $x \in V_\epsilon(a)$ for every $\epsilon > 0$, then $x = a$.





\subsection{Completeness Properties, \textasciitilde}
For a non-empty $S \subseteq \mathbb{R}$, it is \deff{Bounded Above} (\deff{Bounded Below}) if $S$ has an upper bound (a lower bound). $S$ is \deff{Bounded} if it is bounded above and below, and is \deff{Unbounded}, otherwise.


\smallbreak


For a non-empty $S \subseteq \mathbb{R}$, $u$ is the \deff{Supremum} of $S$ if the following conditions are met, and we denote it as $\sup S$:
\begin{enumerate}
    \item $u$ is an upper bound of $S$.
    \item $\forall v \in \mathbb{R}$, if $v$ is an upper bound of $S$, then $v \ge u$.
\end{enumerate}
For a non-empty $S \subseteq \mathbb{R}$, $w$ is the \deff{Infinum} of $S$ if the following conditions are met, and we denote it as $\inf S$:
\begin{enumerate}
    \item $w$ is a lower bound of $S$.
    \item $\forall v \in \mathbb{R}$, if $v$ is a lower bound of $S$, then $v \le w$.
\end{enumerate}
\underline{Note}: Sup. and Inf. are \textbf{uniquely determined}, if they exist.


Alternative Definition (Similarly for Infinum):

\thm{Lemma 2.3.4} For $u$ an upper bound of $S \subseteq \mathbb{R}$, $u = \sup S$ iff.
\[
\forall \epsilon > 0, \exists s_\epsilon \in S, \ u-\epsilon < s_\epsilon 
\]



\smallbreak


For a non-empty $S \subseteq \mathbb{R}$, $u$ is the \deff{Maximum} (\deff{Minimum}) of $S$, if $u = \sup S$ ($u = \inf S$) and $u \in S$.
\\
\underline{Note}: Sup. and Inf. are not necessarily elements in $S$ (if they exist), but maximum and minimum are.



\thm{Supremum Property of $\mathbb{R}$} Every non-empty subset of $\mathbb{R}$ that has an upper bound has a supremum.



\thm{The Archimedeam Property} If $x \in \mathbb{R}$, then $\exists n_x \in \mathbb{N}$ s.t. $x < n_x$.



\thm{Corollary 2.4.6} If $x > 0$, then $\exists n \in \mathbb{N}$ such that $n-1 \le x <  n$.



\thm{Density Theorems} For $x, y \in \mathbb{R}$ with $x < y$, tehre exists $r \in \mathbb{Q}$ ($z \in \mathbb{R} \backslash \mathbb{Q}$) s.t. $x < r < y$ ($x < z < y$).


% Supremum Property 

% Archimedean 

% Density Thm


\subsection{Intervals}
A sequence of intervals $I_n, n \in \mathbb{N}$ is \deff{Nested} if
\[
I_1 \supseteq I_2 \supseteq ... \supseteq I_n \supseteq I_{n+1} \supseteq ...
\]
\underline{Properties}: [1] If $I_n = [a_n, b_n], n \in \mathbb{N}$ is a nested seq. of closed bounded intervals, then $\exists \xi \in \mathbb{R}$ s.t. $\xi \in I_n, \forall n \in \mathbb{N}$. [2] If $I_n = [a_n, b_n], n \in \mathbb{N}$ satisfying $\inf \{ b_n - a_n: n \in \mathbb{N} \} = 0$, then $\xi$ contained in all $I_n$ is unique.





\section{3. Sequences \& Series}
\subsection{Sequence \& Convergence}
% Seq def
\deff{Sequence} in $\mathbb{R}$: a real-valued function $X: \mathbb{R} \to \mathbb{R}$. We write $x_n = X(n)$ for the $n$-th term of the sequence, and denote the sequence as $(x_n,: n \in \mathbb{N})$.


\smallbreak


A sequence $X = (x_n)$ in $\mathbb{R}$ is \deff{Convergent} to $x \in \mathbb{R}$ iff. for every $\epsilon > 0$, there exists $K = K(\epsilon) \in \mathbb{N}$ s.t.
\[
n \ge K(\epsilon) \implies |x_n - x| < \epsilon
\]
, and we call $x$ the \deff{Limit} of $(x_n)$, denoted as $\lim_{n \to \infty} x_n = x$. A sequence is \deff{Divergent} if it is not convergent.


Technique for proving convergence:
\begin{enumerate}
    \item Express $|x_n - x|$ in terms of $n$ and find a simpler upper bound $L = L(n)$, i.e. $|x_n - x| < L$.
    \item Let $\epsilon > 0$ be arbitrary, find $K \in \mathbb{N}$ s.t. for all $n \ge K$, $L = L(n) < \epsilon$, then
    \[
        n \ge K \implies |x_n - x| < L < \epsilon
    \]
\end{enumerate}


\smallbreak


\thm{Squeeze Theorem} If $x_n \le y_n \le z_n$, for all $n \in \mathbb{N}$ and $\lim_{n \to \infty} x_n = \lim_{n \to \infty} z_n = a$, then
\[
\lim_{n \to \infty} y_n = a
\]



A sequence $X = (x_n)$ is \deff{Bounded} if there exists $M > 0$ such that $|x_n| \le M$ for all $n \in \mathbb{N}$.

% Thm 3.2.2 ~ 3.2.5 See fit


\smallbreak


\thm{Monotone Convergence Theorem} Let $(x_n)$ be a monotone sequence of real numbers, then $(x_n)$ is convergent iff. $(x_n)$ is bounded.
\\
If it is bounded and increasing, then $\lim_{n \to \infty} x_n = \sup \{ x_n: n \in \mathbb{N} \}$. (Similarly for decreasing.)


\smallbreak


For a sequence $(x_n)$, it \deff{tends to $+ \infty$}, i.e. $\lim_{n \to \infty} x_n = + \infty$ if for all $\alpha \in \mathbb{R}$, there exists $K = K(\alpha) \in \mathbb{N}$ such that if $n \ge K(\alpha)$, then $x_n > \alpha$. (Similarly for $\lim_{n \to \infty} x_n = - \infty$.)
\\
A sequence $(x_n)$ is \deff{Properly Divergent} if $\lim_{n \to \infty} x_n = \pm \infty$.




\subsection{Subsequences}
% Def
A \deff{Subsequence} of $X = (x_n)$ is $X' = (x_{n_k})$:
\[
X' = (x_{n_1}, x_{n_2}, ..., x_{n_3})
\]
, where $n_1 < n_2 < ... < n_k < ... $ is a strictly increasing sequence in $\mathbb{N}$. \underline{Note}: $n_k \ge n, \forall k$.


\thm{Theorem 3.4.2} If $(x_n)$ converges to $x$, then any subsequence $(x_{n_k})$ also converges to $x$, 
\[
\lim_{n_k \to \infty} x_{n_k} = \lim_{k \to \infty} x_{n_k} = x
\]


\thm{Theorem 3.4.5} If $(x_n)$ has either of the following properties, it is divergent: [1] $(x_n)$ has two convergent subsequences with different limits. [2] $(x_n)$ is unbounded.


\thm{Theorem 3.4.7} Every sequence has a monotone subsequence.


\thm{Bolzano-Weierstrass Theorem} Every bounded sequence has a convergent subsequence.




\subsection{Cauchy Sequences}
A \deff{Cauchy Sequence} $(x_n)$ is a sequence where for all $\epsilon > 0$, there exists $H = H(\epsilon) \in \mathbb{N}$ such that
\[
\forall n, m \in \mathbb{N}, n, m \ge H \implies |x_n - x_m| < \epsilon
\]

\thm{Cauchy Criterion} A sequence is convergent iff. it is Cauchy.


\smallbreak


A \deff{Contractive Sequence} $(x_n)$ is a sequence where there exists $C \in (0, 1)$ s.t. 
\[
|x_{n+2} - x_{n+1}| \le C |x_{n+1} - x_n|, \ \forall n \in \mathbb{N}
\]


\thm{Theorem 3.5.8} Every contractive sequence is Cauchy.




\subsection{Infinite Series}
For $(x_n)$, its \deff{(Infinite) Series} is sequence $(s_n)$, where $s_n = \sum{k=1}^n x_k$ is called a \deff{Partial Sum} of the series, and $x_k$ is a \deff{Term}.

Tests for infinite series' convergence:
\begin{itemize}
    \item \deff{$n$-th Term Test} - If $\sum x_n$ converges, then $\lim_{n\to\infty} x_n = 0$.
    
    \item Cauchy Criterion Test
    
    \item \deff{Partial Sum Bounded Test}, for series w. non-negative terms - Suppose $x_n \ge 0, \forall n \in \mathbb{N}$, then $\sum_{x_n}$ converges iff. $(s_n)$ is bounded.
    
    \item \deff{Comparison Test} - For $(x_n), (y_n)$ with some $K \in \mathbb{N}$, s.t. $n \ge K \implies 0 \le x_n \le y_n$. Then [1] $\sum y_n$ converges $\implies$ $\sum x_n$ converges, and [2] $\sum x_n$ diverges $\implies$ $\sum y_n$ diverges.
    
    \item \deff{Limit Comparison Test} - For \textbf{strictly positive} $(x_n), (y_n)$ with limit $r = \lim_{n \to \infty} (\frac{x_n}{y_n})$. Then [1] if $r = 0$, $\sum y_n$ converges $\implies$ $\sum x_n$ converges. [2] if $r > 0$, $\sum y_n$ converges iff $\sum x_n$ converges.
\end{itemize}





\subsection{Absolute Convergence}
% Def
Series $\sum x_n$ is \deff{Absolutely Convergent} if series $\sum |x_n|$ is convergent. A series is \deff{Conditionally Convergent} if it is convergent but not absolutely convergent.

Tests for absolutely convergence:
\begin{itemize}
    \item Limit Comparison Test - Consider convergence of positive sequences $|x_n|$ and $|y_n|$ if $(x_n), (y_n)$ non-negative.
    \item \deff{Root Test} - For $(x_n)$, [1] if $\exists r \in \mathbb{R}, 0 < r < 1$ and $K \in \mathbb{N}$ s.t. $|x_n|^{1/n} \le r, \ \forall n \ge K$, then $\sum x_n$ is abs. convergent. [2] If $\exists r \in \mathbb{R}, r > 1$ and $K \in \mathbb{N}$ s.t. $|x_n|^{1/n} \ge r > 1, \ \forall n \ge K$, then $\sum x_n$ is \textbf{divergent}.
    \item \deff{Ratio Test} - For $(x_n)$ nonzero, [1] if $\exists r \in \mathbb{R}, 0 < r < 1$ and $K \in \mathbb{N}$ s.t. $|\frac{x_{n+1}}{x_n}| \le r, \ \forall n \ge K$, then $\sum x_n$ is abs. convergent. [2] If $\exists K \in \mathbb{N}$ s.t. $|\frac{x_{n+1}}{x_n}| \ge 1, \ \forall n \ge K$, then $\sum x_n$ is \textbf{divergent}.
\end{itemize}

% Abs convg tests
% TODO: check if finishes




\section{4. Limits}
For $A \subseteq \mathbb{R}$, $c$ is the \deff{Cluster Point} of $A$ iff. $\forall \delta > 0$, there exists $x \in A$ s.t. $0< |x-c| < \delta$.

\thm{Theorem 4.1.2} (Sequential Criterion) $c \in \mathbb{R}$ is a cluster point of $A$ iff. there exists a sequence $(a_n)$ in $A$ s.t. $\lim a_n = c$ and $a_n \neq c, \forall n \in \mathbb{N}$.

\deff{Limit} of a function $f: A \to \mathbb{R}$ at $c \in A$, $L = \lim_{x \to c} f(x)$ iff. $\forall \epsilon > 0$, there exists $\delta = \delta(\epsilon) > 0$ s.t. $0 < |x - c| < \delta \implies |f(x) - L| < \epsilon$.

\thm{Theorem 4.1.8} (Sequential Criterion) $\lim_{x \to c} f(x) = L$ iff. for every seq. $(x_n)$ in $A$ w. $lim_{n \to \infty} x_n = c$ and $x_n \neq c, \forall n \in \mathbb{N}$, $lim_{n \to \infty} f(x_n) = L$.


\smallbreak


For $f: A \to \mathbb{R}$ and $c$ a cluster point of $A$, $f$ is \deff{Bounded} on a neighbourhood of $c$ if $\exists V_\delta(c)$ and constant $M > 0$ s.t. $|f(x)| < M, \forall x \in A \cap V_\delta(c)$.


\thm{Theorem 4.2.2} If $f: A \to \mathbb{R}$has a limit at cluster point $c$, then $f$ is bounded on some neighbourhood of $c$.


\smallbreak


\thm{Theorem 4.2.9} If $\lim_{x\to c} f(x) > 0$, then $\exists V_\delta(c)$ s.t. $f(x) > 0, \forall x \in A \cap V_\delta(c), x \neq c$.
\\
Similar statements for $f(x) < 0$.


\smallbreak



For $A\subseteq \mathbb{R}$, function $f: A \to \mathbb{R}$ and a cluster point $c$ of $A$, \deff{Right Hand Limit} $L_{+}  = \lim_{x \to c^+} f(x)$ iff. $\forall \epsilon > 0, \exists \delta > 0, x \in V_{\delta}(c) \ \{ c \} \implies f(x) \in V_{\epsilon}(L_{+})$.
\\
Similar definition for \deff{Left-Hand Limit} $L_{-}  = \lim_{x \to c^-} f(x)$.
\\
Sequential Criteria for One-sided Limits exist.


\thm{Theorem 4.3.3} $\lim_{x \to c} f(x) = L$ iff. both $\lim_{x \to c^+} f(x)$ and $\lim_{x \to c^-} f(x)$ exist and 
\[
\lim_{x \to c^+} f(x) = \lim_{x \to c^-} f(x) = L
\]




\section{5. Continuous Functions}
\subsection{Continuity}
For $A \subseteq \mathbb{R}$ and $f: A \to \mathbb{R}$, $f$ is \deff{Continuous} at $c \in A$ iff $\forall \epsilon > 0, \exists \delta > 0$ s.t. $x \in V_\delta(c) \implies f(x) \in V_\epsilon(f(c))$.
\\
$f$ is continuous at $c \in A$ iff. $\lim_{x \to c} f(x) = f(c)$.

(Sequential Criterion) $f: A \to \mathbb{R}$ is continuous at $x = c$ iff. for every sequence $(x_n)$ in $A$ s.t. $x_n \to c$, we have $f(x_n) \to f(c)$.



\subsection{Continuous Function on Intervals}
\thm{Boundedness Theorem} If $f$ is continuous on $[a, b]$, then $f$ is bounded on $[a, b]$.
\\
\underline{Note}: It only applies to \textbf{closed bounded} intervals.




\thm{Max-Min Theorem} If $f$ is continuous on $[a, b]$, then $f$ has an absolute maximum and an absolute minimum on $[a, b]$.




\thm{Location of Roots Theorem} If $f$ is continuous on $[a, b]$ and $f(a)f(b) < 0$, then there exists a point $c$ in $(a, b)$ s.t. $f(c) = 0$.




\thm{Bolzano's Intermediate Value Theorem} For interval $I$ and function $f$ continuous on $I$, and $a, b \in I$ with $f(a) \le f(b)$,  then for any $k \in [f(a), f(b)]$, $\exists c \in I$ s.t. $f(c) = k$.




\thm{Preservation of Closed Intervals Theorem} For $f$ continuous on $[a, b]$,
\[
f([a, b]) := \{ f(x): x \in [a, b] \} = [m, M]
\]
, with $m = \inf f([a, b])$ and $M = \sup f([a, b])$.

% TODO: maybe more




\subsection{Monotonicity \& Bijectivity}
A function $f: A \to \mathbb{R}$ is \deff{Increasing} (\deff{Decreasing}) on $A$ if $\forall x_1, x_2 \in A$, $x_1 \le x_2 \implies f(x_1) \le f(x_2)$ ($x_1 \le x_2 \implies f(x_1) \ge f(x_2)$).
\\
$f$ is \deff{Monotone} if it is increasing or decreasing.
\\
\deff{Strictly \textasciitilde}: $x_1 < x_2 \implies f(x_1) < f(x_2)$ and so on.


\smallbreak


For a function $f: A \to B$, it is
\begin{itemize}
    \item \deff{Injective} (\deff{One-One}), iff $\forall x_1 \neq x_2 \in A$, $f(x_1) \neq f(x_2)$.
    \item \deff{Surjective}, iff $f(A) = B$.
    \item \deff{Bijective}, iff it is injective and surjective.
\end{itemize}





\subsection{Uniform Continuity}
For $A \subseteq \mathbb{R}$ and $f: A \to \mathbb{R}$, $f$ is \deff{Uniformly Continuous} on $A$ if for all  $\epsilon > 0$, there exists $\delta = \delta(\epsilon) > 0$ s.t.
\[
\forall x, y \in A, |x-y| < \delta \implies |f(x) - f(y)| < \epsilon
\]
i.e. $\delta = \delta(\epsilon)$ is independent of $x, y \in A$.
\\
\underline{Note}: $f$ is \textbf{not uniformly continuous} on $A$ iff. $\exists \epsilon_0 > 0$ s.t. $\forall \delta > 0, \exists x_\delta, y_\delta \in A$ with $|x_\delta - y_\delta| < \delta$ and $|f(x_\delta) - f(y_\delta)| \ge \epsilon$.

Sequential Criterion
\begin{itemize}
    \item Uniformly continuous - For any $(x_n), (y_n)$ in $A$ with $\lim_{n\to \infty} x_n - y_n = 0$, we have $\lim_{n \to \infty} f(x_n) - f(y_n) = 0$.
    \item Not uniformly continuous - There exists $\epsilon_0 > 0$ and $(x_n), (y_n)$ in $A$, $\lim_{n\to \infty} x_n - y_n = 0$ and $\lim_{n \to \infty} f(x_n) - f(y_n) \ge \epsilon_0$.
\end{itemize}


\thm{Uniform Continuity Theorem} If $f$ is continuous on a \textbf{closed bounded} interval $[a, b]$, then it is uniformly continuous on $[a, b]$.


\smallbreak


A function $f: A \to \mathbb{R}$ is a \deff{Lipschitz Function} on $A$ iff. there exists $K > 0$ s.t.
\[
|f(x) - f(y)| \le K |x-y|, \forall x, y \in A
\]

\thm{Theorem 5.4.5} If $f: A \to \mathbb{R}$ is a Lipschitz function, then $f$ is uniformly continuous on $A$.


\thm{Theorem 5.4.7} If $f: A \to \mathbb{R}$ is uniformly continuous on $A$ and $(x_n)$ a Cauchy sequence in $A$, then $(f(x_n))$ is a Cauchy sequence in $\mathbb{R}$.
\\
i.e. Uniformly continuous functions preserve Cauchy sequences.



\thm{Continuous Extension Theorem} $f$ is uniformly continuous on interval $(a, b)$ iff. it can be defined at the endpoints $a$ and $b$ s.t. the extended function is continuous on $[a, b]$.
\\
\underline{Note}: Define $f(a) = \lim_{x \to a^+} f(x)$ and $f(b) = \lim_{x \to b^-} f(x)$ provided both limits exist.






\subsection{Jumps}
\thm{Theorem 5.6.1} For interval $I \subseteq \mathbb{R}$ and increasing function $f: I \to \mathbb{R}$, $c \in I$ not an endpoint, then
\begin{itemize}
    \item $\lim_{x \to c^-} f(x) = \sup\{ f(x): x \in I, x < c \}$
    \item $\lim_{x \to c^+} f(x) = \inf\{ f(x): x \in I, x > c \}$
\end{itemize}

\deff{Jump} of $f$ at $c$ is defined as $$j_f (c) = \lim_{x \to c^+} f(x) - \lim_{x \to c^-} f(x)$$, and at endpoints, $j_f(a) = \lim_{x \to a^+} f(x) - f(a)$ and so on for $b$.



\thm{Theorem 5.6.4} For interval $I \subseteq \mathbb{R}$ and $f: I \to \mathbb{R}$ monotone on $I$, the set of points $D \subseteq I$ at which $f$ is discontinuous is a countable set.


\thm{Continuous Inverse Theorem} For interval $I \subseteq \mathbb{R}$ and $f: I \to \mathbb{R}$ strictly monotone and continuous, the inverse $f^{-1}$ exists and is also strictly monotone and continuous on $J = f(I)$.




\section{11. Topology Introduction}
% TODO: whether we can use generalised metric definition later.
\subsection{Open \& Closed Sets}
A set $V$ is the \deff{Neighbourhood} of a point $x \in \mathbb{R}$ iff there exists $\epsilon > 0$ s.t. $V_\epsilon(x) \subseteq V$.
\\
A subset $G \subseteq \mathbb{R}$ is \deff{Open} in $\mathbb{R}$ iff. for each $x \in G$, there exists $\epsilon_x > 0$ s.t. $V_{\epsilon_x}(x) \subseteq G$.
\\
A subset $F \subseteq \mathbb{R}$ is \deff{Closed} in $\mathbb{R}$ if the complement $C(F) = \mathbb{R} \setminus F$ is open in $\mathbb{R}$.
\\
\underline{Note}: [1] $\mathbb{R}$ and $\emptyset$ are both open and closed. [2] $\mathbb{Z}$ is closed but not open. [3] $\mathbb{Q}$ is neither open nor closed.


Open \& Closed Set Properties
\begin{itemize}
    \item Open: [1] Union of any collection of open subsets is open. [2] Intersection of finitely many open subsets is open.
    \item Closed: [1] Intersection of any collection of closed subsets is closed. [2] Union of finitely many closed subsets is closed.
\end{itemize}


\thm{Characterisation of Closed Sets Theorem} A subset $F \subseteq \mathbb{R}$ is closed iff. any convergence sequence $(x_n)$ in F has $\lim_{n \to \infty} x_n \in F$.


\thm{Theorem 11.1.8} A subset $F \subseteq \mathbb{R}$ is closed iff. it contains all its cluster points.

\thm{Theorem 11.1.9} A subset $G \subseteq \mathbb{R}$ is open iff it is the union of countably many disjoint open intervals in $\mathbb{R}$.


\smallbreak


\thm{Global Continuity Theorem} A function $f: A \to \mathbb{R}$ is continuous on $A$ iff. for every open set $G \subseteq \mathbb{R}$, there exists open set $H \subseteq \mathbb{R}$ such that $H \cap A = f^{-1}(G)$ where $f^{-1}(G) = \{ x \in A: f(x) \in G \}$.

\thm{Corollary 11.3.3} Function $f: \mathbb{R} \to \mathbb{R}$ is continuous iff. $f^{-1}(G)$ is open in $\mathbb{R}$ for every open $G$.



\subsection{Metric Space}
A \deff{Metric} on a set $S$ is a function $d: S \times S \to \mathbb{R}$ that satisfies
\begin{itemize}
    \item \deff{Positivity} $d(x, y) \ge 0, \forall x, y \in S$
    \item \deff{Definiteness} $d(x, y) = 0 \iff x = y$
    \item \deff{Symmetry} $d(x, y) = d(y, x), \forall x, y \in S$
    \item \deff{Triangle Inequality} $d(x,y) \le d(x, z) + d(z, y), \forall x, y, z \in S$
\end{itemize}
A \deff{Metric Space} $(S, d)$ is a set $S$ with a metric $d$ on $S$.


Generalised definition for a metric space $(S, d)$
\begin{itemize}
    \item \textbf{Neighbourhood}: $V_\epsilon(x_0) = \{ x \in S: d(x, x_0) < \epsilon  \}$ for $\epsilon > 0$ and $x_0 \in S$

    \item \textbf{Boundedness} of $K \subseteq S$: $\exists M > 0, x_0 \in S, d(x, x_0) \le M, \forall x \in K$.
    
    \item \textbf{Convergence} to $x \in S$ of sequence $(x_n)$: $\forall \epsilon > 0, \exists K = K(\epsilon) \in \mathbb{N}, n \ge K \implies x_n \in V_{\epsilon}(x)$
    
    \item  \textbf{Continuity} of $f: S_1 \to S_2$ at $c \in S_1$: $\forall \epsilon > 0, \exists \delta > 0, d_1(x, c) < \delta \implies d_2(f(x), f(c)) < \epsilon$.
    
    \item  \textbf{Open \& Closed Set}
\end{itemize}





\subsection{Compact Set}
For a metric space $S$, an \deff{Open Cover} of a subset $A \subseteq S$ is a collection $\mathcal{G} = \{ G_\lambda: \lambda \in \Lambda \}$ of open subsets of $S$ satisfying
\[
A \subseteq \cup_{\lambda \in \Lambda} G_\lambda
\]
If $\mathcal{G}' \subseteq \mathcal{G}$ whose union also contains $A$, then $\mathcal{G}'$ is a \deff{Subcover} of $\mathcal{G}$.
\\
If $\mathcal{G}'$ is finite, it is a \deff{Finite Subcover} of $\mathcal{G}$.


For a metric space $S$, a subset $K \subseteq S$ is \deff{Compact} iff. for every open cover of $K$ there is a finite subcover.


\thm{Heine-Borel Theorem} For a metric space $(S, d)$, a subset $K \subseteq S$ is compact iff. it is closed and bounded.

\thm{Bolzano-Weierstrass Theorem} A bounded sequence in $(S, d)$ has a convergent subsequence.

\thm{Theorem 11.2.6} $K \subseteq S$ is compact iff. every sequence in $K$ has a subsequence that converges to a point in $K$.


\thm{Preservation of Compactness Theorem} If $(S, d)$ is compact and $f: S \to \mathbb{R}$ is continuous, then $f(S)$ is compact in $\mathbb{R}$.


\smallbreak


A subset $U \subseteq S$ is \deff{Disconnected} iff. $U$ has an open cover $\{ A, B\}$ s.t. $A \cap B \cap U = \emptyset$ and $A \cap U = \emptyset, B \cap U = \emptyset$. Otherwise it is \deff{Connected}.
\\
\underline{Note}: $E \subseteq \mathbb{R}$ is connected iff $E$ is an interval, i.e. $x, y \in E, x < y \implies [x, y] \in E$.


\thm{Intermediate Value Theorem} For $f: S \to \mathbb{R}$ continuous, if $E$ is connected then $f(E)$ is connected.



\noindent\rule{8cm}{0.4pt}




\section{Intermediate Results \& Lemmas}
\citeqn{(Tut10Qn4)} For any two functions $f, g: \mathbb{R} \to \mathbb{R}$ continuous on $\mathbb{R}$, if $f(x) = g(x), \forall x \in \mathbb{Q}$, then $f(x) = g(x), \forall x \in \mathbb{R}$.


Useful statements
\begin{itemize}
    \item $\sum_{i=1}^\infty \frac{1}{n^2} = \frac{\pi^2}{6}$.
\end{itemize}


\end{multicols}
\end{document}
