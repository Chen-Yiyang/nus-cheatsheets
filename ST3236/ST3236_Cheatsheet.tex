%%%%%%%%%%%%%%%%%%%%%%%%%%%%%%%%%%%%%%%%%%%%%%%%%%%%%%%%%%%%%%%%%%%%%%
% Original Source: Dave Richeson (divisbyzero.com), Dickinson College
% Modified By: Chen Yiyang
% 
% A one-size-fits-all LaTeX cheat sheet. Kept to two pages, so it 
% can be printed (double-sided) on one piece of paper
% 
% Feel free to distribute this example, but please keep the referral
% to divisbyzero.com
% 
% Guidance on the use of the Overleaf logos can be found here:
% https://www.overleaf.com/for/partners/logos 
%%%%%%%%%%%%%%%%%%%%%%%%%%%%%%%%%%%%%%%%%%%%%%%%%%%%%%%%%%%%%%%%%%%%%%

\documentclass[10pt,landscape,letterpaper]{article}
\usepackage{amssymb}
\usepackage{amsmath}
\usepackage{amsthm}
\usepackage{physics} % for vectors
%\usepackage{fonts}
\usepackage{multicol,multirow}
\usepackage{spverbatim}
\usepackage{graphicx}
\usepackage{ifthen}
\usepackage[landscape]{geometry}
\usepackage[colorlinks=true,urlcolor=olgreen]{hyperref}
\usepackage{booktabs}
\usepackage{fontspec}
\setmainfont[Ligatures=TeX]{TeX Gyre Pagella}
\setsansfont{Fira Sans}
\setmonofont{Inconsolata}
\usepackage{unicode-math}
\setmathfont{TeX Gyre Pagella Math}
\usepackage{microtype}

\usepackage{empheq}

% new:
\def\MT@is@uni@comp#1\iffontchar#2\else#3\fi\relax{%
  \ifx\\#2\\\else\edef\MT@char{\iffontchar#2\fi}\fi
}
\makeatother

\ifthenelse{\lengthtest { \paperwidth = 11in}}
    { \geometry{margin=0.4in} }
	{\ifthenelse{ \lengthtest{ \paperwidth = 297mm}}
		{\geometry{top=1cm,left=1cm,right=1cm,bottom=1cm} }
		{\geometry{top=1cm,left=1cm,right=1cm,bottom=1cm} }
	}
\pagestyle{empty}
\makeatletter
\renewcommand{\section}{\@startsection{section}{1}{0mm}%
                                {-1ex plus -.5ex minus -.2ex}%
                                {0.5ex plus .2ex}%x
                                {\sffamily\large}}
\renewcommand{\subsection}{\@startsection{subsection}{2}{0mm}%
                                {-1explus -.5ex minus -.2ex}%
                                {0.5ex plus .2ex}%
                                {\sffamily\normalsize\itshape}}
\renewcommand{\subsubsection}{\@startsection{subsubsection}{3}{0mm}%
                                {-1ex plus -.5ex minus -.2ex}%
                                {1ex plus .2ex}%
                                {\normalfont\small\itshape}}
\makeatother
\setcounter{secnumdepth}{0}
\setlength{\parindent}{0pt}
\setlength{\parskip}{0pt plus 0.5ex}
% -----------------------------------------------------------------------

\usepackage{academicons}

\begin{document}

\definecolor{mathBlue}{cmyk}{1,.72,0,.38}
\definecolor{defOrange}{cmyk}{0, 0.5, 1, 0.3}
\definecolor{codeInlineRed}{cmyk}{0, 0.9, 0.9, 0.45}

\everymath{\color{mathBlue}}
\everydisplay{\color{mathBlue}}

% for vector notation in this module
\newcommand{\vect}[1]{\boldsymbol{#1}}
\newcommand{\deff}[1]{\textcolor{defOrange}{\textbf{#1}}}
\newcommand{\codein}[1]{\textcolor{codeInlineRed}{\texttt{#1}}}
\newcommand{\citeqn}[1]{\underline{\textit{#1}}}

\footnotesize
%\raggedright

\begin{center}
  {\huge\sffamily\bfseries ST3236 Cheatsheet} \huge\bfseries\\
  by Yiyang, AY21/22
\end{center}
\setlength{\premulticols}{0pt}
\setlength{\postmulticols}{0pt}
\setlength{\multicolsep}{1pt}
\setlength{\columnsep}{1.8em}
\begin{multicols}{3}


% -----------------------------------------------------------------------

\section{Probability Theory, Review}
\subsection{Boole's Inequality}
For any events $A_1, A_2, ...$,
\[
\mathbb{P} \Big( \bigcup_{n \geq 1} A_n \Big) \leq \sum_{n \geq 1} \mathbb{P} (A_n)
\]

\subsection{Conditional Expectation, Properties}
\begin{itemize}
    \item Linearity: $\mathbb{E} (aX_1 + bX_2 | Y) = a \mathbb{E} (X_1 | Y) + b \mathbb{E} (X_2 | Y)$
    \item Law of Iterated Expectation: $\mathbb{E} ( \mathbb{E} (X|Y) ) = \mathbb{E} (X)$
    \item Tower Property: $\mathbb{E} ( \mathbb{E} (X|Y,Z)|Y ) = \mathbb{E} (X|Y)$
    \item Independence: $\mathbb{E} (X|Y) = \mathbb{E} (X)$, for $X$ and $Y$ independent.
\end{itemize}




\section{Markov Chain (MC) Basics}
A (discrete-time) \deff{Markov Chain} (MC) is a stochastic process $ \{ X_n \}_{n=0}^{\infty}$ that satisfies,
\begin{align*}
& \mathbb{P} (X_{n+1} = t_{n+1} | X_n = t_n, X_{n-1} = t_{n-1}, ..., X_1 = t_1) \\
= & \mathbb{P} (X_{n+1} = t_{n+1} | X_n = t_n) 
\end{align*}
,whenever $\mathbb{P} (X_n = t_n, X_{n-1} = t_{n-1}, ..., X_1 = t_1) > 0$. This property is also called the \deff{Markovian Property}.
\smallskip

A \deff{Time-homogeneous} Markov Chain is one where the conditional probability $\mathbb{P} (X_{n+1} = j | X_n = i)$ does not depend on $n$. Equivalently, it means for all $n \geq 0$ and $i, j \in S$,
\[
\mathbb{P} (X_{n+1} = j | X_n = i) = \mathbb{P} (X_1 = j | X_0 = i)
\]
For time-homogeneous MCs, $p_{ij} = \mathbb{P} (X_1 = j | X_0 = i)$ is the \deff{1-step transition probability}, and $P = ((p_{ij}))_{i,j \in S}$ is the \deff{transition matrix}. Similarly, $p_{ij}^{(k)} = \mathbb{P} (X_k = j | X_0 = i)$ is the \deff{k-Step Transition Probability}.

\subsection{Chapman-Komogorov Equation}
\[
p_{ij}^{(k)} = \sum_{i_1, ..., i_{k-1} \in S} p_{ii_1}p_{i_1i_2}...p_{i_{k-1}j}
\]
Equivalently, it means $P^{(k)} = P^k$.




\section{Accessing States, MC}
\subsection{Accessible States}
For states $i$, $j$ of a MC, $j$ is \deff{accesible} from $i$, $i \to j$, if there exists $k \geq 0$ such that,
\[
p_{ij} = \mathbb{P} (X_k = j | X_0 = i) > 0
\]
\\
States $i$ and $j$ \deff{communicate}, i.e. $i \leftrightarrow j$ iff. $i \to j \land j \to i$.
\medskip

$\leftrightarrow$ is a \textbf{Equivalence Relation}. The equivalent classes of $S$ partitioned by it are called \deff{Communicating classes}.
\medskip

Notes:
\begin{itemize}
    \item $i \leftrightarrow i$ as $p_{ii}^{(0)} = 0$ for all $i \in S$
    \item For $i \in S_i$ and $j \in S_j$, it is possible that $i \to j$.
    \item For $i \in S_i$ and $j \in S_j$, $i \to j \implies j \not\to i$.
\end{itemize}

A MC is \deff{irreducible} if there is only one communicating class for the state space. Otherwise, the MC is \deff{reducible}.


\subsection{Essential States}
A state $i$ is \deff{essential} if for every state $j$ it satisfies
\[
i \to j \implies j \to i
\]
A state that is not essential is called \deff{inessential}.
\medskip

Properties of Essential States
\begin{itemize}
    \item If $i$ essential and $i \to j$ then $j$ is essential.
    \item All states in one communicating class are either all essential or all inessential.
    \item A \textbf{finite state} MC must have at least one essential state.
    \item An absorbing state is essential.
\end{itemize}


\end{multicols}
\end{document}
